\usepackage[utf8]{inputenc}

\usepackage[english]{babel}
\newcommand{\acknowledgments}{@undefined}

% English
\addto\captionsenglish{\renewcommand{\acknowledgments}{Acknowledgments}}

% Portuguese
\addto\captionsportuguese{\renewcommand{\acknowledgments}{Agradecimentos}}
\addto\captionsportuguese{\renewcommand{\nomname}{Lista de S\'{i}mbolos}} % Nomenclatura

% Use Arial font as default
\renewcommand{\rmdefault}{phv}
\renewcommand{\sfdefault}{phv}

% Define cover page fonts
\def\FontLn{% 16 pt normal
  \usefont{T1}{phv}{m}{n}\fontsize{16pt}{16pt}\selectfont}
\def\FontLb{% 16 pt bold
  \usefont{T1}{phv}{b}{n}\fontsize{16pt}{16pt}\selectfont}
\def\FontMn{% 14 pt normal
  \usefont{T1}{phv}{m}{n}\fontsize{14pt}{14pt}\selectfont}
\def\FontMb{% 14 pt bold
  \usefont{T1}{phv}{b}{n}\fontsize{14pt}{14pt}\selectfont}
\def\FontSn{% 12 pt normal
  \usefont{T1}{phv}{m}{n}\fontsize{12pt}{12pt}\selectfont}

% > set the page margins (2.5cm minimum in every side, as per IST rules)
% > allow setting line spacing (line spacing of 1.5, as per IST rules)
\usepackage{geometry}
\geometry{verbose,tmargin=2.5cm,bmargin=2.5cm,lmargin=2.5cm,rmargin=2.5cm}
\usepackage{setspace}
\setstretch{1.5}

\usepackage{graphicx}

\usepackage{amssymb}
\usepackage{mathtools}

\usepackage[pdftex]{hyperref}       % enhance documents that are to be
                                    % output as HTML and PDF
\hypersetup{colorlinks,             % color text of links and anchors,
                                    % eliminates borders around links
            linkcolor=black,        % color for normal internal links
            anchorcolor=black,      % color for anchor text
            citecolor=black,        % color for bibliographical citations
            filecolor=black,        % color for URLs which open local files
            menucolor=black,        % color for Acrobat menu items
            pagecolor=black,        % color for links to other pages
            urlcolor=black,         % color for linked URLs
	          bookmarksopen=false,    % don't expand bookmarks
	          bookmarksnumbered=true, % number bookmarks
	          pdftitle={Thesis},
            pdfauthor={Rodrigo André Moreira Bernardo},
            pdfstartview=FitV,
            pdfdisplaydoctitle=true,
            pdfpagemode=UseOutlines
}

% Flexible bibliography support.
\usepackage[numbers, sort&compress]{natbib}

% This package lets us display pseudocode in various fashions
\usepackage[linesnumbered,ruled,vlined,algochapter]{algorithm2e}


\usepackage[acronym,toc]{glossaries-extra} % must be after hyperref
\makeglossaries
\setabbreviationstyle[acronym]{long-short}
\loadglsentries[main]{Glossary}

% Document management
\usepackage{import}

% Support for highlighting and popup notes (works with Adobe Acrobat Reader).
\usepackage[author={Rodrigo}]{pdfcomment}
\newcommand{\note}[3]{\pdfmarkupcomment[color=#1]{#2}{#3}}
\newcommand{\todo}[2]{\note{yellow}{#1}{#2}}
\newcommand{\fixme}[2]{\note{red}{#1}{#2}}
\newcommand{\new}[1]{\pdfmarkupcomment[color=green]{#1}{}}

\usepackage{xcolor}

% Diagrams
\usepackage{tikz}
\usetikzlibrary{arrows,backgrounds,positioning, fit,shapes}

\usepackage{pifont}
\newcommand{\cmark}{\ding{51}}
\newcommand{\xmark}{\ding{55}}

\usepackage{cprotect}

% Nice, clean tables.
\usepackage{booktabs}

\usepackage{verbatimbox}

% Newtheorem
\newtheorem{definition}{Definition}
\newtheorem{example}{Example}
\numberwithin{example}{chapter}

% Listings

\usepackage{listings}
\definecolor{dkgreen}{rgb}{0,0.6,0}
\definecolor{gray}{rgb}{0.5,0.5,0.5}
\definecolor{mauve}{rgb}{0.58,0,0.82}

\lstset{
  % frame=tb,
  % language=Haskell,
  % aboveskip=3mm,
  % belowskip=3mm,
  showstringspaces=false,
  columns=flexible,
  basicstyle={\ttfamily},
  numbers=none,
  numberstyle=\tiny\color{gray},
  keywordstyle=\color{blue},
  commentstyle=\color{dkgreen},
  stringstyle=\color{mauve},
  breaklines=true,
  breakatwhitespace=true,
  tabsize=3,
  literate={->}{$\rightarrow$}{2}
}

% Context-Free Grammars
\newcommand*\OR{\ |\ }

% Inline lists
\usepackage[inline, shortlabels]{enumitem}

% Multicols
\usepackage{multicol}

% Math binary operators
\newcommand{\strconcat}{\mathbin{++}}

\usepackage[multiple]{footmisc}