\section*{Resumo}

\addcontentsline{toc}{section}{Resumo}

Síntese de programas é o problema de gerar automaticamente implementações
concretas de programas a partir de especificações de alto nível que definem a
intenção do utilizador.
OutSystems é uma plataforma \textit{low-code} para desenvolvimento rápido de
aplicações e integração simples com sistemas já existentes, com recurso a
programação visual ou textual.
Neste trabalho, focamo-nos no lado da programação textual.
Abordamos o problema de sintetizar expressões OutSystems em um ambiente
onde as especificações são fornecidas na forma de exemplos de entrada-saída,
com foco em expressões que manipulam dados dos tipos inteiro e texto.
Fazemos um estudo do estado da arte em síntese de programas e implementamos dois
sintetizadores baseados em componentes e numa arquitectura de síntese inductiva
guiada a oráculos.
Ambos os sintetizadores aplicam uma mistura de satisfação de restrições com
procura enumerativa básica, diferindo um do outro na quantidade de trabalho que
colocam na fase de satisfação de restrições.
Ambos os sintetizadores são aferidos e comparados com o SyPet num conjunto de
problemas do mundo real fornecidos pela OutSystems, demonstrando resultados
interessantes para programas até tamanho 4.

\vfill

\textbf{\Large Palavras-chave:} Síntese de Programas, Síntese Baseada em
Componentes, Satisfação de Restrições, Satisfação Módulo Teorias

\cleardoublepage
