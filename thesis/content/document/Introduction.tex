\chapter{Introduction}
\label{chapter:introduction}

\todo{Program synthesis is the problem of automatically generating
implementations from high-level specifications.}{add ref} \todo{Amir Pnuelli,
former computer scientist and Turing Award, described it as ``one of the most
central problems in the theory of programming''}{add ref}, and it has been
\todo{portrayed as one of the holy grails of computer science}{solar-lezama phd
thesis, gulwani et al overview}.

It is easy to understand why: if only we could tell the computer \textit{what to
do} and let it figure out \textit{how to do} it, the task of programming would
be so much easier!

Program synthesis is a hard problem, though. \todo{Programming is a task that is
hard for humans}{add ref 45 solar-lezama thesis} and, given its generality,
there is no reason to believe it should be any easier for computers. Computers
lack algorithmic insight and domain expertise. Moreover, the challenge is
actually twofold: we need to find out both how to tackle the intractibility of
the program space, and how to accurately capture user intent~\cite{Gulwani2017}.

% 1. Missing an explanation of why the search space is big.

It is important to have in mind that program synthesis is not a panacea for
solving problems in computer programming. For example, while we are interested
in properties besides functional correctness, such as efficiency or
succinctness, it is impossible for program synthesis to eliminate all sources of
bugs. Particularly, it cannot solve problems originating from bad specifications
that come from a bad understanding of the problem domain.

Writing specifications is, indeed, a delicate process. This might be better
understood with a simple, non-trivial, example.

\section{Example: Sorting}
\label{sec:sorting-example}

In order to exemplify how the interaction between the user and the computer
(from now on referred to as the ``synthesizer'') might occur, let us suppose we
are interested in developing a sorting procedure, \lstinline{sort}, for lists of
integers:

% \begin{lstlisting}[language=Haskell]
%   sort :: [Int] -> [Int]
%   sort [] = []
%   sort (x:xs) = insert x (sort xs)

%   insert :: Int -> [Int] -> [Int]
%   insert x [] = [x]
%   insert x (y::ys) = if x <= y then (x:y:ys) else (y:insert x ys)
% \end{lstlisting}

\begin{lstlisting}
  sort: (xs: List Int) -> (xs': List Int)
  sort([])    = []
  sort(x::xs) = insert(x, sort(xs))

  insert: (x: Int, xs: List Int) -> (xs': List Int)
  insert(x, [])    = [x]
  insert(x, y::ys) = if x <= y then (x::y::ys)
                               else y::insert(x, ys)
\end{lstlisting}

The function \lstinline{sort} would take a list of integers as input and return
it sorted in ascending order. One approach to implement \lstinline{sort} is to
resort to an auxiliary function, \lstinline{insert}, taking an element
\lstinline{x} and a list \lstinline{xs} as inputs, and returning a new list
\lstinline{xs'}. Assuming that \lstinline{xs} is sorted, \lstinline{insert}
guarantees that \lstinline{xs'} is also sorted by placing \lstinline{x} in the
``right place''. It is easy to see, by induction, that \lstinline{sort} is
correctly defined.

Nevertheless, it would be nicer if we could just give the synthesizer a
specification of what it means for a list to be sorted and let it figure out the
implementation. For example, the synthesizer could support using type signatures
and predicates as specifications. We could also hint the structure of the
implementation to the synthesizer by giving a specification for another
function, \lstinline{insert}, which we could believe to be useful to implement
\lstinline{sort}:

\begin{lstlisting}
  isSorted: List Int -> Bool
  isSorted([])       = True
  isSorted([x])      = True
  isSorted(x::y::ys) = x <= y and isSorted(y::ys)

  sort: (xs: List Int) -> (xs': List Int)
  sortSpec: isSorted(xs')

  insert: (x: Int, xs: List Int) -> (xs': List Int)
  insertSpec: isSorted(xs) ==> isSorted(xs')
\end{lstlisting}

There is a problem with our specification, however, as there are many unwanted
programs that satisfy it. The program that ignores its input and simply outputs
the empty list is one such example.

% \begin{lstlisting}
%   sort: (xs: List Int) -> (xs': List Int)
%   sort(xs) = []

%   insert: (x: Int, xs: List Int) -> (xs': List Int)
%   insert(x, xs) = []
% \end{lstlisting}

The problem is that the specification does not model our intent precisely. It
should be clear that the output \lstinline{xs'} should be in some way related to
the input \lstinline{xs}, but the current specification does not capture that
relation. We must require that \lstinline{xs'} has exactly the same contents as
those of \lstinline{xs}, meaning that every element of \lstinline{xs} should
appear in \lstinline{xs'} in the same amount. We could express that requirement
as a binary predicate \lstinline{sameContents} over lists (whose implementation
is omitted) and add it to specification:

\begin{lstlisting}
  sort: (xs: List Int) -> (xs': List Int)
  sortSpec: isSorted(xs') and sameContents(xs, xs')

  insert: (x: Int, xs: List Int) -> (xs': List Int)
  insertSpec: isSorted(xs) ==>
    isSorted(xs') and sameContents(xs', x::xs)
\end{lstlisting}

It should be clear by now why writing specifications can be tricky, even in
simple cases. What might be less clear is that the specification as it is might
still not be fully specified. There are countless other properties that we might
want our sorting procedure to satisfy, such as stability, complexity or
adaptability.