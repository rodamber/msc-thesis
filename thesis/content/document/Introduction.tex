\chapter{Introduction}
\label{chapter:introduction}

% $\exists P \ldotp \phi(P) $

\todo{Program synthesis is the problem of automatically generating
implementations from high-level specifications.}{add ref} \todo{Amir Pnuelli,
former computer scientist and Turing Award, described it as ``one of the most
central problems in the theory of programming''}{add ref}, and it has been
\todo{portrayed as one of the holy grails of computer science}{solar-lezama phd
thesis, gulwani et al overview}.

It is easy to understand why: if only we could tell the computer \textit{what to
do} and let it figure out \textit{how to do} it, the task of programming would
be so much easier!

Program synthesis is a hard problem, though. \todo{Programming is a task that is
hard for humans}{add ref 45 solar-lezama thesis} and, given its generality,
there is no reason to believe it should be any easier for computers. Computers
lack algorithmic insight and domain expertise. Moreover, the challenge is
actually twofold: we need to find out both how to tackle the intractibility of
the program space, and how to accurately capture user intent \cite{Gulwani2017}.

% 1. explicar porque que o espaco de procura e grande

It is important to have in mind that program synthesis is not a panacea for
solving problems in computer programming. For example, while we are interested
in properties besides functional correctness, such as efficiency or
succinctness, it is impossible for program synthesis to eliminate all sources of
bugs. Particularly, it cannot solve problems originating from bad specifications
that come from a bad understanding of the problem domain.

Writing specifications is a delicate process. This is better understood with an
\todo{example}{adicionar o exemplo}.
% 2. explicar porque que especificar a intencao do utilizador e uma tarefa delicada
%    (colocar o exemplo da apresentacao)

% ------------------------------------------------------------------------------
% Questions left to answer?
% - Extend the motivation.
% - In practice where can this be applied (present/future)?
% - How does prog. synth. compare to other kinds of automatic programming? In
% particular, how does it compare to ML?
% - Can you give an example?
% - What does this thesis introduce? Give a summary of the work done in the rest
% of the thesis.

% Other questions that will be answered in the following sections include:
% - How to provide user intent?
% - How to solve the problem efficiently?

% ------------------------------------------------------------------------------
% Compromise:
% - reducing the scope or domain of application
% - make the synthesizers domain-specific and less general (embed the insights
%   directly into the synthesizers)

% Solutions are:
% - solver-aided programming: human provides high-level insights while the
% synthesizer takes care of the low-level details.
% - human-computer working together (interaction through active learning)

% Automating this work is good because:
% - code is correct by construction

% - sintetizadores altamente especializados e calibrados para dominios particulares
% - bugs enviados para a camada de abstraccao acima

