\section{Introduction}
\label{sec:introduction}

Program synthesis is the problem of automatically generating program
\textit{implementations} from high-level \textit{specifications}.
Amir Pnuelli, former computer scientist and Turing Award, described it as ``one
of the most central problems in the theory of
programming''~\cite{Pnueli:1989:ARM}, and it has been portrayed as one of the
holy grails of computer science~\cite{Solar-Lezama:2008,Gulwani2017}.
It is easy to understand why: if only we could tell the computer \textit{what to
do} and let it figure out \textit{how to do} it, the task of programming would
be so much easier!
However, Program synthesis is a very hard problem.
Programming is a task that is hard for humans and, given its generality,
there is no reason to believe it should be any easier for computers.
Computers lack algorithmic insight and domain expertise.
Moreover, the challenge is twofold: we need to find out both how to tackle the
intractability of the program space, and how to accurately capture user
intent~\cite{Gulwani2017}.

It is important to have in mind that program synthesis is not a panacea for
solving problems in computer programming.
For example, we might be interested in other properties besides functional
correctness, such as efficiency or succinctness of the generated program.
Also, in the general case, it is impossible for program synthesis to eliminate
all sources of bugs.
Particularly, it cannot solve problems originating from bad specifications
that come from a poor understanding of the problem domain.

\subsection{OutSystems and \Glsfmtlong{pbe}}
\label{sec:outsystems-pbe}

This work was done during an internship at
OutSystems.~\footnote{\url{https://www.outsystems.com}} 
OutSystems is also the name of a low-code platform that features visual
application development, easy integration with existing systems, and the
possibility to add own code when needed.
With OutSystems one should be able to build enterprise-grade applications
swiftly and with a great degree of
automation.
To that effect, the kind of programs we are interested in are expressions in the
OutSystems language.
Thus, we were interested in building a synthesizer for OutSystems expressions
that would be \textit{performant} and \textit{easy to use}.
This would imply that the synthesis process should finish in a matter of seconds
(instead of hours or days), and the synthesizer should have a
``push-button''-style interface that does not force the user to acquire new
skills.
The paradigm in program synthesis that best fits this scenario is called
\glsfmtfull{pbe}.
In \gls{pbe}, the synthesizer should be able to synthesize ``correct'' programs
merely from a small set of input-output examples.
% For instance, one possible input-output example for the program from
% example~\ref{ex:intro:first-name} is, for example, (<``John Michael Doe'', ``Dr.
% ''>, ``Dr. John'').
By correct we mean that the program matches the user intent.

The most notable success in the area of program synthesis is incorporated in a
tool called FlashFill~\cite{Gulwani:2011}.
FlashFill employs \gls{pbe} technology and is currently integrated in Microsoft
Excel.
FlashFill is able to synthesize programs that manipulate strings very fast,
typically under 10 seconds.
Another notable \gls{pbe} synthesizer is SyPet~\cite{Feng:2017:CSC}, which has
been applied in the domain of Java programs.

\subsection{Contributions}

In this work, we surveyed the state of the art in program synthesis, and
implemented two component-based \gls{pbe} synthesizers based on an \gls{ogis}
architecture.
Both synthesizers employ a mixture of constraint solving with basic enumerative
search, differing on the amount of work they put on the constraint solving
phase, for which they make use of an \Glsfmtfull{smt} solver.
The components that we used correspond to a small subset of expressions of the
OutSystems API that manipulate Integers and Text.
However, the synthesizers are not specific to OutSystems expressions and could
be instantiated in other domains.
We benchmarked both synthesizers and compared them to SyPet~\cite{Feng:2017:CSC}.