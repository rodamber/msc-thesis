\section{Synthesis}
\label{sec:synthesis}

We are working in the context of the OutSystems platform. OutSystems is a
low-code platform that features visual application development, easy integration
with existing systems, and the possibility to add own code when needed. To that
effect, the kind of programs we are interested in are expressions in the
OutSystems language.
\footnote{\url{https://success.outsystems.com/Documentation/11/Reference/OutSystems_Language/Logic/Expressions}}

We can think of OutSystems expressions as a simple functional language of
operands and operators that compose themselves to create pure, stateless,
loopless programs. This means that OutSystems expressions do not have
side-effects, like printing to the screen, or writing to a database, and do not
permit any variables or (for/while) loops. They do, however, have conditional
expressions in the form of ``if'' statements. The library of builtin expressions
includes functions that manipulate builtin data types such a text strings,
numbers, or dates.

\begin{table*}[]
  \footnotesize
  \noindent\makebox[\textwidth]{
    \begin{tabular}{@{}ll@{}ll@{}}
      \toprule

      \multicolumn{1}{c}{Function Signatures}
      & \multicolumn{1}{c}{Description}
      & \multicolumn{1}{c}{Examples}
      \\ \midrule

      Concat(t: Text, s: Text): Text
      & \begin{tabular}{@{}l@{}}
          Concatenation of the texts \textit{t}\\
          and \textit{s}.
        \end{tabular}
      & \begin{tabular}[c]{@{}c@{}}
          Concat("", "") = ""\\
          Concat("x", "yz") = "xyz"
        \end{tabular}
      \\ \midrule

      Index(t: Text, s: Text, n: Integer): Text
      & \begin{tabular}{@{}l@{}}
          Retrieves first position of \textit{s} at\\
          or after \textit{n} characters in t.\\
          Returns -1 if there are no\\
          occurrences of \textit{s} in \textit{t}.
        \end{tabular}
      & \begin{tabular}[c]{@{}c@{}}
          Index("abcbc", "b", 0) = 1\\
          Index("abcbc", "b", 2) = 3\\
          Index("abcbc", "d", 0) = -1
        \end{tabular}
      \\ \midrule

      Length(t: Text): Integer
      & \begin{tabular}{@{}l@{}}
          Returns the number of\\
          characters in \textit{t}.
        \end{tabular}
      & \begin{tabular}[c]{@{}c@{}}
          Length("") = 0\\
          Length("abc") = 3
        \end{tabular}
      \\ \midrule

      Substr(t: Text, i: Integer, n: Integer): Text
      & \begin{tabular}{@{}l@{}}
          Returns the substring of \textit{t} with\\
          \textit{n} characters starting at index\\
          \textit{i}.
        \end{tabular}
      & \begin{tabular}[c]{@{}c@{}}
          Substr("abcdef", 2, 3) = "cde"\\
          Substr("abcdef", 2, 100) = "cdef"
        \end{tabular}
      \\ \midrule

      +(x: Integer, y: Integer): Integer
      & Integer addition.
      & \begin{tabular}[c]{@{}c@{}}
          1 + 2 = 3\\
          0 + 1 = 1
        \end{tabular}
      \\ \midrule

      -(x: Integer, y: Integer): Integer
      & Integer subtraction.
      & \begin{tabular}[c]{@{}c@{}}
          1 - 2 = -1\\
          2 - 1 = 1
        \end{tabular}
      \\ \bottomrule
    \end{tabular}}
  \caption{Description of the builtin functions used for synthesis.}
  \label{table:builtin-description}
\end{table*}

In this work we are mainly interested in synthesizing expressions that
manipulate text strings, like concatenation, substring slicing or whitespace
trimming. As some of these operations involve indexing, we are also interested
in synthesizing simple arithmetic expressions involving addition and
subtraction. Therefore, the data types we are working with are text strings and
integers. In particular, we are not dealing neither with boolean, nor conditional
expressions. Table~\ref{table:builtin-description} describes the builtin
expressions that our synthesized programs can be composed of, and shows
input-output examples for each of the functions.

\begin{example}\label{ex:first-name}
  Suppose that, given a text representing a person's name, we are interested in
extracting the first name and prepend it with a prefix. For example, given the
text ``John Michael Doe'' and the prefix ``Dr. '' we would like to obtain
``Dr. John''. The following expression satisfies the specification.
 
\begin{lstlisting}
  prog(name, prefix) = Concat(prefix,
      Substr(name, 0, Index(name, " ", 0)))
\end{lstlisting}
\end{example}

\subsection{Setwise Encoding}
\label{sec:setwise-encoding}

Because OutSystems expressions are composed of self-contained pure functions,
this synthesis problem fits nicely in the component-based synthesis
paradigm~(\ref{sec:components}). Therefore, assume we are given a \textit{library}
of base components $F$ that the synthesizer can use in order to compose the
programs. These components will be builtin functions drawn from the OutSystems
library, or combinations of them. Each component can take a finite number of
inputs and return exactly one output. More formally, a component $f \in F$ is
represented by an expression $\phi{}_f$ (a specification) that specifies how its
input parameters $P_f$ relate to its return value $r_f$.

We can see from the previous examples that OutSystems expressions can also
include constant literals, like \lstinline{" "}, or \lstinline{0}. These could
have be given as input, but we would like them to be figured out automatically
by the synthesizer.

\begin{figure}
  \begin{lstlisting}[frame=tlrb]
    prog(name, prefix):
      c0 = " "
      c1 = 0
      r1 = Index(name, c0, c1)
      r2 = Substr(name, c1, r1)
      r3 = Concat(prefix, r2)
  \end{lstlisting}
\caption{\gls{ssa} representation of the program from
    Example~\ref{ex:first-name}. The last variable, \lstinline{r3}, is assumed
    here to be the return value of the program.}
\label{fig:first-name-ssa}
\end{figure}

OutSystems expressions can be represented in \gls{ssa} form.
A program in \gls{ssa} form is a line program where every variable is assigned
exactly once and defined before it is used.
For example, the program from Example~\ref{ex:first-name} could be written
as shown in Listing~\ref{fig:first-name-ssa}.
The body of a program in this format can be described succinctly with the
following \gls{cfg}:
%
\[S ::= ID = c \OR ID = f(x_1, ..., x_n) \OR S;S\]

\noindent
$ID$ stands for an identifier in the OutSystems language. The non-terminal $S$
represents a line in the program. A line is an assignment of a variable to a
constant literal $c$ or to the return value of a component $f$ on inputs $x_1,
..., x_n$. As long as the program is well-typed, an input to a component can
be one of the inputs of the program, or variable defined in a preceding line.
Thus, the general structure of a program in \gls{ssa} form is a sequence of
assignments.

The synthesizer follows the \gls{ogis} model, described in
Section~\ref{sec:ogis}.
It has a learner part and an oracle part, which will call here the
\textit{enumerator} and the \textit{solver}, respectively.

The enumerator receives the set of input-output examples as input, and is
parameterized by the library of components. The enumerator is responsible for
drawing a subset of components from the library. The components are drawn by
trying all sets of combinations (with replacement) in order of increasing size.
It then passes these components to the solver, along with the input-output
examples, and queries whether there is any program made only of those components
that satisfies the examples. There is the additional restriction that the
program must use each of the components in the query exactly once.

The solver works by encoding the query into an \gls{smt} formula, and uses an
automated \gls{smt} solver to check for satisfiability. The \gls{smt} solver
might or might not be able to solve the formula. If the formula is satisfiable,
the solver responds to the enumerator with SAT and a solution (called a
\textit{model}) to that formula. If not, it responds UNSAT or UNKNOWN, depending
on whether the formula is unsatisfiable or the \gls{smt} solver could not verify
its satisfiability, respectively.

The procedure keeps going in a loop until the enumerator receives SAT from the
solver. The enumerator then decodes the model into an actual program, which is
then returned.

\paragraph{Program Formula}
\label{sec:program-formula}

Let us take the program from Figure~\ref{fig:first-name-ssa} as a running
example in order to understand how, given the input-output examples and a set of
components, we can construct a formula whose model can be decoded into a program
that satisfies the examples. This program is good because it is a small,
non-trivial program, which uses non-input constant variables (\lstinline{" "}
and \lstinline{0}), and not every component has the same return type.

In order to encode the space of valid programs, the solver has to decide
\begin{enumerate*}[(1)]
\item how many constant variables to create and which values to assign them,
\item in which order the components appear in the program, and
\item which \textit{actual} values to pass to the \textit{formal} parameters of
  each component.
\end{enumerate*}

In order to encode the program we need variables in the formula to model several
entities:
\begin{enumerate*}[(1)]
\item the input variables to the program;
\item the constant variables;
\item the formal parameters of each component;
\item the return variables of all components;
\item the output variable of the program; and
\item the \textit{connections} between the variables, that specify which
  actual parameters are passed to the formal parameters of each component.
\end{enumerate*}
Thus, we have a set $I$ of input variables, a set $C$ of constant variables, a
set $P$ of the formal parameters of all components, a set $R$ of the return
variables of all components, and a variable $o$, the output variable of the
program. We will denote the formal parameter variables and return variable of
component $f$ by $P_f$ and $r_f$ , respectively. Also, we will use $F'$ to refer
to the components available to the solver (in this case, \lstinline{Index},
\lstinline{Substr}, and \lstinline{Concat} --- recall that $F$ is used to denote
the library of all components).

\paragraph{Well-Formedness Constraint}
\label{sec:wfp-constraint}

To encode the connections, we require a set $L$ of integer-valued
\textit{location} variables $l_x$ for each variable $x \in I \cup C \cup P \cup
R \cup \{o\}$. Intuitively, if $x$ is the return variable of component $f$, then
$l_x$ is the line number where $f$ appears in the program. If $x$ is a formal
parameter of some component, then $l_x$ is the line number where the actual
parameter is defined. In practice, each variable in $I$ is assigned a line
number from $1$ to $|I|$ (in the obvious way), variables in $C$ are assigned
a number from $|I| + 1$ to $C$, and the output variable $\{o\}$ is assigned the
line number $|I| + |C| + |R|$ (the last line). The locations of variables in $R$
range from $|I| + |C| + 1$ to $|I| + |C| + |R|$. The location of each formal
parameter $x \in P$ ranges from $1$ up to the location of its corresponding
component.
In general, we can capture these constraints with the following formula:
%
\begin{align*}
  \psi{}_{\mathtt{range}}(I, C, P, R) &=
  %
  \bigwedge_{f \in F'} (M + 1 \leq l_{r_f} \leq M + |R|)\\
  %
  &\wedge
  \bigwedge_{f \in F'}
  \bigwedge_{p \in P_f} (1 \leq l_p < l_{r_f})
\end{align*}

\noindent
where $M = |I| + |C|$.
 
The locations of the variables $x \in I \cup C \cup \{o\}$ are known as soon as
we decide how many constant variables the program will have at its disposal. The
objective is then to find an assignment to the locations of the variables $x \in
P \cup R$. These give us all the information we need to decode back the program.
We need a few more constraints in order to encode the space of well-formed
programs. First, no two components should have the same location. Thus,
we have $l_{r_1} \neq l_{r_2} \neq l_{r_3}$. In the general case:
%
\[
  \psi{}_{\mathtt{rloc}}(R) =
  \bigwedge_{\substack{x, y \in R\\ x \not\equiv y}} (l_{x} \neq l_{y}) 
\]

\noindent
Second, the program must be well-typed, so the location of each formal parameter
$x \in P$ should differ from the location of any $y \in I \cup C \cup R$ whose
type does not match with $x$. In the same vein, only components whose return
value has the same type as the output may appear in the last line. These
constraints can be written in the following way:
%
\begin{align*}
  \psi{}_{\mathtt{tloc}}(I, o, C, P, R) &=
  %
  \bigwedge_{p \in P}
  \bigwedge_{\substack{x \in I \cup C \cup R \\ type(p) \neq type(x)}}
  (l_p \neq l_x)\\
  %
  &\wedge
  \bigwedge_{\substack{r \in R \\ type(r) \neq type(o)}}
  (l_r \neq l_o)
\end{align*}

Combining formulas $\psi{}_{\mathtt{range}}$, $\psi{}_{\mathtt{rloc}}$, and
$\psi{}_{\mathtt{tloc}}$ we get the full program well-formedness constraint:
%
\begin{align*}
  \psi{}_{\mathtt{wfp}}(I, o, C, P, R) &=
  \psi{}_{\mathtt{range}}(I, C, P, R)\\
  &\wedge \psi{}_{\mathtt{rloc}}(R) \wedge \psi{}_{\mathtt{tloc}}(I, o, C, P, R)
\end{align*}

\paragraph{Functional Constraint}
\label{sec:functional-constraint}

The formula we arrived to in the last section encodes the space of all
\textit{syntactically} well-formed programs. However, in no way does it
constrain the programs to have the correct \textit{semantics}.
In particular, it does not
\begin{enumerate*}[(1)]
\item \label{itm:fc-one} relate the return values to their corresponding components;
\item \label{itm:fc-two} make sure that variables share the same value if they share the same
location; nor
\item \label{itm:fc-three} make sure that the program satisfies the input-output example.
\end{enumerate*}

Constraint~\ref{itm:fc-one} can be guaranteed by the following formula:
%
\[
  \psi{}_{spec}(P, R) = \bigwedge_{f \in F'} \phi{}_f (P_f, r_f)
\]

\noindent
Recall that $\phi{}_f$ denotes the specification of component $f$, which relates
its formal parameters to its return value.
Constraint~\ref{itm:fc-two} refers to
the dataflow properties of the program.
In general, these properties can be encoded in the
following formula:
%
\begin{align*}
  \psi{}_{flow}&(I, C, P, R) =\\
  % 
    &\bigwedge_{p \in P}
    \bigwedge_{\substack{x \in I \cup C \cup R \\ type(p) = type(x)}} 
    (l_p = l_x \implies p = x)\\
    % 
    &\wedge \bigwedge_{\substack{r \in R \\ type(r) = type(o)}}
    (l_r = l_o \implies r = o)
\end{align*}

\noindent
We would like to make sure that every component given by the enumerator is
effectively used in the generated program, meaning that their correspondent
return value should be either the actual parameter of some other component, or
the final output of the program. This makes sense because of the way that the
enumerator draws components from the library (combinations with replacement in
order of increasing size), as every subset of $F'$ would have already been
passed to the solver and deemed insufficient in order to build a satisfying
program. In general, this can be encoded in the formula
%
\begin{align*}
  \psi{}_{out}(o, P, R) &=
  \bigwedge_{f \in F}
  \bigvee_{\substack{p \in P - P_f \\ type(p) = type(r_f)}}(l_{r_f} = l_p)\\
  %
  &\vee \bigwedge_{\substack{r \in R \\ type(r) = type(o)}} (l_{r} = l_o)
\end{align*}

\noindent
Formula $\psi{}_{flow}$ along with $\psi{}_{out}$ guarantee that the generated
program has the correct output, thus ensuring that constraint~\ref{itm:fc-three}
is satisfied. Moreover, we would also like to make sure that no program input
goes ignored, significantly cutting down the search space, which can be
guaranteed by a formula similar to $\psi{}_{out}$:
%
\[
  \psi{}_{in}(I, P) = \bigwedge_{i \in I}\bigvee_{p \in P}(l_i = l_p)
\]

The functional constraint is obtained by adding to $\psi{}_{\mathtt{wfp}}$ the
formulas from this section, wrapping the formal parameter and return value
variables $x \in P \cup R$ under an existential quantifier:
%
\begin{align*}
  \psi{}_{prog}&(I, o, C) = \exists P,R\ldotp
                 (\psi{}_{\mathtt{wfp}}(I, o, C, P, R)\\
               &\wedge \psi{}_{spec}(P, R) \wedge \psi{}_{flow}(I, C, P, R)\\
               &\wedge \psi{}_{out}(o, P, R) \wedge \psi{}_{in}(I, P))
\end{align*}

\paragraph{Full Constraint}
\label{sec:full-constraint}

% FIXME Forgot to pass `C' as a parameter to `\psi_prog'
% FIXME Forgot to pass `L' as a parameter almost everywhere
We are now in position to show the full formula. The formula $\psi{}_{prog}(I,
o)$ encodes a well-formed program that satisfies the input-output example $(I,
o)$. We can get a program that works over \textit{all} provided
input-output examples $(I, o) \in E$ with a simple conjuction over $E$ like so:
%
\[
  \Psi{} = \exists L,C\ldotp \bigwedge_{(I, o) \in E}\psi{}_{prog}(I, o, C)
\]

\noindent
In essence, this formula encodes the different runs of the program over all
the provided input-output examples. A model of this formula corresponds to a
program that uses only the components provided by the enumerator, and satisfies
all input-output examples. The variables $x \in L \cup C$ should retain their
values across all runs, and are the only information we need in order to decode
back the program.

\subsection{Whole Encoding}
\label{sec:whole-encoding}

In the previous approach the workload was split between the enumerator and the
solver: the enumerator would exhaustively search for all combinations (with
replacement) of the library of components in increasing order of size, while
passing them one by one to the solver. In turn, the solver would then verify if
there was any way to make a satisfying program using the given components each
exactly once. However, the number of combinations of a given size grows
exponentially. One wonders if it is possible to absolve the enumerator by
pushing as much workload as possible to the solver.

In this next approach we reduce the enumerator to query the solver with a single
number $n$.
Instead of drawing components from the library and passing them to the solver,
the enumerator now queries if there is any program of exactly $n$ components
that satisfies the examples. Thus, the solver is now parameterized by the
library of components $F$. This process is repeated for increasing values of $n$
until a program is found.

\paragraph{Program Formula}
\label{sec:program-formula-whole}

Here, we adapt the location-based encoding from the previous section in order to
have a solver that finds a program with the exact number of allowed components,
$n$. 
Just as before, we will have a set $I$ of input variables, an output variable
$o$, a set $C$ of constant variables, a set $L$ of location variables, a set $R$
of return variables, and a set $P$ of formal parameter variables, with some
changes, as we will see next.

This time we do not know how many times each component will be used in the
synthesized program.
This means we must have some way to model the choices of which components get
picked and which do not.
For this we introduce a new set of integer-valued variables $A = \{a_1, a_2,
\ldots, a_n\}$, which we will call the \textit{activation} variables.
The activation variables have values in the set $\{1, 2, \ldots, |F|\}$, with
the interpretation that if we have $a_i = k$, then component $f_k$ will be the
$i$-th component in the program (the one appearing in line $|I| + |C| + i$).
For the same reason this time we do not assign return variables to each
component.
Instead, as we know the number $n$ of components we are aiming for, we have one
return variable $r_i$, associated with each $a_i$ in the obvious way, for
$i = 1, \ldots, n$.
These will have their locations fixed beforehand: variable $r_i \in R$ will be
assigned to location $l_{r_i} = |I| + |C| + i$.

As the values of the activation variables $x \in A$ will be found by the solver
at constraint-solving time, we also do not know apriori the concrete types of
the return variables $x \in R$.
Lacking this information, we circumvent the problem by augmenting the type of
each return variable to be the union of all possible return types.
In our case, this would mean that all return variables may be either of type
\lstinline{Text}, or of type \lstinline{Integer}.

That is also the case for the types of the formal parameter variables $x \in P$.
Moreover, we do not know exactly how many formal parameter variables to create
because we do not know the arities of the components that are going to be picked.
We know, however, the maximum arity $m$ of all components, so we have an upper
bound on the number of formal parameter variables we might need, namely
$n * m$.
Thus, we introduce variables $p_{ij} \in P$, for $i = 1, 2, \ldots, n$, and
$j = 1, 2, \ldots, m$, with the interpretation that $p_{ij}$ is the $j$-th
parameter of $f_{a_i}$, the component activated by $a_i$.
We also augment their type to be the union of all possible formal parameter
types, plus a \lstinline{Null} type, inhabited by a single value,
\lstinline{null}, to indicate the absence of value.
A variable $p_{ij} \in P$ will be \lstinline{null} if and only if the arity of
component $f_{a_i}$ is less than $j$, and will be ignored by $f_{a_i}$ if that
is the case.
Finally, we perform the same \lstinline{Null} type augmentation to the location
variables $x \in L$, because there is no sense in having a location number if a
variable is null.
To encode these properties we need to enforce well-formedness and functional
constraints.

\begin{figure}
  \begin{lstlisting}[frame=tlrb,mathescape=true]
    prog($i_0$, $i_1$):
      $c_0$ = ?
      $c_1$ = ?
      $r_1$ = $f_{a_1}$($p_{11}$, $p_{12}$, $p_{13}$)
      $r_2$ = $f_{a_2}$($p_{21}$, $p_{22}$, $p_{23}$)
      $r_3$ = $f_{a_2}$($p_{31}$, $p_{32}$, $p_{33}$)
  \end{lstlisting}
\caption{Symbolic representation of a program with two inputs, $i_0, i_1 \in I$
  two constant variables $c_0, c_1 \in C$, three return value variables $r_i \in
  R$, for $i = 1, \ldots, n$, and nine formal parameter variables $p_{ij} \in
  P$, for $i = 1, \ldots, n$, $j = 1, \ldots, m$, with $n = m = 3$.
  The question marks '?' are values that will be found by the solver.}
\label{fig:whole-encoding-prog}
\end{figure}

\paragraph{Well-Formedness Constraint}
\label{sec:whole-well-formedness-constraint}

As with the last encoding, each input variable $x \in I$ is assigned a distinct
location $l_x$ from 1 to $|I|$, and each constant variable $x \in C$ is assigned
a distinct location from $|I| + 1$ to $|I| + |C|$.
As said in the previous section, this time we do the same for each return
variable $x \in R$, assigning to each a location from $|I| + |C| + 1$ to
$|I| + |C| + |R|$.
The upper bound on the location (if non-null) of each parameter variable $p_{ij}
\in P$ is now statically known to be $|I| + |C| + i$.

\paragraph{Functional Constraint}

Just as with the last encoding, we have the following concerns regarding the
semantics of the program:
\begin{enumerate*}[(1)]
\item \label{itm:whole-fc-one} relating return variables to the values of their
components and their parameters;
\item \label{itm:whole-fc-two} value sharing between variables with the same
location; and
\item \label{itm:whole-fc-three} effectively mapping the inputs $x \in I$ to the
output $o$.
\end{enumerate*}

For further details we refer the interested reader to Bernardo's Master's
thesis~\cite{Bernardo:2019:PSO}.