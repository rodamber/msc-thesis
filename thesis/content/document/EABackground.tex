\section{Background}
\label{sec:background}

The first part of solving a program synthesis problem is figuring out how the
user will communicate their intention to the synthesizer. An intention is
communicated by a \textit{specification}, and may be given in many different
ways, including:
logical specifications~\cite{Itzhaky:SIS:2010},
type signatures~\cite{Osera:2015:TPS,
  Frankle:2016:EST, Polikarpova:2016:PSP};
syntax-guided methods~\cite{Alur:sygus:2013} such as
sketches~\cite{Solar-Lezama:2008}, or components~\cite{Feng:2017:CST,
Feng:2017:CSC, Feng:2018:PSU}; and
inductive specifications such as input-output examples~\cite{Frankle:2016:EST,
Gulwani:2012:SDM, Leung:2015:IPS}, or demonstrations~\cite{Lau2003}.
The kind of specification should be chosen according to the particular use case
and to the background of the user, and it might dictate the type of techniques
used to solve the problem.

\paragraph{Component-Based Synthesis}
\label{sec:components}

In component-based
synthesis~\cite{Feng:2018:PSU,Feng:2017:CST,Feng:2017:CSC,Jha:oracle:2010}
we are interested in finding a loop-free program made out of a combination of
fundamental building blocks called \textit{components}. These components could
be, for example, methods in a library
\gls{api}~\cite{Feng:2017:CSC}, and the way in which they can be
combined forms the syntactic specification for the programs we want to find.
They may also be supplemented by additional constraints in the form of logical
formulas~\cite{Feng:2018:PSU}.

\paragraph{\gls{pbe}}

Inductive synthesis is an instance of the program synthesis problem where the
constraints are underspecified.
Sometimes the domain we want to model is complex enough that a complete
specification could be as hard to produce as the program itself, or might not
even exist.
In other cases, we might want the synthesizer to be as easy and intuitive to use
as possible for users coming from different backgrounds.
\gls{pbe} is an instance of inductive synthesis where the specification is given
by input-output examples that the desired program must satisfy.
Explicitly giving examples can be preferred due to their ease of use, especially
by non-programmers, when compared to more technical kinds of specification, such
as logical formulas.
The examples may be either \textit{positive}, i.e., an example that the desired
program must satisfy, or \textit{negative}, i.e., an example that the desired
should \textit{not} satisfy.
More generally, given some (implicit) input-output example, we may include asserting
properties of the output instead of specifying it completely
\cite{Polozov:2015:FFI}.
This can be helpful if it is impractical or impossible
to write the output concretely, e.g., if it is infinite.


The second part of solving a program synthesis problem is deciding which search
technique to apply in order to find the intended program.
First, we want to ensure that the program satisfies the semantic and syntactic
specifications.
Second, we want to leverage the specifications and the knowledge we have from
the problem domain in order to guide the search.
Common search techniques are
% deductive search (Section~\ref{sec:deductive-synthesis}),
enumerative search~\cite{Phothilimthana:2016:SUS,Alur:2017:SEP},
stochastic search~\cite{Schkufza:2013:SS,Singh:ranking:2015}, and
constraint solving~\cite{Feng:2018:PSU,Feng:2017:CST,Feng:2017:CSC}.
Modern synthesizers usually apply a combination of those, enabling us to
consider frameworks and techniques for structuring their construction, such as
\gls{cegis}~\cite{Solar-Lezama:2008},
\gls{cegis}($\mathcal{T}$)~\cite{Abate:2018:CMT}, and
\gls{ogis}~\cite{Jha:2017:TFS}.

\paragraph{Enumerative Search}
\label{sec:enumerative-search}

In the context of program synthesis, enumerative search consists of enumerating
programs by working the intrinsic structure of the program space to guide the
search.
The programs can be ordered using many different program metrics, the simplest
one being program size, and pruned by means of semantic equivalence checks with
respect to the specification.
Perhaps surprisingly, synthesizers based on enumerative search have been some of
the most effective to synthesize short programs in complex program spaces.
A reason why is that the search can be precisely tailored for the domain at
hand, encoding domain-specific heuristics and case-by-case scenarios that
result in highly effective pruning strategies.

\paragraph{Constraint Solving}
\label{sec:constraint-solving}

Another approach to program synthesis is to reduce the problem to 
constraint solving by the use of off-the-shelf automated constraint
solvers~\cite{Feng:2018:PSU,Feng:2017:CST,Feng:2017:CSC,Solar-Lezama:2008,Jha:oracle:2010}
(typically SAT or SMT solvers).
The idea is to encode the specification in a logical constraint whose solution
corresponds to the desired program.

\subsection{Oracle-Guided Inductive Synthesis}
\label{sec:ogis}

\textit{\Glsfmtlong{ogis}~(\Gls{ogis})} is an approach to program synthesis
where the synthesizer is split into two components -- the \textit{learner} and the
\textit{oracle} -- which communicate in an iterative \textit{query/response}
cycle.
The learner implements the search strategy to find the program and is
parameterized by some form of semantic and/or syntactic specifications.
The usefulness of the oracle is defined by the type of queries it can handle and
the properties of its responses. The characteristics of these components are
typically imposed by the application.
Typical queries and response types are some of the following~\cite{Jha:2017:TFS}:
%
\begin{compactitem}
\item \textit{Counterexample queries}, where given a candidate program $p$ the
  oracle responds with a positive input-output counterexample that $p$ does not
  satisfy, if it can find any, or $\bot$ otherwise.
\item \textit{Correctness queries}, where given a candidate program $p$ the
  oracle responds if $p$ is correct or not. If it is not, the oracle responds
  with a positive input-output counterexample.
\item \textit{Distinguishing input queries}, where given program $p$ and a set
  $X$ of input-output examples that $p$ satisfies, the oracle responds with a
  new counterexample $x$ to $p$ such that another program $p'$ exists that
  satisfies both $x$ and all the other examples in $X$.
\end{compactitem}

In general, the higher the capabilities of the oracles, the more expensive they
are to run. Distinguishing oracles are (typically) not as strong as
counterexample or correctness oracles as the returned counterexample is not
necessarily positive. To understand why they might be effectives tools we can
turn to the Bounded Observation Hypothesis \cite{Solar-Lezama:2008}, which
asserts that ``an implementation that works correctly for the common case and
for all the different corner cases is likely to work correctly for all inputs.''

The concept of \gls{ogis} was introduced by Jha~et~al.~\cite{Jha:oracle:2010}
as a generalization of \gls{cegis} when they applied this idea to a \gls{pbe}
synthesizer based on distinguishing inputs in order to deobfuscate malware and
to generate bit-manipulating programs.
