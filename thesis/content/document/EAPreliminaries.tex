\section{Terminology and Basic Definitions}
\label{sec:terminology}

This section covers some essential terminology that is used throughout the
document.

Program synthesis is fundamentally a search problem, hence, for it to be
well-defined there must be some \textit{search space} that contains the objects
we are searching for.
In our case, these objects are programs, so we have a \textit{program space}.

\begin{definition}[Program Space]
  Program space is the set of all programs in a given programming language.
\end{definition}

Programming languages are usually defined by means of a grammar.
For our purposes, we are interested in a particular kind of grammar, called a
\Glsfmtlong{cfg}~(\Glsfmtshort{cfg}).

\begin{definition}[\Glsfmtlong{cfg}]
  A \Glsfmtlong{cfg} is a tuple $(\mathcal{V}, \Sigma{}, R, S)$, where:
  \begin{enumerate*}[(1)]
  \item $\mathcal{V}$ is a finite set. It denotes the set of non-terminal symbols.
  \item $\Sigma{}$ is a finite set. It denotes the set of terminal symbols.
  \item $R$ is a finite relation from $\mathcal{V}$ to $(\mathcal{V} \cup
    \Sigma{})^*$, where the asterisk denotes the Kleene star operation.
  \item $S$ is a non-terminal symbol, corresponding to the start symbol of the
    grammar.
  \end{enumerate*}
\end{definition}

In our context, we are working in a \gls{pbe} setting, where the user explains
their intent by means of a set of input-output \textit{examples}.

\begin{definition}[Input-Output Example]
  An input-output example is a pair $(\mathbf{x}, y)$ where $\mathbf{x}$ is a
  list of inputs $<x_0, x_1, \ldots, x_n>$, and $y$ is the output.
\end{definition}

We say that a program $f$ \textit{satisfies} an input-output example
$(\mathbf{x}, y)$ if $f(\mathbf{x}) = y$.
We say that a program satisfies a \textit{set} of input-output examples if it
satisfies every example in that set.

% \section{Preliminaries on \Glsfmtfull{smt}}
% \label{sec:smt}

Many problems in the real world can be modeled in the form of logical formulas.
Thus, it is of great interest to have access to efficient off-the-shelf
logic engines, usually called \textit{solvers}, for these formulas.
The satisfiability problem is the problem of checking if a given logical formula
has a solution.
\gls{smt} solvers check the satisfiability of first-order logic formulas with
symbols and operations drawn from \textit{theories}, such as the theory of
uninterpreted functions, the theory of strings, or the theory of linear integer
arithmetic.
\gls{smt} solvers have seen a multitude of applications, particularly in
problems from artificial intelligence and formal methods, such as program
synthesis, or verification.

% \subsection{Syntax of \gls{smt} Formulas}
% \label{sec:smt-syntax}
%
% Here we define the language of well-formed \gls{smt} formulas. Formulas are
% composed of symbols and logical connectives over those symbols.

\begin{definition}[Signature]
  A signature $\Sigma{} = \Sigma{}^F \cup{} \Sigma{}^P$ is a set of
  ($\Sigma{}$-)symbols.
  $\Sigma{}^F$ is the set of \textit{function} symbols, and $\Sigma{}^P$ is the
  set of \textit{predicate} symbols.
  Each symbol has an associated arity. A zero-arity symbol $x$ is called a
  \textit{constant} symbol if $x \in \Sigma{}^F$, and is called a
  \textit{propositional} symbol if $x \in \Sigma{}^P$.
\end{definition}

\begin{definition}[Terms and Formulas]\label{def:terms-and-formulas}
  A ($\Sigma{}$-)term $t$ is an expression of the form:
  %
  \[t ::= c \OR f(t_1,\ldots,t_n) \OR ite(\phi{}, t_0, t_1)\]
  %
  where $c \in \Sigma{}^F$ with arity 0, $f \in \Sigma{}^F$ with arity
  $n > 0$, and $\phi{}$ is a formula.
  A ($\Sigma{}$-)formula $\phi{}$ is an expression of the form:
  %
  \begin{align*}
  \phi{} &::= A
    \OR p(t_1,\ldots,t_n)
    \OR t_0 = t_1 \\
    &\OR \bot{}
    \OR \top{}
    \OR \neg{}\phi{} \\
    &\OR \phi{}_0 \rightarrow{} \phi{}_1
    \OR \phi{}_0 \leftrightarrow{} \phi{}_1
    \OR \phi{}_0 \land{} \phi{}_1
    \OR \phi{}_0 \lor{} \phi{}_1 \\
    &\OR (\exists{}x\ldotp\phi{}_0)
    \OR (\forall{}x\ldotp\phi{}_0)
  \end{align*}
  %
  where $A \in \Sigma{}^P$ with arity 0, and $p \in \Sigma{}^P$ with arity
  $n > 0$.
\end{definition}

% \subsection{Semantics of \gls{smt} Formulas}
% \label{sec:smt-semantics}
%
% In this section we explore how \gls{smt} formulas are given meaning.

\begin{definition}[Model]
  Given a signature $\Sigma{}$, a \\ ($\Sigma{}$-)\textit{model} $\mathcal{A}$
  for $\Sigma{}$ is a tuple $(A, (\_)^{\mathcal{A}})$ where
  %
  $A$, called the \textit{universe} of the model, is a non-empty set, and
  $(\_)^{\mathcal{A}}$ is a function with domain $\Sigma{}$, mapping:
    each constant symbol $a \in \Sigma{}^F$ to an element
      $a^{\mathcal{A}} \in A$;
    each function symbol $f \in \Sigma{}^F$ with arity $n > 0$ to a total
      function $f^{\mathcal{A}}\colon A^n \to A$;
    each propositional symbol $B \in \Sigma{}^P$ to an element
      $B^{\mathcal{A}} \in \{\mathbf{true}, \mathbf{false}\}$;
    and each predicate symbol $p \in \Sigma{}^P$ with arity $n > 0$ to a
      total predicate $p^{\mathcal{A}}\colon A^n \to \{\mathbf{true},
      \mathbf{false}\}$.
\end{definition}

\begin{definition}[Interpretation]
  Given a model $\mathcal{A} = (A, (\_)^{\mathcal{A}})$ for a signature
  $\Sigma{}$, an \textit{interpretation} for $\mathcal{A}$ is a function, also
  called $(\_)^{\mathcal{A}}$, mapping each $\Sigma{}$-term $t$ to an element
  $t^{\mathcal{A}}\in A$ and each $\Sigma{}$-formula $\phi{}$ to an element
  $\phi{}^{\mathcal{A}}\in \{\mathbf{true}, \mathbf{false}\}$, in the following
  manner:
  %
  $f(t_1,\ldots,t_n)^{\mathcal{A}}$ is mapped to
    $f^{\mathcal{A}}(t_1^{\mathcal{A}},\ldots,t_n^{\mathcal{A}})$;
  $p(t_1,\ldots,t_n)^{\mathcal{A}}$ is mapped to
    $p^{\mathcal{A}}(t_1^{\mathcal{A}},\ldots,t_n^{\mathcal{A}})$;
  $ite(\phi{}, t_1, t_2)^{\mathcal{A}}$ is equal to $t_1^{\mathcal{A}}$ if
    $\phi{}^{\mathcal{A}}$ is $\mathbf{true}$, and equal to $t_2^{\mathcal{A}}$
    otherwise;
  $\bot{}^{\mathcal{A}}$ is mapped to $\mathbf{false}$;
  $\top{}^{\mathcal{A}}$ is mapped to $\mathbf{true}$;
  $(t_1 = t_2)$ is mapped to $\mathbf{true}$ if
    $t_1^{\mathcal{A}}$ is equal to $ t_2^{\mathcal{A}}$, and is mapped to
    $\mathbf{false}$ otherwise.
  $\Sigma$-symbols are mapped according to the mapping of the model just
    as before.
\end{definition}

\begin{definition}[Satisfiability]
  Given a model $\mathcal{A} = (A, (\_)^{\mathcal{A}})$ for a signature
  $\Sigma{}$, the model $\mathcal{A}$ is said to satisfy a $\Sigma{}$-formula
  $\phi{}$ if and only if $\phi{}^{\mathcal{A}}$ is $\mathbf{true}$.
  The formula $\phi{}$ is said to be \textit{satisfiable}.
\end{definition}

\begin{definition}[Theory]\label{def:theory}
  Given a signature $\Sigma{}$, a ($\Sigma{}$-)theory $\mathcal{T}$ for
  $\Sigma{}$ is a non-empty, and possibly infinite, set of models for $\Sigma{}$.
\end{definition}

\begin{definition}[$\mathcal{T}$-Satisfiability]
  Given a signature $\Sigma{}$ and a $\Sigma{}$-theory $\mathcal{T}$, a
  $\Sigma{}$-formula $\phi{}$ is said to be $\mathcal{T}$-\textit{satisfiable}
  if and only if (at least) one of the models of $\mathcal{T}$ satisfies
  $\phi{}$.
\end{definition}

\begin{definition}[SMT Problem]
  Given a signature $\Sigma{}$ and a $\Sigma{}$-theory $\mathcal{T}$, the
  \gls{smt} problem is the problem of determining the
  $\mathcal{T}$-satisfiability of $\Sigma{}$-formulas.
\end{definition}
