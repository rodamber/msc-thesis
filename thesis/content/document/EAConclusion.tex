\section{Conclusion and Future Work}
\label{sec:conclusion}

In this work, we tackle the problem of synthesizing OutSystems expressions from
examples, focusing on expressions that manipulate integers and text.
OutSystems sells solutions to help building enterprise-grade applications
swiftly and with a great degree of automation.
Thus, we were interested in an ``push-button''-style approach that would be
performant, and could generalize from a small number of examples.
We surveyed the state of the art in program synthesis, and implemented two
component-based \gls{pbe} synthesizers -- setwise and whole.
Both synthesizers employ a mixture of constraint solving with basic enumerative
search, differing on the amount of work they put on the constraint solving
phase.
We benchmarked both synthesizers and compared them to SyPet~\cite{Feng:2017:CSC}.

The setwise synthesizer can consistently synthesize programs that satisfy the
given input-output with up to four lines, but only manages to match the user
intent on programs with up to three lines.
Still, it manages to be competitive with SyPet, even when we configured the
latter for a scenario simulating user-provided constants.
On the other hand, the whole synthesizer fails to produce good results on both
fronts in all but the most trivial instances (programs with one line).
However, further benchmarking should be done, and refinements should be applied.
In particular, in the future we should benchmark configurations allowing for the
use of more constants to properly assess the impact on the number of instances
solved and on run time.
This should allow for more instances to be solved because it is very common for
programs to use more than one or two constants.
It would also be interesting to test configurations with user-provided inputs.
Another interesting scenario, probably involving extensions to the synthesizers,
includes trying to figure out from the input-output examples which constants the
program might use.
Moreover, it would be interesting to see how our synthesizers compare to a
synthesizer implemented using the PROSE
framework.\footnote{\url{https://microsoft.github.io/prose/}}

Both approaches use encodings which size scales linearly with the number of
input-output examples.
While not ideal, it is not unacceptable given that we are interested in
offloading as much work as possible from the user, and so the approach should
generalize with a small number of examples.
Given the experimental results, tending towards approaches more reliant on
general-purpose constraint solving might not be the best way to tackle this kind
of problems.
On the other hand, it would be interesting to explore how solvers specialized
for synthesis, or approaches with a tighter integration with the underlying
solver might fare.
In fact, some approaches in the literature apply this
idea~\cite{Feng:2018:PSU,Abate:2018:CMT}.
Moreover, it should be explored whether this approach scales when a larger
library of components is used, especially components that do not have a direct
encoding in \gls{smt} (and, in turn, could be more difficult, or even impossible
to synthesize).
It would be particularly interesting to support if-then-else expressions, along
with frequently used predicates over Integers and Texts.
