\section{Problem Description}
\label{sec:problem-description}

We are working in the context of the OutSystems platform. OutSystems is a
low-code platform that features visual application development, easy integration
with existing systems, and the possibility to add own code when needed. To that
effect, the kind of programs we are interested in are expressions in the
OutSystems language.
\footnote{\url{https://success.outsystems.com/Documentation/11/Reference/OutSystems_Language/Logic/Expressions}}

We can think of OutSystems expressions as a simple functional language of
operands and operators that one may combine in order to create pure, stateless,
and loopless programs. This means that OutSystems expressions do not have
side-effects, like printing to the screen, or writing to a database, and do not
allow variable declarations, or (for/while) loops.\footnote{Strictly speaking, it is possible to have expressions that produce side effects, but these are not the focus of this work.}
They do, however, have conditional expressions in the form of ``if'' statements.
The library of builtin expressions includes functions that manipulate builtin
data types such a text strings, numbers, or dates.
\footnote{\url{https://success.outsystems.com/Documentation/10/Reference/OutSystems_Language/Data/Data_Types/Available_Data_Types}}%
\footnote{\url{https://success.outsystems.com/Documentation/10/Reference/OutSystems_Language/Logic/Built-in_Functions}}

\begin{table}[]
  \noindent\makebox[\textwidth]{
    \begin{tabular}{@{}ll@{}ll@{}}
      \toprule

      \multicolumn{1}{c}{Function Signature}
      & \multicolumn{1}{c}{Description}
      & \multicolumn{1}{c}{Examples}
      \\ \midrule

      Concat(t: Text, s: Text): Text
      & \begin{tabular}{@{}l@{}}
          Concatenation of the texts \textit{t}\\
          and \textit{s}.
        \end{tabular}
      & \begin{tabular}[c]{@{}c@{}}
          Concat("", "") = ""\\
          Concat("x", "yz") = "xyz"
        \end{tabular}
      \\ \midrule

      Index(t: Text, s: Text, n: Integer): Text
      & \begin{tabular}{@{}l@{}}
          Retrieves first position of \textit{s} at\\
          or after \textit{n} characters in t.\\
          Returns -1 if there are no\\
          occurrences of \textit{s} in \textit{t}.
        \end{tabular}
      & \begin{tabular}[c]{@{}c@{}}
          Index("abcbc", "b", 0) = 1\\
          Index("abcbc", "b", 2) = 3\\
          Index("abcbc", "d", 0) = -1
        \end{tabular}
      \\ \midrule

      Length(t: Text): Integer
      & \begin{tabular}{@{}l@{}}
          Returns the number of\\
          characters in \textit{t}.
        \end{tabular}
      & \begin{tabular}[c]{@{}c@{}}
          Length("") = 0\\
          Length("abc") = 3
        \end{tabular}
      \\ \midrule

      Substr(t: Text, i: Integer, n: Integer): Text
      & \begin{tabular}{@{}l@{}}
          Returns the substring of \textit{t} with\\
          \textit{n} characters starting at index\\
          \textit{i}.
        \end{tabular}
      & \begin{tabular}[c]{@{}c@{}}
          Substr("abcdef", 2, 3) = "cde"\\
          Substr("abcdef", 2, 100) = "cdef"
        \end{tabular}
      \\ \midrule

      +(x: Integer, y: Integer): Integer
      & Integer addition.
      & \begin{tabular}[c]{@{}c@{}}
          1 + 2 = 3\\
          0 + 1 = 1
        \end{tabular}
      \\ \midrule

      -(x: Integer, y: Integer): Integer
      & Integer subtraction.
      & \begin{tabular}[c]{@{}c@{}}
          1 - 2 = -1\\
          2 - 1 = 1
        \end{tabular}
      \\ \bottomrule
    \end{tabular}}
  \caption{Description of the builtin functions used for synthesis.}
  \label{table:builtin-description}
\end{table}

In this work we are mainly interested in synthesizing expressions that
manipulate text strings, like concatenation, substring slicing or whitespace
trimming.
The reason for this focus is that text manipulation is a tedious and error prone
task, and it's often easier to provide examples than it is to describe them in
terms of programming language primitives.
As some of these operations involve indexing, we are also interested
in synthesizing simple arithmetic expressions involving addition and
subtraction.
Therefore, the data types we are working with are text strings and integers.
In particular, we are not dealing neither with booleans, nor conditional
expressions.
Table~\ref{table:builtin-description} describes the builtin expressions that our
synthesized programs can be composed of, i.e., the components of our \gls{dsl}.
These expressions were chosen because they are some of the most frequently found
in OutSystems applications, and because their semantics are easily translatable
to \gls{smt}.

\begin{example}
  Suppose that, given a text representing an email, we are interested in
  extracting its domain part.
  For example, given the text ``john.doe@outsystems.com'', we would like to
  obtain ``outsystems.com''.
  The following expression satisfies the specification.

\begin{lstlisting}
  prog(email) = Substr(email, Add(Index(email, "@", 0), 1), Length(email))
\end{lstlisting}
\end{example}

\begin{example}\label{ex:first-name}
  Suppose that, given a text representing a person's name, we are interested in
extracting the first name and prepend it with a prefix. For example, given the
text ``John Michael Doe'' and the prefix ``Dr. '' we would like to obtain
``Dr. John''. The following expression satisfies the specification.
 
\begin{lstlisting}
  prog(name, prefix) = Concat(prefix, Substr(name, 0, Index(name, " ", 0)))
\end{lstlisting}
\end{example}

\begin{example}\label{ex:second-name}
  Suppose now that, given a text representing a person's name, we are interested
in extracting not the first but the second name. For example, given the name
``John Michael Doe'' we would like to obtain ``Michael''. The following
expression satisfies the specification.
 
\begin{lstlisting}
  prog(name) = Substr(name,
                      Index(name, " ", 0),
                      Index(name, " ",
                            Index(name, " ", 0) + 1) - Index(name, " ", 0))
\end{lstlisting}
\end{example}

In our context, we are working in a \gls{pbe} setting, so we are interested in
synthesizing an OutSystems expression from a set of input-output examples.
For example, \lstinline|{(<"John Michael Doe">, "Michael")}|
is a set of input-output examples that are satisfied by the program in
Example~\ref{ex:second-name}.
However, the program \lstinline|prog(name) = Substr(name, 5, 7)|
also satisfies the examples, but does not capture the user's intent.
In order to rule out this program, we could extend the input-output examples
with a second example, for instance,
\lstinline|(<"Catherine Marie Joe">, "Marie")|.

