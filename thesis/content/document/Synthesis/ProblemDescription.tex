\section{Problem Description}
\label{sec:problem-description}

We are working in the context of the OutSystems platform. OutSystems is a
low-code platform that features visual application development, easy integration
with existing systems, and the possibility to add own code when needed. To that
effect, the kind of programs we are interested in are expressions in the
OutSystems language
\footnote{\url{https://success.outsystems.com/Documentation/11/Reference/OutSystems_Language/Logic/Expressions}}.

We can think of OutSystems expressions as a simple functional language of
operands and operators that compose themselves to create pure, stateless,
loopless programs. This means that OutSystems expressions do not have
side-effects, like printing to the screen, or writing to a database, and do not
permit any variables or (for/while) loops. They do, however, have conditional
expressions in the form of ``if'' statements. The library of builtin expressions
includes functions that manipulate builtin data types such a text strings,
numbers, or dates
\footnote{\url{https://success.outsystems.com/Documentation/10/Reference/OutSystems_Language/Data/Data_Types/Available_Data_Types}}
\footnote{\url{https://success.outsystems.com/Documentation/10/Reference/OutSystems_Language/Logic/Built-in_Functions}}.

\begin{table}[]
  \noindent\makebox[\textwidth]{
    \begin{tabular}{@{}ll@{}}
      \toprule

      \multicolumn{1}{c}{Function Signatures}
      & \multicolumn{1}{c}{Description}
      \\ \midrule

      Concat(t: Text, s: Text): Text
      & Concatenation of the texts \textit{t} and \textit{s}.
      \\ \midrule

      Index(t: Text, s: Text, n: Integer): Text
      & \begin{tabular}{@{}l@{}}
          Retrieves first position of \textit{s} at or after \textit{n}
          characters in t.\\
          Returns -1 if there are no occurrences of \textit{s} in \textit{t}.
        \end{tabular}
      \\ \midrule

      Length(t: Text): Integer
      & Returns the number of characters in \textit{t}.
      \\ \midrule

      Replace(t: Text, s: Text, s': Text): Text
      & Returns \textit{t} after replacing all occurrences of \textit{s} with
        \textit{s'}.
      \\ \midrule

      Substr(t: Text, i: Integer, n: Integer): Text
      & Returns the substring of \textit{t} with \textit{n} characters starting
        at index \textit{i}.
      \\ \midrule

      ToLower(t: Text): Text
      & Converts all characters in \textit{t} to lowercase.
      \\ \midrule

      ToUpper(t: Text): Text
      & Converts all characters in \textit{t} to uppercase.
      \\ \midrule

      Trim(t: Text): Text
      & Removes all leading and trailing space characters (' ') from \textit{t}.
      \\ \midrule

      TrimStart(t: Text): Text
      & Removes all leading space characters (' ') from \textit{t}.
      \\ \midrule

      TrimEnd(t: Text): Text
      & Removes all trailing space characters (' ') from \textit{t}.
      \\ \midrule

      +(x: Integer, y: Integer): Integer
      & Integer addition.
      \\ \midrule

      -(x: Integer, y: Integer): Integer
      & Integer subtraction.
      \\ \bottomrule
    \end{tabular}}
  \caption{Description of the builtin functions used for synthesis.}
  \label{table:builtin-description}
\end{table}

\begin{table}[]
  \noindent\makebox[\textwidth]{
    \begin{tabular}{@{}ll@{}}
      \toprule

      \multicolumn{1}{c}{Function Signatures}
      & \multicolumn{1}{c}{Examples}
      \\ \midrule

      Concat(t: Text, s: Text): Text
      & \begin{tabular}[c]{@{}l@{}}
          Concat("", "") = ""\\
          Concat("x", "yz") = "xyz"
        \end{tabular}
      \\ \midrule

      Index(t: Text, s: Text, n: Integer): Text
      & \begin{tabular}[c]{@{}l@{}}
          Index("abcbc", "b", 0) = 1\\
          Index("abcbc", "b", 2) = 3\\
          Index("abcbc", "d", 0) = -1
        \end{tabular}
      \\ \midrule

      Length(t: Text): Integer
      & \begin{tabular}[c]{@{}l@{}}
          Length("") = 0\\
          Length("abc") = 3
        \end{tabular}
      \\ \midrule

      Replace(t: Text, s: Text, s': Text): Text
      & Replace("ababa", "a", "xy") = "xybxybxy"
      \\ \midrule

      Substr(t: Text, i: Integer, n: Integer): Text
      & \begin{tabular}[c]{@{}l@{}}
          Substr("abcdef", 2, 3) = "cde"\\
          Substr("abcdef", 2, 100) = "cdef"
        \end{tabular}
      \\ \midrule

      ToLower(Text): Text
      & ToLower("AbC ") = "abc "
      \\ \midrule

      ToUpper(Text): Text
      & ToUpper("aBc ") = "ABC "
      \\ \midrule

      Trim(Text): Text
      & Trim("    xyz    ") = "xyz"
      \\ \midrule

      TrimStart(Text): Text
      & TrimStart("    xyz    ") = "xyz    "
      \\ \midrule

      TrimEnd(Text): Text
      & TrimEnd("    xyz    ") = "    xyz"
      \\ \bottomrule
    \end{tabular}}
  \caption{Examples for the builtin functions used for synthesis.}
  \label{table:builtin-examples}
\end{table}

In this work we are mainly interested in synthesizing expressions that
manipulate text strings, like concatenation, substring slicing or whitespace
trimming. As some of these operations involve indexing, we are also interested
in synthesizing simple arithmetic expressions involving addition and
subtraction. Therefore, the data types we are working with are text strings and
integers. In particular, we are not dealing neither with boolean, nor conditional
expressions. Table~\ref{table:builtin-description} describes the builtin
expressions that our synthesized programs can be composed of.
Table~\ref{table:builtin-examples} shows input-output examples for each of the
functions in Table~\ref{table:builtin-description}.

\begin{example}
  Suppose that we would like to capitalize a given text string. For example,
given the name ``joseph'' we would like to obtain ``Joseph''. The following
expression satisfies the specification.

\begin{lstlisting}
  prog(name) = ToUpper(Concat(Substr(name, 0, 1)),
                              Substr(name, 1, Length(name)))
\end{lstlisting}
\end{example}

\begin{example}\label{ex:first-name}
  Suppose that, given a text representing a person's name, we are interested in
extracting the first name and prepend it with a prefix. For example, given the
text ``John Michael Doe'' and the prefix ``Dr. '' we would like to obtain
``Dr. John''. The following expression satisfies the specification.
 
\begin{lstlisting}
  prog(name, prefix) = Concat(prefix,
                              Substr(name, 0,
                                     Index(name, " ", 0)))
\end{lstlisting}
\end{example}

\begin{example}\label{ex:second-name}
  Suppose now that, given a text representing a person's name, we are interested
in extracting not the first but the second name. For example, given the name
``John Michael Doe'' we would like to obtain ``Michael''. The following
expression satisfies the specification.
 
\begin{lstlisting}
  prog(name) = Substr(name, Index(name, " ", 0),
                      Index(name, " ",
                            Index(name, " ", 0) + 1) - Index(name, " ", 0))
\end{lstlisting}
\end{example}

In our context, we are working in a \gls{pbe} setting, so we are interested in
synthesizing an OutSystems expression from a set of input-output examples
$E = \{(I_i, o_i)\}_i$. For example,
\lstinline|{(<"John Michael Doe">, "Michael")}|
is a set of input-output examples that are satisfied by the previous program.
However, in this case, the program
\lstinline|prog(name) = Substr(0, 5, 7)|
would have done just fine. In order to rule out this program, we could extend
the input-output examples with a second example, for instance,
\lstinline|(<"Anne Marie Joe">, "Marie")|.

