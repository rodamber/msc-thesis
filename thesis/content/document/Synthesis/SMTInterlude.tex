\section{\gls{smt} Interlude}
\label{sec:smt-interlude}

At this point it might be nice to sit back and understand how the components of
our language (see Table~\ref{table:builtin-description}) are encoded in
\gls{smt}.
However, it should be noted that the encoding is independent of the particular
components used, and works as long as their semantics can be fully encoded in
\gls{smt}.

In section~\ref{sec:smt} we introduced the concepts of \gls{smt} necessary for
the purpose of this thesis.
One in particular, was the concept of theory (Definition~\ref{def:theory}).
Theories constrain the interpretation given to certain symbols.
Because we are dealing with text and integers, we make use of the \gls{smt}
theory of strings with linear integer arithmetic.
This theory allows one to reason about operations such as integer addition,
string length, or substring extraction.
The symbols we are interested in are represented in
Table~\ref{table:slia-symbols}.

\begin{table}[]
  \noindent\makebox[\textwidth]{
    \begin{tabular}{@{}lp{13cm}}
      \toprule

      \multicolumn{1}{c}{Symbol}
      & \multicolumn{1}{c}{Description}
      \\ \midrule

      $x \strconcat y$
      & Function symbol.
        The terms $x$ and $y$ are strings.
        Its value, a string, is the string concatenation of $x$ and $y$.
      \\ \midrule

      $IndexOf(s, t, i)$
      & Function symbol.
        The terms $s$ and $t$ are strings, and $i$ is an integer.
        Its value, an integer, is the first occurrence of $t$ in $s$ after the
        index $i$, or $-1$ if $t$ is not in $s$.
      \\ \midrule

      $Length(s)$
      & Function symbol.
        The term $s$ is a string.
        Its value, an integer, is the length (number of characters) of $s$.
      \\ \midrule

      $Substr(s,i,j)$
      & The term $s$ is a string, and the terms $i$ and $j$ are integers.
        Its value, a string, is the substring of $s$ starting at index $i$ with
        length $j$ (or from $i$ until the end of $s$ if the length of $s$ is
        less than $i + j$).
      \\ \midrule

      $x + y$
      & Function symbol.
        The terms $x$ and $y$ are integers.
        Its value, an integer, is the addition of $x$ and $y$.
      \\ \midrule
      
      $x - y$
      & Function symbol.
        The terms $x$ and $y$ are integers.
        Its value, an integer, is the result of subtracting $y$ from $x$.
      \\ \bottomrule
    \end{tabular}}
  \caption{Some symbols constrained by the theory of strings with linear integer
    arithmetic.}
  \label{table:slia-symbols}
\end{table}
\raggedbottom
%
The semantics of the components \lstinline{Concat}, \lstinline{IndexOf},
\lstinline{Length}, \lstinline{Substr}, \lstinline{Add}, and \lstinline{Sub} are
directly captured by their \gls{smt} counterparts $\strconcat$, $IndexOf$,
$Length$, $Substr$, $+$, and $-$, respectively. Concretely, their specifications
are as follows:
%
\begin{align*}
  \phi{}_{\text{\lstinline{Concat}}}(x, y, r)     &= (x \strconcat y   = r) \\
  \phi{}_{\text{\lstinline{IndexOf}}}(s, t, i, r) &= (IndexOf(s, t, i) = r) \\
  \phi{}_{\text{\lstinline{Length}}}(s, r)        &= (Length(s)        = r) \\
  \phi{}_{\text{\lstinline{Substr}}}(s, i, j, r)  &= (Substr(s, i, j)  = r) \\
  \phi{}_{\text{\lstinline{Add}}}(x, y, r)        &= (x + y            = r) \\
  \phi{}_{\text{\lstinline{Sub}}}(x, y, r)        &= (x - y            = r)
\end{align*}
