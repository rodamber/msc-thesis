\section{Whole Encoding}
\label{sec:whole-encoding}

In the previous approach the workload was split between the enumerator and the
solver: the enumerator would exhaustively search for all combinations (with
replacement) of the library of components in increasing order of size, while
passing them one by one to the solver. In turn, the solver would then verify if
there was any way to make a satisfying program using the given components each
exactly once. However, the number of combinations of a given size grows
exponentially. One wonders if it is possible to absolve the enumerator by
pushing as much workload as possible to the solver.

In this next approach we reduce the enumerator to query the solver with a single
number $n$ (plus the input-output examples). Instead of drawing components from the
library and passing them to the solver, the enumerator now queries if there is 
any program of exactly $n$ components that satisfies the examples. Thus, the
solver is now parameterized by the library of components $F$. This process is
repeated for increasing values of $n$ until a program is found.

\subsection{Program Formula}
\label{sec:program-formula-whole}

Here, we adapt the location-based encoding from the previous section in order to
have a solver that finds a program with the exact number of allowed components,
$n$. 
Just as before, we will have a set $I$ of input variables, an output variable
$o$, a set $C$ of constant variables, a set $L$ of location variables, a set $R$
of return variables, and a set $P$ of formal parameter variables, with some
changes, as we will see next.

This time we do not know how many times each component will be used in the
synthesized program.
This means we must have some way to model the choices of which components get
picked and which do not.
For this we introduce a new set of integer-valued variables $A = \{a_1, a_2,
\ldots, a_n\}$, which we will call the \textit{activation} variables.
The activation variables have values in the set $\{1, 2, \ldots, |F|\}$, with
the interpretation that if we have $a_i = k$, then component $f_k$ will be the
$i$-th component in the program (the one appearing in line $|I| + |C| + i$).
For the same reason this time we do not assign return variables to each
component.
Instead, as we know the number $n$ of components we are aiming for, we have one
return variable $r_i$, associated with each $a_i$ in the obvious way, for
$i = 1, \ldots, n$.
These will have their locations fixed beforehand: variable $r_i \in R$ will be
assigned to location $l_{r_i} = |I| + |C| + i$.

\begin{example}
  Consider program~\ref{fig:first-name-ssa}.
  Assuming that component \lstinline{Concat} corresponds to $k = 1$,
  \lstinline{Index} to $k = 2$, and \lstinline{Substr} to $k = 3$, then we would
  have $a_1 = 2 $, $a_2 = 3$, and $a_3 = 1$.
  We would also have $l_{r_1} = 5$, $l_{r_2} = 6$, and $l_{r_3} = 7$.
\end{example}

As the values of the activation variables $x \in A$ will be found by the solver
at constraint-solving time, we also do not know apriori the concrete types of
the return variables $x \in R$.
Lacking this information, we circumvent the problem by augmenting the type of
each return variable to be the union of all possible return types.
In our case, this would mean that all return variables may be either of type
\lstinline{Text}, or of type \lstinline{Integer}.

That is also the case for the types of the formal parameter variables $x \in P$.
Moreover, we do not know exactly how many formal parameter variables to create
because we do not know the arities of the components that are going to be picked.
We know, however, the maximum arity $m$ of all components, so we have an upper
bound on the number of formal parameter variables we might need, namely
$n * m$.
Thus, we introduce variables $p_{ij} \in P$, for $i = 1, 2, \ldots, n$, and
$j = 1, 2, \ldots, m$, with the interpretation that $p_{ij}$ is the $j$-th
parameter of $f_{a_i}$, the component activated by $a_i$.
We also augment their type to be the union of all possible formal parameter
types, plus a \lstinline{Null} type, inhabited by a single value,
\lstinline{null}, to indicate the absence of value.
A variable $p_{ij} \in P$ will be \lstinline{null} if and only if the arity of
component $f_{a_i}$ is less than $j$, and will be ignored by $f_{a_i}$ if that
is the case.
Finally, we perform the same \lstinline{Null} type augmentation to the location
variables $x \in L$, because there is no sense in having a location number if a
variable is null.

\begin{figure}
  \begin{lstlisting}[frame=tlrb,mathescape=true]
    prog($i_0$, $i_1$):
      $c_0$ = ?
      $c_1$ = ?
      $r_1$ = $f_{a_1}$($p_{11}$, $p_{12}$, $p_{13}$)
      $r_2$ = $f_{a_2}$($p_{21}$, $p_{22}$, $p_{23}$)
      $r_3$ = $f_{a_2}$($p_{31}$, $p_{32}$, $p_{33}$)
  \end{lstlisting}
\caption{Symbolic representation of a program with two inputs, $i_0, i_1 \in I$
  two constant variables $c_0, c_1 \in C$, three return value variables $r_i \in
  R$, for $i = 1, \ldots, n$, and nine formal parameter variables $p_{ij} \in
  P$, for $i = 1, \ldots, n$, $j = 1, \ldots, m$, with $n = m = 3$.
  The question marks '?' are values that will be found by the solver.}
\label{fig:whole-encoding-prog}
\end{figure}

\begin{example}
  Consider program~\ref{fig:first-name-ssa} and its corresponding symbolic
  skeleton from Figure~\ref{fig:whole-encoding-prog}.
  We have three components: \lstinline{Concat}, \lstinline{Index}, and
  \lstinline{Substr}.
  Components \lstinline{Index} and \lstinline{Substr} have the arity equal to 3,
  which are the largest among the three components.
  Thus, we have $n = m = 3$. This means we have nine variables $p_{ij} \in P$.
  Variables $p_{1j}$, with $j = 1, \ldots, 3$ are the formal parameters of
  $f_{a_1}$, which is component \lstinline{Concat}.
  However, \lstinline{Concat} only takes two parameters.
  These will be $p_{11}$ and $p_{12}$, meaning that $p_{13}$ will take value
  \lstinline{null}.
\end{example}

\subsubsection{Well-Formedness Constraint}
\label{sec:whole-well-formedness-constraint}

As with the last encoding, each input variable $x \in I$ is assigned a distinct
location $l_x$ from 1 to $|I|$, and each constant variable $x \in C$ is assigned
a distinct location from $|I| + 1$ to $|I| + |C|$.
As said in the previous section, this time we do the same for each return
variable $x \in R$, assigning to each a location from $|I| + |C| + 1$ to
$|I| + |C| + |R|$.
The upper bound on the location (if non-null) of each parameter variable $p_{ij}
\in P$ is now statically known to be $|I| + |C| + i$, so now our well-formedness
constraint is just:
%
\[
  \psi{}_{\mathtt{wfp}}(L,C,I,P) =
  \bigwedge_{i = 1, \ldots, n}\bigwedge_{j = 1, \ldots, m}
  (l_{p_{ij}} \neq \mathtt{null} \implies 1 \leq l_{p_{ij}} < |I| + |C| + i)
\]

\noindent
Compared to the well-formedness constraint of the setwise encoding, this one is
arguably made simpler by the fact that the lines of the return variables are
fixed apriori, and because we are not doing any type checking.

\begin{example}
  Consider the symbolic skeleton from Figure~\ref{fig:whole-encoding-prog}.
  We have:
  % 
  \begin{align*}
    &l_{i_0} = 1 \quad l_{i_1} = 2 \\
    &l_{c_0} = 3 \quad l_{c_1} = 4 \\
    &l_{r_1} = 5 \quad l_{r_2} = 6 \quad l_{r_3} = 7
  \end{align*}
  % 
  Also, the well-formedness constraint is the conjunction of the following:
  % 
  \begin{align*}
           l_{p_{11}} \neq \mathtt{null} \implies 1 \leq l_{p_{11}} < 5
    \qquad l_{p_{12}} \neq \mathtt{null} \implies 1 \leq l_{p_{12}} < 5
    \qquad l_{p_{13}} \neq \mathtt{null} \implies 1 \leq l_{p_{13}} < 5 \\
    %
           l_{p_{21}} \neq \mathtt{null} \implies 1 \leq l_{p_{21}} < 6
    \qquad l_{p_{23}} \neq \mathtt{null} \implies 1 \leq l_{p_{23}} < 6
    \qquad l_{p_{22}} \neq \mathtt{null} \implies 1 \leq l_{p_{22}} < 6 \\
    %
           l_{p_{31}} \neq \mathtt{null} \implies 1 \leq l_{p_{31}} < 7
    \qquad l_{p_{32}} \neq \mathtt{null} \implies 1 \leq l_{p_{32}} < 7
    \qquad l_{p_{33}} \neq \mathtt{null} \implies 1 \leq l_{p_{33}} < 7
  \end{align*}
\end{example}
%
\subsubsection{Functional Constraint}

Just as with the last encoding, we have the following concerns regarding the
semantics of the program:
\begin{enumerate*}[(1)]
\item \label{itm:whole-fc-one} relating return variables to the values of their
components and their parameters;
\item \label{itm:whole-fc-two} value sharing between variables with the same
location; and
\item \label{itm:whole-fc-three} effectively mapping the inputs $x \in I$ to the
output $o$.
\end{enumerate*}

Constraint $\psi{}_{\mathtt{spec}}$, which guarantees
constraint~\ref{itm:whole-fc-one}, should be similar to the one from
section~\ref{sec:functional-constraint}, but we will have to pay attention to
some special points.
First, we have to make sure that, for every component, we relate the correct
number of formal parameters to the return value (recall that we may have more
formal parameter variables than we may need because we do not know apriori which
component goes on which line):
%
\[
  \psi{}_{pnum}(A, P, R) =
  \bigwedge_{i=1,\ldots,n}\bigwedge_{k=1,\ldots,|F|}
  (a_i = k \implies \phi{}_{f_{k}}(p_{i1}, p_{i2}, \ldots, p_{ij_k}, r_i))
\]

\noindent
In the constraint shown above, $j_k$ is used as shorthand for the arity of
component $f_k$.
Second, we should ensure that formal parameter variables that are in excess
have value \lstinline{null}:
%
\[
  \psi{}_{null}(A, P) =
  \bigwedge_{i=1,\ldots,n} \bigwedge_{k=1,\ldots,|F|}
  (a_i = k \implies \bigwedge_{j=j_k+1,\ldots,m} p_{ij} = \mathtt{null})
\]

\noindent
Finally, we need to constrain the formal parameters to have the correct type,
because the specifications $\phi{}$ of the components are typed.
An example might, perhaps, be the best way to show how this constraint, which we
will call $\psi{}_{type}$, can be materialized.

\begin{example}
  Consider the symbolic skeleton from Figure~\ref{fig:whole-encoding-prog}, and
  suppose that we are working with components
  $f_1 = \mathtt{Concat}$, $f_2 = \mathtt{Index}$, and $f_3 = \mathtt{Substr}$.
  We will use the predicates $isstring$ and $isint$ to mean that a variable must
  be of type \lstinline{Text}, or \lstinline{Integer}, respectively.
  For this concrete case, we would the constraint to ensure that the type
  signatures of these components are satisfied:
  % 
  \begin{align*}
    \psi{}_{type}(A, P)
    &= \bigwedge_{i=1,\ldots,n}
      (a_i = 1 \implies (isstring(p_{i1}) \wedge isstring(p_{i2}))) \\
    &\wedge \bigwedge_{i=1,\ldots,n}
      (a_i = 2 \implies (isstring(p_{i1}) \wedge isstring(p_{i2}) \wedge isint(p_{i3}))) \\
    &\wedge \bigwedge_{i=1,\ldots,n}
      (a_i = 3 \implies (isstring(p_{i1}) \wedge isint(p_{i2}) \wedge isint(p_{i3}))) \\
  \end{align*}
\end{example}

\noindent
Constraint $\psi{}_{\mathtt{spec}}$ is then just the conjunction of the three
previous constraints:
%
\[
  \psi{}_{\mathtt{spec}}(A, P, R) =
  \psi{}_{pnum}(A, P, R) \wedge \psi{}_{null}(A, P) \wedge \psi{}_{type}(A, P)
\]

The dataflow properties of constraint~\ref{itm:whole-fc-two} are encoded in a
similar way to what we did in section~\ref{sec:functional-constraint}, with the
added constraint that every formal parameter variable whose value is
\lstinline{null} must also have a \lstinline{null} location:
%
\begin{align*}
  \psi{}_{flow}(C,I,P,R)
  &=\bigwedge_{i=1,\ldots,n}
    \bigwedge_{j=1,\ldots,m}
    \bigwedge_{\substack{x \in I \cup C \cup R}}
    (l_{p_{ij}} = l_x \implies p_{ij} = x) \\
  &\wedge\bigwedge_{i=1,\ldots,n}
    \bigwedge_{j=1,\ldots,m}
    (l_p = \mathtt{null} \implies p = \mathtt{null}) \\
\end{align*}
%
\noindent
Just as last time, we would like to ensure that every return variable is used
(as an actual parameter or as the output of the program), a property that be
encoded in the following way:
%
\begin{align*}
  \psi{}_{out}(o, P, R)
  =\bigwedge_{k=1,\ldots,n-1}
    \bigvee_{i=k+1,\ldots,n}
    \bigvee_{j=1,\ldots,m}
    (l_{r_k} = l_{p_{ij}})
  \wedge (r_n = o)
\end{align*}
%
\noindent
Formula $\psi{}_{out}$, along with $\psi{}_{flow}$, ensures that
constraint~\ref{itm:whole-fc-three} is satisfied.
We would also like to ensure that every input variable appears as an actual
parameter of some component, a property encoded by the following formula:
%
\begin{align*}
  \psi{}_{in}(I, P)
  =\bigwedge_{x \in I}
    \bigvee_{i=1,\ldots,n}
    \bigvee_{j=1,\ldots,m}
    (l_{x} = l_{p_{ij}})
\end{align*}
%
The functional constraint is obtained by conjoining the previous constraints
with the well-formedness constraint $\psi{}_{\mathtt{wfp}}$,
again wrapping the formal parameter and return value variables $x \in P \cup R$
under an existential quantifier:
%
\begin{align*}
  \psi{}_{prog}(A,L,C,I,o) &= \exists P,R\ldotp
  (\psi{}_{\mathtt{wfp}}(L, P) \wedge
  \psi{}_{spec}(A, P, R) \wedge
  \psi{}_{flow}(I, C, P, R)\\&\wedge 
  \psi{}_{out}(o, P, R) \wedge
  \psi{}_{in}(I, P))
\end{align*}

\subsubsection{Full Constraint}
\label{sec:whole-full-constraint}

Just as last time, the functional constraint $\psi{}_{prog}$ encodes the space
of well-formed programs that satisfy a specific input-output example.
To obtain a formula that ranges over all input-output examples, we just make a
conjuction over the set $E$:
%
\[
  \Psi{}=\exists{A,L,C}\ldotp\bigwedge_{(I, o) \in E}\psi{}_{prog}(A,L,C,I,o)
\]

A model of this formula is a valuation of the variables $x \in A \cup L \cup C$,
which can then be used to reconstruct a program that satisfies the input-output
examples $E$ using only $n$ components.