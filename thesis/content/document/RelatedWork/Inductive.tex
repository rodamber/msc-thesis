\subsection{Inductive Synthesis}
\label{sec:inductive}

Inductive synthesis is an instance of the program synthesis problem where the
constraints are under-specified.

\subsubsection{Programming By Demonstration}

\pdfmarkupcomment[color=red]{Incomplete}{}

Lau et al. define it as "inferring generalized actions based on examples of the
state changes resulting from demonstrated actions".

Number of traces needed must be small in order to be practical.

Lau et al. applied VSA to PBD by implementing SMARTedit, a system that induces
repetitive text-editing programs from as few as one or two examples.

\subsubsection{Execution Traces}

\pdfmarkupcomment[color=red]{Incomplete}{}

In \cite{Lau:traces:2003} Lau et. al apply version space algebra to infer
procedural programs from execution traces.
They present a language-neutral framework and implementation of a system that
``correct programs from a remarkably short 5.1 traces on average''.

\subsubsection{Programming By Examples}

\pdfmarkupcomment[color=green]{...}{Should I put the PBE section here instead?}


%%% Local Variables:
%%% mode: latex
%%% TeX-master: "../../Thesis"
%%% End:
