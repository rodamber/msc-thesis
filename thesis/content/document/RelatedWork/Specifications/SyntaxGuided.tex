\subsection{Syntax-Guided Synthesis}
\label{sec:syntax-guided}

Syntax-guided synthesis is an instance of program synthesis where the semantic
specification is complemented with some form of syntactic restriction over the
program candidates.

This approach was introduced in the \todo{SKETCH}{Explain its
shortcomings} system by Solar-Lezama \cite{Solar-Lezama:2008}, which allowed the
synthesis of imperative programs in a C-like language.
It was then generalized and formalized with the objective of ``formulating the
core computational problem common to many recent tools for program synthesis in
a canonical and logical manner'' \cite{Alur:sygus:2013}.

The syntactic \todo{restrictions}{Professor asks if this is the term used in the
literature and whether I mean ``constraints'' instead} are typically provided
in the form of a context-free grammar \cite{Alur:sygus:2013} or with sketches
\cite{Solar-Lezama:2008}. These restrictions provide structure to the set of
candidate programs, possibly resulting in more efficient search procedures,
while making this problem suitable for \todo{machine learning and inductive
inference}{missing references} \cite{Alur:sygus:2013}. They can also be used for
the purpose of performance optimizations, e.g., by limiting the search space to
implementations that only use a limited amount of lines of code. The learned
programs also tend to be more readable and explainable.

\subsubsection{Sketching and Metasketching}
\label{sec:sketching}

The idea of sketching is to provide skeletons of the programs we want to
synthesize, called \textit{sketches}, leaving missing details, called
\textit{holes}, for the synthesizer to fill.
The synthesizer is then directed by the high-level structure of the skeleton
while taking care of finding the low-level details according to user-specified
assertions.

Sketching is an accessible form of program synthesis, as it does not require
learning new specification languages, allowing the users to use the programming
model with which they are already familiarized.
