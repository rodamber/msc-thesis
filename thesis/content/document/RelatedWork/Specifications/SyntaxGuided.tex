\subsection{Syntax-Guided Synthesis}
\label{sec:sygus}

% \fixme{Not sure if this should be the first section in the chapter.}{}

% Not really a specification \textit{per se}

Syntax-guided synthesis is a form of program synthesis where the semantic
specification is complemented with some form of syntactic restriction over the
program candidates.

This approach was introduced in the \todo{SKETCH}{Explain its
shortcomings} system by Solar-Lezama \cite{Solar-Lezama:2008}, which allowed the
synthesis of imperative programs in a C-like language.
It was then generalized and formalized in \cite{Alur:sygus:2013}, with the
objective of ``formulating the core computational problem common to many recent
tools for program synthesis in a canonical and logical manner''.

The syntactic restrictions are typically provided in the form of a context-free
grammar \cite{Alur:sygus:2013} or with sketches \cite{Solar-Lezama:2008}.
These restrictions provide structure to the set of candidate programs, possibly
resulting in more efficient search procedures, while making this problem
suitable for \todo{machine learning and inductive inference}{missing
references} \cite{Alur:sygus:2013}. They can also be used for the purpose of
performance optimizations, e.g., by limiting the search space to implementations
that only use a limited amount of lines of code.
The learned programs also tend to be more readable and explainable.

% ------------------------------------------------------------------------------

\subsubsection{Sketching and Metasketching}
\label{sec:sketching}

% \todo{This subsubsection was left kind of dull after the modifications... Maybe
% it should be removed?}{}

The idea of sketching is to provide skeletons of the programs we want to
synthesize, called \textit{sketches}, leaving missing details, called
\textit{holes}, for the synthesizer to fill.
The synthesizer is then directed by the high-level structure of the skeleton
while taking care of finding the low-level details according to user-specified
assertions.

Sketching is an accessible form of program synthesis, as it does not require
learning new specification languages, allowing the users to use the programming
model with which they are already familiarized.

% \textit{Metasketches} \cite{Bornholt:2016:OSM} generalize sketches.

% Instead of just one sketch, a metasketch represents an 

% enabling a description of an infinite space of candidate programs with a
% countable, ordered set of finite sketches.

% This representation permits fine-grained control over the shape of the candidate
% space, which is critical for effective search.

% A metasketch additionally provides a means of assigning cost to programs and of
% directing the search toward lower-cost regions of the candidate space.

% \todo{Check out PSKETCH}{\cite{Gulwani2017}: page 32}
