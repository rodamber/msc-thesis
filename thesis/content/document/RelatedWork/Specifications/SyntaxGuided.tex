\subsection{Syntax-Guided Synthesis}
\label{sec:sygus}

\pdfmarkupcomment[color=red]{Needs rewrite}{} % sic

Syntax-guided synthesis is a form of program synthesis where the program
candidates are defined by a syntactic template \cite{Alur:sygus:2013}.
The semantic correctness can then be provided by any other kind of high-level
specification.

The user can use the candidate set to limit the search space implementations,
not only for its computational benefits, but also for the purpose of performance
optimizations, e.g., by limiting the search space to implementations that only
use a limited amount of lines of code.

This approach gives the programmer the flexibility to express the desired
artifact using a combination of syntactic and semantic constraints.

Because this formulation boils down to finding a correct expression from
the syntactic space of expressions, it lends itself to machine learning and
inductive inference.

The syntactic restrictions are typically provided in the form of a context-free
grammar \cite{Alur:sygus:2013} or with sketches \cite{Solar-Lezama:2008}.

\subsubsection{Sketching}
\label{sec:sketching}

\pdfmarkupcomment[color=red]{Incomplete}{}

The idea of sketching is to provide skeletons (sketches) of the programs with
missing details, called holes, left for the synthesizer to fill.
The synthesizer is then directed by the high-level structure of the skeleton
while taking care of finding the low-level details according to user-specified
assertions.
Sketching is an accessible form of program synthesis, as it does not require
learning new specification languages, allowing the users to use the programming
model with which they are already familiarized.

\pdfmarkupcomment[color=red]{Expand this paragraph.}{}
This approach was introduced in the SKETCH system by Solar-Lezama
\cite{Solar-Lezama:2008}, allowing for the synthesis of imperative programs
in a C-like language. ``SKETCH synthesizes integers that are bounded for each
hole.''

% \subsubsection{Language Grammar}
% \label{sec:language-grammar}

% The syntax of the programs can 

\subsubsection{Metasketching}
\label{sec:metasketching}

\pdfmarkupcomment[color=red]{Needs rewrite}{} % sic
\pdfmarkupcomment[color=red]{Expand}{}

Metasketches generalize sketches, enabling a description of an infinite space of
candidate programs with a countable, ordered set of finite sketches.
This representation permits fine-grained control over the shape of the candidate
space, which is critical for effective search.
A metasketch additionally provides a means of assigning cost to programs and of
directing the search toward lower-cost regions of the candidate space.


 % Might be merged into sec:sketching



%%% Local Variables:
%%% mode: latex
%%% TeX-master: "../../Thesis"
%%% End:
