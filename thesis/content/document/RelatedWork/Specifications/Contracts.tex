% This should be the first subsection and should be composed mainly of examples.
% Maybe it doesn't even need to be a subsection by itself and could be the
% beginning of the specifications section.

\subsection{Logical Specifications}

Logical specifications are the canonical way of introducing specifications. In
the sorting example from the introduction (\ref{sec:sorting-example}) we already
saw an example of this where the specifications were given as
pre/post-conditions over the inputs/outputs of the program. In that case, the
specifications were written as predicates on the host programming language.
Logical specifications \todo{may also be given as loop invariants or general
assertions in the code}{refs. leon?}, in order to give more hints to the
synthesizer.

\fixme{Colocar aqui um exemplo.Ideias: Leon, POPL'19 Polikarpova. Talvez ambos
porque é uma ideia fundamental que pode valer a pena explorar com detalhe.}{}

\fixme{Acho que estão em falta referências para trabalhos em concreto.}{}

% - sygus (nao queria introduzir aqui, ate porque provavelmente merece uma
% seccao propria)
% - leon
% - rosette

% Almost sic, page 8, \cite{Gulwani2017}:
% ``precise and succinct, but complete ones are often quite tricky to write''

% - expert-oriented
% - poder do sintetizador vs tipo de logica que suporta (primeira-ordem, etc)

% Type-based:
% synquid, lifty
