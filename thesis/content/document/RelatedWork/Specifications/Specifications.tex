\section{\textit{How to ask it}: Specifications and Syntactic Bias}
\label{sec:specifications}

% \fixme{General problems with this section}{}:
% \begin{itemize}
% \item Fails to explore the shortcomings of each approach.
% \item Lacks examples to explain the exposed ideas.
% \item Few references. I'm sure the coverage of the state of the art could be
%   better.
% \item Still incomplete.
% \end{itemize}

% Comentário da Professora: Usar mesma ordem e mesmos métodos que vêm a seguir.

The first part of solving a program synthesis problem is figuring out in which
way the users will express their intent.
That intent may be specified in many different ways, including:
  logical \cite{Itzhaky:SIS:2010}
  or type-based specifications \cite{Osera:2015:TPS, Frankle:2016:EST,
    Polikarpova:2016:PSP};
  syntax-guided methods \cite{Alur:sygus:2013} such as
  sketches \cite{Solar-Lezama:2008};
  or incomplete specifications such as
  input-output examples \cite{Frankle:2016:EST, Gulwani:2012:SDM, Leung:2015:IPS},
  demonstrations \cite{Lau2003}
  or program traces \cite{Lau:traces:2003}.
The kind of specification used typically depends on the particular problem at
hand, especially when taking into account the multitude of different user
backgrounds, and it might also dictate the type of techniques used to solve the
problem.

\subsection{Syntax-Guided Synthesis}
\label{sec:syntax-guided}

Syntax-guided synthesis is an instance of program synthesis where the semantic
specification is complemented with some form of syntactic restriction over the
program candidates.

This approach was introduced in the \todo{SKETCH}{Explain its
shortcomings} system by Solar-Lezama \cite{Solar-Lezama:2008}, which allowed the
synthesis of imperative programs in a C-like language.
It was then generalized and formalized with the objective of ``formulating the
core computational problem common to many recent tools for program synthesis in
a canonical and logical manner'' \cite{Alur:sygus:2013}.

The syntactic \todo{restrictions}{Professor asks if this is the term used in the
literature and whether I mean ``constraints'' instead} are typically provided
in the form of a context-free grammar \cite{Alur:sygus:2013} or with sketches
\cite{Solar-Lezama:2008}. These restrictions provide structure to the set of
candidate programs, possibly resulting in more efficient search procedures,
while making this problem suitable for \todo{machine learning and inductive
inference}{missing references} \cite{Alur:sygus:2013}. They can also be used for
the purpose of performance optimizations, e.g., by limiting the search space to
implementations that only use a limited amount of lines of code. The learned
programs also tend to be more readable and explainable.

\subsubsection{Sketching and Metasketching}
\label{sec:sketching}

The idea of sketching is to provide skeletons of the programs we want to
synthesize, called \textit{sketches}, leaving missing details, called
\textit{holes}, for the synthesizer to fill.
The synthesizer is then directed by the high-level structure of the skeleton
while taking care of finding the low-level details according to user-specified
assertions.

Sketching is an accessible form of program synthesis, as it does not require
learning new specification languages, allowing the users to use the programming
model with which they are already familiarized.

\subsection{Inductive Synthesis}
\label{sec:inductive}

Inductive synthesis is an instance of the program synthesis problem where the
constraints are under-specified.

\subsubsection{Programming By Demonstration}

\pdfmarkupcomment[color=red]{Incomplete}{}

Lau et al. define it as "inferring generalized actions based on examples of the
state changes resulting from demonstrated actions".

Number of traces needed must be small in order to be practical.

Lau et al. applied VSA to PBD by implementing SMARTedit, a system that induces
repetitive text-editing programs from as few as one or two examples.

\subsubsection{Execution Traces}

\pdfmarkupcomment[color=red]{Incomplete}{}

In \cite{Lau:traces:2003} Lau et. al apply version space algebra to infer
procedural programs from execution traces.
They present a language-neutral framework and implementation of a system that
``correct programs from a remarkably short 5.1 traces on average''.

\subsubsection{Programming By Examples}

\pdfmarkupcomment[color=green]{...}{Should I put the PBE section here instead?}


%%% Local Variables:
%%% mode: latex
%%% TeX-master: "../../Thesis"
%%% End:

% % This should be the first subsection and should be composed mainly of examples.
% Maybe it doesn't even need to be a subsection by itself and could be the
% beginning of the specifications section.

\subsection{Logical Specifications}

Logical specifications are the canonical way of introducing specifications. In
the sorting example from the introduction (\ref{sec:sorting-example}) we already
saw an example of this where the specifications were given as
pre/post-conditions over the inputs/outputs of the program. In that case, the
specifications were written as predicates on the host programming language.
Logical specifications \todo{may also be given as loop invariants or general
assertions in the code}{refs. leon?}, in order to give more hints to the
synthesizer.

\fixme{Colocar aqui um exemplo.Ideias: Leon, POPL'19 Polikarpova. Talvez ambos
porque é uma ideia fundamental que pode valer a pena explorar com detalhe.}{}

\fixme{Acho que estão em falta referências para trabalhos em concreto.}{}

% - sygus (nao queria introduzir aqui, ate porque provavelmente merece uma
% seccao propria)
% - leon
% - rosette

% Almost sic, page 8, \cite{Gulwani2017}:
% ``precise and succinct, but complete ones are often quite tricky to write''

% - expert-oriented
% - poder do sintetizador vs tipo de logica que suporta (primeira-ordem, etc)

% Type-based:
% synquid, lifty

\subsection{Programs}
\label{sec:programs}

A program can also be used as a specification and the job of the synthesizer is
then to find another program with the same semantics. We might also be
interested in programs that behave the same way as the original one, but, for
example, are more efficient (according to some metric).

Typically, the original programs are not made to be efficient, but to be easy to
read or to prove correct. They may also appear naturally as specifications in
certain use cases, such as
superoptimization~\cite{Phothilimthana:2016:SUS},
deobfuscation~\cite{Jha:oracle:2010} and
synthesis of program inverses~\cite{Srivastava:2011:PIS}.

