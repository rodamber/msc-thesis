\subsection{Inductive Synthesis}
\label{sec:inductive}

Inductive synthesis is an instance of the program synthesis problem where the
constraints are under-specified. Sometimes the domain we want to model is too
complex and a complete specification could be, not only as hard to produce, but
also of size of the same order of magnitude as the program itself. In other
cases, we might want the synthesizer to be as easy and intuitive to use as
possible for users with myriad different backgrounds.

\subsubsection{Programming By Examples}

\ac{PBE} is an instance of inductive synthesis where the specification is given
by \ac{IO} examples that the desired program must satisfy. Explicitly giving
examples can be preferred due to their ease of use, especially by
non-programmers, when compared to more technical kinds of specification, such as
logical formulas. The examples may either be \todo{positive}{explain me; example
ref} or \todo{negative}{explain me; example ref}. More generally, given some
(implicit) \ac{IO} pair, we may include asserting properties of the output
instead of specifying it completely \cite{Gulwani2017, Polozov:2015:FFI}. This
can be helpful if its impractical or impossible to write the output concretely,
e.g., if it is infinite.

\subsubsection{Programming By Demonstration}

% Watch What I Do – Programming by Demonstration

In \ac{PBD} the user does not write a specification \textit{per se}; instead the
synthesizer actively records user actions on concrete examples using them to
infer the intended program. The program must then generalize to be used with
different inputs. \ac{PBD} can be seen as a refinement of \ac{PBE} that
considers an entire execution trace (step-by-step instructions of the program
behavior on a given input) instead of a single \ac{IO} example. It depicts how
to achieve the corresponding output instead of just specifying what it should
be.

Though the concept is easy to understand, the task of the user can be tedious
and time-consuming. Therefore, the synthesizer must be able to infer the
intended program from a small set of user demonstrations. Ideally, it would also
be able effectively interact with and receive feedback from the user. However,
the concept might also be interesting when applied to non-interactive contexts,
such as \textit{reverse engineering}.

Lau et al. \cite{Lau2003} applied \ac{PBD} to the text-editing domain by
implementing SMARTedit, a system that induces repetitive text-editing programs
from as few as one or two examples . The system resembles familiar
keystroke-based macro interfaces, but it generalizes to a more robust program
that is likely to work in more situations.

In \cite{Lau:traces:2003} Lau et al. present a language-neutral framework and
implementation of a system that learns procedural programs from 5.1 traces on
average.

% ------------------------------------------------------------------------------
