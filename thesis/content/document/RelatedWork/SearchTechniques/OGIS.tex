\subsection{Oracle-Guided Inductive Synthesis}
\label{sec:ogis}

\textit{\Gls{ogis}} is an approach to program synthesis where the synthesizer is
split into two components: the \textit{learner} and the \textit{oracle}. The two
components communicate in an iterative \textit{query/response} cycle, as shown
in Figure~\ref{fig:ogis}. The learner implements the search strategy to
find the program and is parameterized by some form of program specification
and/or syntactic bias (see~\ref{sec:specifications}). The usefulness of the
oracle is defined by the type of queries it can handle and the properties of its
responses. The characteristics of these components are typically imposed by the
\todo{application}{give an example}.

% TODO figures should come at the top of the page and before the text that
% references it, when possible.
\begin{figure}[htb]
  \centering
  \begin{tikzpicture}
    [semithick, >=stealth, auto,
     rectangular/.style={rectangle, draw, rounded corners, text width=4cm,
       align=center, minimum size=1.5cm},
     spherical/.style={circle, draw, text width=2cm, align=center}]

    \node [rectangular] (S)  {Learner};
    \node [left=1.95cm of S, align=center] (I) {Specification\\and/or Syntactic Bias}
      edge [->] (S);
    \node [below=of S, align=center] {Program $p$\\or Fail}
      edge [<-] (S);
    \node [spherical] (V)  [right=3cm of S] {Oracle}
      ([yshift=0.2cm]S.east) edge [->, bend left]  node        {Query}    ([yshift=0.2cm]V.west)
      ([yshift=-.2cm]S.east) edge [<-, bend right] node [swap] {Response} ([yshift=-.2cm]V.west);
  \end{tikzpicture}
  \caption{\Acrlong{ogis}. Adapted from
    \protect\citeauthor{Jha:2017:TFS}~\protect\cite{Jha:2017:TFS}.}
  \label{fig:ogis}
\end{figure}

Typical queries and response types are some of the following~\cite{Jha:2017:TFS}:

\begin{itemize}
\item \textit{Membership queries}, where given an I/O example $x$ the oracle
  responds with the answer to whether $x$ is positive or not.
\item \textit{Positive (resp. negative) witness queries}, where the oracle
  responds with a positive (resp. negative) I/O example, if it can find any, or
  $\bot$ otherwise.
\item \textit{Counterexample queries}, where given a candidate program $p$ the
  oracle responds with a positive I/O counterexample that $p$ does not satisfy,
  if it can find any, or $\bot$ otherwise.
\item \textit{Correctness queries}, where given a candidate program $p$ the
  oracle responds with the answer to whether $p$ is correct or not. If it is not,
  the oracle responds with a positive I/O counterexample.
\item \textit{Verification queries}, where given program $p$ and specification
  $\phi$ the oracle responds with the answer to whether $p$ satisfies $\phi$ or
  not, or $\bot$ if it cannot find the answer.
\item \textit{Distinguishing input queries}, where given program $p$ and set $X$
  of I/O examples that $p$ \todo{satisfies}{did I define what satisfying a set of
    examples means?} the oracle responds with a new program $p'$ and a
  counterexample $x$ to $p$ that $p'$ satisfies along with all the other
  examples in $X$.
\end{itemize}
% TODO: Maybe switch to ''Jha et al.'s distinguishing inputs`'?
% FIXME: Is this ^^^ too close to the original?

An \gls{ogis} system responding to counterexample queries corresponds to the
\textit{\gls{cegis}} system, introduced by
\citeauthor{Solar-Lezama:2008}~\cite{Solar-Lezama:2008} in the context of the
SKETCH synthesizer. Correctness oracles are more powerful than counterexample
oracles because they are guaranteed to return a counterexample if the program is
not correct, where the counterexample oracles might not.

The concepts of \gls{ogis} was introduced by
\citeauthor{Jha:2017:TFS}~\cite{Jha:2017:TFS} as a generalization of \gls{cegis}
when they applied this idea to a \gls{pbe} synthesizer based on distinguishing
inputs in order to deobfuscate malware and to generate bit-manipulating
programs. Jha et al. further developed this idea by presenting a new theoretical
framework for inductive synthesis \cite{Jha:2017:TFS}.

% TODO Add example based on the last paragraph.

% FIXME This paragraph is confusing. What do I mean by ''not necessarily positive`'?
In general, the higher the capabilities of the oracles, the more expensive they
are to run. Distinguishing oracles are (typically) not as strong as
counterexample or correctness oracles as the returned counterexample is not
necessarily positive. To understand why they might be effectives tools we can
turn to the Bounded Observation Hypothesis \cite{Solar-Lezama:2008}, which
asserts that ``an implementation that works correctly for the common case and
for all the different corner cases is likely to work correctly for all inputs.''

In a setting of \todo{interactive program synthesis}{defined?} we could see the
users take the role of the oracles. However, the interesting cases are the ones
where the ratio between the amount of work the users are given and the
information given to the synthesizer is maximized. A system that frequently
queries the users for correctness checks would probably feel very cumbersome. On
the other hand, a system that queries for membership or positiveness checks
might be more realistic, as usually the \todo{users}{sometimes I refer to the
user in the singular and sometimes not. I probably should be paying more
attention to this.} have an idea of what sort of examples fit their desired
model.

