\section{\textit{How to answer it}: Searching and Solving}
\label{sec:search-techniques}

% You can find an overview of the most common search techniques used in the field
% of program synthesis.
\citeauthor{Alur:sygus:2013}~\cite{Alur:sygus:2013} presented an overview of the
most common search techniques used in the field of program synthesis.

\subsection{Oracle-Guided Inductive Synthesis}
\label{sec:ogis}

\textit{\Gls{ogis}} is an approach to program synthesis where the synthesizer is
split into two components: the \textit{learner} and the \textit{oracle}. The two
components communicate in an iterative \textit{query/response} cycle, as shown
in Figure~\ref{fig:ogis}. The learner implements the search strategy to
find the program and is parameterized by some form of program specification
and/or syntactic bias (see~\ref{sec:specifications}). The usefulness of the
oracle is defined by the type of queries it can handle and the properties of its
responses. The characteristics of these components are typically imposed by the
\todo{application}{give an example}.

\begin{figure}[htb]
  \centering
  \begin{tikzpicture}
    [semithick, >=stealth, auto,
     rectangular/.style={rectangle, draw, rounded corners, text width=4cm,
       align=center, minimum size=1.5cm},
     spherical/.style={circle, draw, text width=2cm, align=center}]

    \node [rectangular] (S)  {Learner};
    \node [left=1.95cm of S, align=center] (I) {Specification\\and/or Syntactic Bias}
      edge [->] (S);
    \node [below=of S, align=center] {Program $p$\\or Fail}
      edge [<-] (S);
    \node [spherical] (V)  [right=3cm of S] {Oracle}
      ([yshift=0.2cm]S.east) edge [->, bend left]  node        {Query}    ([yshift=0.2cm]V.west)
      ([yshift=-.2cm]S.east) edge [<-, bend right] node [swap] {Response} ([yshift=-.2cm]V.west);
  \end{tikzpicture}
  \caption{\Acrlong{ogis}. Adapted from
    \protect\citeauthor{Jha:2017:TFS}~\protect\cite{Jha:2017:TFS}.}
  \label{fig:ogis}
\end{figure}

Typical queries and response types are some of the following~\cite{Jha:2017:TFS}:

\begin{itemize}
\item \textit{Membership queries}, where given an I/O example $x$ the oracle
  responds with the answer to whether $x$ is positive or not.
\item \textit{Positive (resp. negative) witness queries}, where the oracle
  responds with a positive (resp. negative) I/O example, if it can find any, or
  $\bot$ otherwise.
\item \textit{Counterexample queries}, where given a candidate program $p$ the
  oracle responds with a positive I/O counterexample that $p$ does not satisfy,
  if it can find any, or $\bot$ otherwise.
\item \textit{Correctness queries}, where given a candidate program $p$ the
  oracle responds with the answer to whether $p$ is correct or not. If it is not,
  the oracle responds with a positive I/O counterexample.
\item \textit{Verification queries}, where given program $p$ and specification
  $\phi$ the oracle responds with the answer to whether $p$ satisfies $\phi$ or
  not, or $\bot$ if it cannot find the answer.
\item \textit{Distinguishing input queries}, where given program $p$ and set $X$
  of I/O examples that $p$ \todo{satisfies}{did I define what satisfying a set of
    examples means?} the oracle responds with a new program $p'$ and a
  counterexample $x$ to $p$ that $p'$ satisfies along with all the other
  examples in $X$.
\end{itemize}
% TODO: Maybe switch to ''Jha et al.'s distinguishing inputs`'?
% FIXME: Is this ^^^ too close to the original?

An \gls{ogis} system responding to counterexample queries corresponds to the
\textit{\gls{cegis}} system, introduced by \citeauthor{Solar-Lezama:2008}
~\cite{Solar-Lezama:2008} in the context of the SKETCH synthesizer. Correctness
oracles are more powerful than counterexample oracles because they are
guaranteed to return a counterexample if the program is not correct, where the
counterexample oracles might not.

The concept of \gls{ogis} and distinguishing inputs were introduced by Jha et.
al \cite{Jha:2017:TFS} as a generalization of \gls{cegis} when they applied
this idea in a example-based synthesizer in order to deobfuscate malware and to
generate bit-manipulating programs. Jha et al. further developed this idea by
presenting a new theoretical framework for inductive synthesis
\cite{Jha:2017:TFS}.

\fixme{Aqui tenho dúvida se devo ou não colocar pseudocódigo do algoritmo deles
(Como na figura 3.3. da overview) Ou se já é aprofundar demais. Note que eles
\textbf{não} fazem queries ao utilizador}{}

In general, the higher the capabilities of the oracles, the more expensive they
are to run. Distinguishing oracles are (typically) not as strong as
counterexample or correctness oracles as the returned counterexample is not
necessarily positive. To see why they might by such effectives tools we can
recur to the \todo{Bounded Observation Hypothesis}{caps?} \fixme{first
introduced}{as far as I know} by Solar-Lezama \cite{Solar-Lezama:2008}, which
asserts that ``an implementation that works correctly for the common case and
for all the different corner cases is likely to work correctly for all inputs.''

In a setting of \todo{interactive program synthesis}{defined?} we could see the
users take the role of the oracles. However, the interesting cases are the ones
where the ratio between the amount of work the users are given and the
information given to the synthesizer is maximized. A system that frequently
queries the users for correctness checks would probably feel very cumbersome. On
the other hand, a system that queries for membership or positiveness checks
might be more realistic, as usually the \todo{users}{sometimes I refer to the
user in the singular and sometimes not. I probably should be paying more
attention to this.} have an idea of what sort of examples fit their desired
model.


\subsection{Enumerative Search}
\label{sec:enumerative-search}

In the context of program synthesis an enumerative search consists of
enumerating programs by working the intrinsic structure of the program space to
guide the search. The programs can be ordered using many different program
metrics, the simplest one being program size~\cite{Alur:sygus:2013}, and pruned
by means of \todo{semantic equivalence checks}{Missing an example to explain
what I mean. Maybe add a simple example such as, how by associativity of
addition, x+y is the same as y+x} with respect to the specification.
Synthesizers based on enumerative search haven been some of the most effective
to synthesize short programs in complex program spaces~\cite{Gulwani2017}.

A simple enumerative search algorithm is described in Figure \fixme{???}{ainda
tenho que produzir esta figura}. The algorithm is parameterized by the
specification $\phi{}$ and two procedures, $next$, that given a set of programs
generates a new set of programs, and $prune$, that given a set of programs
filters out the \todo{unwanted ones}{Isto poderia ser dito de uma outra
forma...}. The algorithm works by iteratively calling the given procedures until
it finds a program that matches the specification.

In their overview of the field of program synthesis~\cite{Gulwani2017},
\citeauthor{Gulwani2017} present some simple enumerative search algorithms for
finding programs in program spaces defined by a \gls{cfg}, which we describe
next. The algorithms can be generalized to other types of program spaces such
as, e.g., partial programs.

\subsubsection{Top-Down Tree Search}
\label{sec:top-down-tree-search}

The first algorithm, shown in Figure \fixme{?}{ainda tenho que produzir esta
figura.}, takes as input a \gls{cfg} $G = (V, \Sigma{}, R, S)$ and a
specification $\phi{}$, and works by exploring the derivations of $G$ in a
top-down fashion. The algorithm stores the current programs in a priority queue,
$P$, and stores all the programs found so far in the set $P'$. Both $P$ and $P'$
are initialized with the partial program that corresponds to the start symbol
$S$ of $G$. The algorithm runs until it finds a program $p$ that matches the
specification $\phi{}$ or there are no more programs waiting in the queue
(meaning that the algorithm fails). At every iteration, we take the highest
priority program $p$ from the queue and check whether it satisfies $\phi{}$. If
yes, we return $p$. Otherwise, the algorithm finds new (possibly partial)
programs by applying the production rules of the grammar to $p$. The program
space is pruned in the next step by ignoring programs that are semantically
equivalent with respect to $\phi{}$ to programs already considered in the past
($P'$).

% TODO: Adicionar (possivelmente) um exemplo de um traço de execução do
% algoritmo.

\subsubsection{Bottom-Up Tree Search}
\label{sec:bottom-up-tree-search}

The second algorithm, shown in Figure \fixme{?}{ainda tenho que produzir esta
figura.}, also takes a \gls{cfg} $G = (V, \Sigma{}, R, S)$ and a specification
$\phi{}$, and works by exploring the derivations of the grammar in a bottom-up
dynamic programming fashion. This strategy has the advantage over the top-down
search that (in general) only complete programs may be evaluated for semantic
equivalence. The algorithm maintains a set of semantically equivalent
expressions, first considering the programs corresponding to leafs of the syntax
tree of the grammar $G$, and then composing them in order to build expressions
of increasing complexity, essentially applying the rules of the grammar in the
opposite direction.

% TODO: Check out references [4, 141] of Gulwani2017.

% This algorithm is shown in Figure \fixme{???}{ainda tenho que produzir esta
% figura}, using program size as the metric of program complexity.

\subsubsection{Bidirectional Tree Search}
\label{sec:bidirectional-search}

We can see that a top-down tree search starts from a set of input states, while
a bottom-up tree search starts from a set of output states. The third algorithm,
shown in Figure \fixme{?}{ainda tenho que produzir esta figura.}, combines the
previous two approaches, starting from both a set of input states and a set of
output states. It maintains both sets, evolving in the same way the previous two
algorithms did and stopping when it finds a state that belongs to both sets in a
sort of meet-in-the-middle approach.

% TODO: Explicar porque que isto e bom.
% Note that instructions _ and _ may be parallelizable.

% TODO: It might be the case that it makes more sense to expose SyGuS here
% instead of in the previous section. It's more of a learning bias than a kind
% of specification. The overview seems to do this.
% ON THE OTHER HAND: It might also make sense to put it in the previous section
% because it answers the question of how to ask it. Maybe the title should be
% changed to Specifications and Syntactic Bias?

% - deductive search
% - transformation-based search
% - enumerative search (top-down, bottom-up, bidirectional, offline exhaustive enumeration and composition)
% - constraint-based search (component based, sketch generation/completion, solver-aided programming, ILP, )
% - statistical search (machine learning, genetic programming, MCMC, sampling, probabilistic inference)
% - Neural-guided search
% - Graph neural networks
% - PBE VSAs, deduction-based, inverse semantics, type-based, ambiguity, intent
% - meta-synthesis, rosette, prose, sketch, solver-aided programming, domain separation
% - ranking: MCMC, VSA, ML, Metasketches
% - conflict-driven learning
% - Counterexample Guided Inductive Synthesis Modulo Theories*


% % \cite{Jha:2017:TFS}
% ``In deductive synthesis (e.g., [42]), a program is synthesized by constructively
% proving a theorem, employing logical inference and constraint solving.''

% ==============================================================================

% % \cite{Solar-Lezama:2008}
% ``Deductive systems rely on derivation, but this makes it hard to take advantage
% of partial information about the solution.''
