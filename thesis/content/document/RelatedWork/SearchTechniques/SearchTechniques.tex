\section{\textit{How to answer it}: Searching and Solving}
\label{sec:search-techniques}

% You can find an overview of the most common search techniques used in the field
% of program synthesis.
\citeauthor{Alur:sygus:2013}~\cite{Alur:sygus:2013} presented an overview of the
most common search techniques used in the field of program synthesis.

\subsection{Oracle-Guided Inductive Synthesis}
\label{sec:ogis}

\Acrfull{ogis} is an approach to program synthesis where the synthesizer is
split into two components: the \textit{learner} and the \textit{oracle}. The two
components communicate in an iterative \textit{query/response} cycle, as shown
in \fixme{Figure}{missing figure}. The learner implements the search strategy to
find the program and is parameterized by some form of program specification
and/or syntactic bias (see~\ref{sec:specifications}). The usefulness of the
oracle is defined by the type of queries it can handle and the properties of its
responses. The characteristics of these components are typically imposed by the
\todo{application}{give an example}.

% TODO: Talk about the Bounded Observation Hypothesis

\begin{figure}
  \centering
  \begin{tikzpicture}
    [semithick, >=stealth, auto,
    component/.style={rectangle, draw, rounded corners, text width=4cm,
      align=center, minimum size=1.5cm}]

    \node [component] (S)                   {Solver \\ (\textit   {search} component)};
    \node [component] (V)  [right=4cm of S] {Verifier \\ (\textit {validation} component)}
      ([yshift=0.2cm]S.east) edge [->] node        {Candidate program $P$}  ([yshift=0.2cm]V.west)
      ([yshift=-.2cm]S.east) edge [<-] node [swap] {Counterexample $x^{-}$} ([yshift=-.2cm]V.west);

    \node [above=of S] {Search space}
      edge [->] (S);
    \node [above=of V] {Specification $\phi$}
      edge [->] (V);

    \node [below=of S] {\xmark{} Fail}
      edge [<-] (S);
    \node [below=of V] {\checkmark{} Success}
      edge [<-] (V);
  \end{tikzpicture}
  \caption{Counterexample-guided inductive synthesis.}
  \label{fig:cegis}
\end{figure}

Common queries and response types include the following:

\begin{itemize}
\item Membership queries, where, given an I/O example $x$, the oracle responds
  with the answer to whether $x$ is positive or not.
\item Positive (resp. negative) witness queries, where the oracle responds with a
  positive (resp. negative) I/O example, if it can find any, or $\bot$, otherwise.
\item Counterexample queries, where, given a candidate program $p$, the oracle
  responds with an I/O counterexample that $p$ does not satisfy, if it can find
  any, or $\bot$, otherwise.
\item Correctness queries, where, given a candidate program $p$, the oracle
  responds with the answer to whether $p$ is correct or not. If it is not, the
  oracle responds with an I/O counterexample.
\item Verification queries, where, given a program $p$ and specification $\phi$,
  the oracle responds with the answer to whether $p$ satisfies $\phi$.
\item Distinguishing input queries, where, given a program $p$ and set of I/O
  examples $X$ that $p$ \todo{satisfies}{did I define what satisfying a set of
    examples means?}, the oracle responds with a new program $p'$ and a
  counterexample $x \not\in X$ such that $p'$ satisfies $X \cup x$ and $p$ does not satisfy
  $x$.
\end{itemize}

% TODO: Discussion on the various types of queries.
  % The counterexample is a \textit{constructive} proof that the
  % program is \todo{incorrect}{should I define somewhere what this means?}.

  % The distinguishing input query has been found useful in scenarios where it is
  % computationally hard to check correctness using the specificatio, such as in
  % malware deobfuscation [30].

  % semantic equivalence, see overview

  % counterexample vs correctness
  % counterexample vs distinguishing (note that $x \not\in X$, but that doesn't
  % mean that $P'$ is the program we want)

% TODO: Where did it come from?

% ------------------------------------------------------------------------------

% - Introduce the concept of OGIS. (Why are we talking about it? Second-order problem reduction?)

% - Describe OGIS, the concept of oracles and distinguishing inputs, and comment
% on them (e.g., counterexample -> CEGIS, validation oracles may be too
% expensive, distinguishing inputs lend themselves to interactive program
% synthesis, etc)

% - Relate the oracles (in particular, the distinguishing oracle and relate it
% to interactive program synthesis)

% - Maybe add a note on the complexity of OGIS.

% In the paper of OGIS, they describe a component-based approach, where each
% component's specification is a set of I/O examples (TODO: check how they do this).
% This is in contrast to the approach used in Ruben's paper, where the
% specifications are more general.
% Moreover, the program specification in OGIS seems to be a set of I/O examples
% (TODO: check this).

% % Bounded Observation Hypothesis
% % \cite{Solar-Lezama:2008}
% ``The crucial observation that makes sketch [inductive] synthesis possible is
% that for many sketches, an implementation that works correctly for the common
% case and for all the different corner cases is likely to work correctly for all
% inputs.''
% % It then gives the example of doubly-linked lists.

% ------------------------------------------------------------------------------

% % \cite{Solar-Lezama:2008}
% ``Inductive synthesis is the process of generating a program from concrete
% observations of its behavior, where an observation describes the expected
% behavior of the program on a specific input. The inductive synthesizer uses each
% new observation to refine its hypothesis about what the correct program should
% be until it converges to a solution. Inductive synthesis had its origin in the
% work by Gold [33] on language learning, and the pioneering work by Shapiro [57]
% on inductive synthesis and its application to algorithmic debugging among
% others.''


\subsection{Enumerative Search}
\label{sec:enumerative-search}

In the context of program synthesis, enumerative search consists of enumerating
programs by working the intrinsic structure of the program space to guide the
search.
The programs can be ordered using many different program metrics, the simplest
one being program size, and pruned by means of semantic equivalence checks with
respect to the specification.
Perhaps surprisingly, synthesizers based on enumerative search have been some of
the most effective to synthesize short programs in complex program spaces.
A reason why is that the search can be precisely tailored for the domain at
hand, encoding domain-specific heuristics and case-by-case scenarios that
result in highly effective pruning strategies.

In their overview of the field of program synthesis~\cite{Gulwani2017},
\citeauthor{Gulwani2017} describe some enumerative search algorithms for finding
programs in program spaces defined by a \glsfmtfull{cfg}, which we describe
next.
The algorithms can be generalized to other types of program spaces such as,
e.g., sketches.

\subsubsection{Top-Down Tree Search}
\label{sec:top-down-tree-search}

\begin{algorithm}
  \DontPrintSemicolon
  \LinesNotNumbered

  \SetKw{Not}{not}

  \SetKwInOut{Input}{input}
  \SetKwInOut{Output}{output}

  \SetKwFunction{Queue}{Queue}
  \SetKwFunction{PopFirst}{popFirst}
  \SetKwFunction{NonTerminals}{nonTerminals}
  \SetKwFunction{Subsumed}{subsumed}

  \Input{A specification $\phi{}$ and a \gls{cfg} $G$}
  \Output{A program $p$ in the grammar $G$ that satisfies $\phi{}$}
  \Begin{
    $P\leftarrow$\Queue{}\;
    $P'\leftarrow \{S\}$\;

    \While{$P\neq \emptyset$}{
      $p\leftarrow$\PopFirst{$P$}\;
      \If{$p$ satisfies $\phi{}$}{
        \Return{$p$}\;
      }
      \For{$a \in$ \NonTerminals{$p$}}{
        \For{$b \in \{b | (a,b) \in R\}$}{
          \If{\Not \Subsumed{$p[a\rightarrow b]$, $P'$}}{
            $P\leftarrow P \cup \{p[a\rightarrow b]\}$\;
            $P'\leftarrow P' \cup \{p[a\rightarrow b]\}$\;
          }
        }
      }
    }
  }
  \caption{Enumerative Top-Down Tree Search.
    Adapted from \citeauthor{Gulwani2017}'s overview~\cite{Gulwani2017}.}
  \label{alg:enum-top-down}
\end{algorithm}

The first enumerative strategy is the top-down tree search algorithm
(Algorithm~\ref{alg:enum-top-down}).
It takes as input a \gls{cfg} $G = (V, \Sigma{}, R, S)$ and a specification
$\phi{}$, and works by exploring the derivations of $G$ in a best-first top-down
fashion.
The algorithm stores the current programs in a priority queue, $P$, and stores
all the programs found so far in the set $P'$.
Both $P$ and $P'$ are initialized with the partial program that corresponds to
the start symbol $S$ of $G$.
The algorithm runs until it finds a program $p$ that matches the specification
$\phi{}$ or there are no more programs waiting in the queue (meaning that the
algorithm fails).
At every iteration, we take the program $p$ with the highest priority from the
queue and check whether it satisfies $\phi{}$.
If yes, we return $p$.
Otherwise, the algorithm finds new (possibly partial) programs by applying the
production rules of the grammar to $p$.
The program space is pruned in the next step by ignoring programs that are
semantically equivalent (with respect to $\phi{}$) to programs already
considered in the past (i.e., subsumed within $P'$).

\subsubsection{Bottom-Up Tree Search}
\label{sec:bottom-up-tree-search}

\begin{algorithm}
  \DontPrintSemicolon
  \LinesNotNumbered

  \SetKw{Not}{not}

  \SetKwInOut{Input}{input}
  \SetKwInOut{Output}{output}

  \SetKwFunction{EnumerateExprs}{enumerateExprs}
  \SetKwFunction{Subsumed}{subsumed}

  \Input{A specification $\phi{}$ and a \gls{cfg} $G$}
  \Output{A program $p$ in the grammar $G$ that satisfies $\phi{}$}
  \Begin{
    $P\leftarrow \emptyset$\;
    \For{progSize $= 1,2,\ldots$}{
      $P'\leftarrow$\EnumerateExprs{$G$, $E$, progSize}\;
      \For{$p \in P'$}{
        \If{$p$ satisfies $\phi{}$}{
          \Return{$p$}\;
        }
        \If{\Not \Subsumed{$p$, $P$}}{
          $P\leftarrow P \cup \{p\}$
        }
      }
    }
  }
  \caption{Enumerative Bottom-Up Tree Search.
    Adapted from \citeauthor{Gulwani2017}'s overview~\cite{Gulwani2017}.}
  \label{alg:enum-bottom-up}
\end{algorithm}

The bottom-up tree search algorithm (Algorithm~\ref{alg:enum-bottom-up}) is the
dual to top-down tree search algorithm.
It also takes a \gls{cfg} $G = (V, \Sigma{}, R, S)$ and a specification
$\phi{}$, and works by exploring the derivations of the grammar in a bottom-up
dynamic programming fashion.
This strategy has the advantage over the top-down search that (in general) only
complete programs may be evaluated for semantic equivalence.
The algorithm maintains a set of equivalent expressions, first considering the
programs corresponding to leafs of the syntax tree of the grammar $G$, and then
composing them in order to build expressions of increasing complexity,
essentially applying the rules of the grammar in the opposite direction.

\subsubsection{Bidirectional Tree Search}
\label{sec:bidirectional-search}

We can see that a top-down tree search starts from a set of input states, while
a bottom-up tree search starts from a set of output states.
In both approaches the size of the search space grows exponentially with size of
the programs.
The bidirectional tree search algorithm tries to attenuate this problem by
combining the previous two approaches, starting from both a set of input states
and a set of output states.
It maintains both sets, evolving in the same way as the previous two
algorithms, and stops when it finds a state that belongs to both sets in a
sort of meet-in-the-middle approach.


% TODO: It might be the case that it makes more sense to expose SyGuS here
% instead of in the previous section. It's more of a learning bias than a kind
% of specification. The overview seems to do this.
% ON THE OTHER HAND: It might also make sense to put it in the previous section
% because it answers the question of how to ask it. Maybe the title should be
% changed to Specifications and Syntactic Bias?

% - deductive search
% - transformation-based search
% - enumerative search (top-down, bottom-up, bidirectional, offline exhaustive enumeration and composition)
% - constraint-based search (component based, sketch generation/completion, solver-aided programming, ILP, )
% - statistical search (machine learning, genetic programming, MCMC, sampling, probabilistic inference)
% - Neural-guided search
% - Graph neural networks
% - PBE VSAs, deduction-based, inverse semantics, type-based, ambiguity, intent
% - meta-synthesis, rosette, prose, sketch, solver-aided programming, domain separation
% - ranking: MCMC, VSA, ML, Metasketches
% - conflict-driven learning
% - Counterexample Guided Inductive Synthesis Modulo Theories*


% % \cite{Jha:2017:TFS}
% ``In deductive synthesis (e.g., [42]), a program is synthesized by constructively
% proving a theorem, employing logical inference and constraint solving.''

% ==============================================================================

% % \cite{Solar-Lezama:2008}
% ``Deductive systems rely on derivation, but this makes it hard to take advantage
% of partial information about the solution.''
