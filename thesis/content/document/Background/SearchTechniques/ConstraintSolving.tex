\subsection{Constraint Solving}
\label{sec:constraint-solving}

Another approach to program synthesis is to reduce the problem to
constraint solving by the use of off-the-shelf automated constraint
solvers~\cite{Shi:2019:FCS,Feng:2018:PSU,Feng:2017:CST,Feng:2017:CSC,Solar-Lezama:2008,Jha:oracle:2010}
(typically SAT or SMT solvers).
The idea is to encode the specification in a logical constraint whose solution
corresponds to the desired program.
\citeauthor{Gulwani2017}~\cite{Gulwani2017} illustrate this in a simple way with
an example which we show here.
This example also serves as a short introduction to the SMT-LIB
language~\cite{BarFT-RR-17}, a standard language for SMT solvers.
Suppose our programs are composed of operations over two input bitvectors, $x$
and $y$, of length eight:
%
\begin{align*}
  \text{program }P    &::= plus(E,E) \OR mul(E,E) \OR shl(E,C) \OR shr(E,C)    \\
  \text{expression }E &::= x \OR C                                             \\
  \text{constant }C   &::= 00000000_2 \OR 00000001_2 \OR \ldots \OR 11111111_2 \\
\end{align*}
%
\noindent
We consider an expression to be either the input variable $x$ or an 8-bit
constant.
A program consists of additions and multiplications between expressions, or of
shift left/right operations over an expression by a constant.
We can declare the type of bitvectors of length 8 in SMT-LIB as:
%
\begin{lstlisting}[language=Lisp,numbers=left,
  firstnumber=1,
  morekeywords={define-sort},
  xleftmargin=.2\textwidth, xrightmargin=.2\textwidth]
  (define-sort Bit8 () (_ BitVec 8))
\end{lstlisting}
%
\noindent
To encode the grammar of well-formed programs we first need to introduce the
constant symbols \texttt{hP}, \texttt{hE0}, \texttt{hE1}, \texttt{c0} and
\texttt{c1}, as well as the function symbol \texttt{prog}:
%
\begin{lstlisting}[language=Lisp,
  numbers=left,
  firstnumber=2,
  morekeywords={declare-const,define-fun,ite},
  xleftmargin=.2\textwidth, xrightmargin=.2\textwidth]
  (declare-const hP Int)
  (declare-const hE0 Bool)
  (declare-const hE1 Bool)
  (declare-const c0 Bit8)
  (declare-const c1 Bit8)

  (define-fun prog ((x Bit8)) Bit8
  (let ((left (ite hE0 c0 x))
       (right (ite hE1 c1 x)))
    (ite (= hP 0) (bvadd left right)
    (ite (= hP 1) (bvmul left right)
    (ite (= hP 2) (bvshl left c1)
                  (bvlshr left c1))))))
\end{lstlisting}
%
The symbol \texttt{hP} encodes the choice of which language construct to pick,
while the symbols \texttt{hE0}, \texttt{hE1} encode the choice of whether
\texttt{left} and \texttt{right} are assigned the values of constants
(\texttt{c0}, \texttt{c1}) or the value of the input (\texttt{x}).
An assignment to these constant symbols corresponds to a valid program in our
grammar, if we ensure that \texttt{hP} is in a valid range.
We can use \texttt{assert} to introduce new clauses that must be satisfied:
%
\begin{lstlisting}[language=Lisp,
  numbers=left,
  firstnumber=15,
  morekeywords={assert},
  xleftmargin=.2\textwidth, xrightmargin=.2\textwidth]
  (assert (>= hP 0))
  (assert (<  hP 4))
\end{lstlisting}
%
Finally, we can also use \texttt{assert} to encode a semantic specification.
For example, suppose we are interested in a bitvector program $P$ that, for all
input $x$, its output is always positive, i.e., $\forall x\ldotp P(x) \geq 0$.
In SMT-LIB that can be written as:
%
\begin{lstlisting}[language=Lisp,
  numbers=left,
  firstnumber=17,
  morekeywords={assert},
  xleftmargin=.2\textwidth, xrightmargin=.2\textwidth]
  (assert (forall ((x Bit8)) (bvsge (prog x) #x00)))
\end{lstlisting}
 
This example shows an end-to-end constraint solving approach to program
synthesis. However, encoding the problem this way can sometimes be non-trivial
or time-consuming.
This led to the appearance of the concept of \textit{solver-aided programming},
where programming languages are enlarged with high-level constructs that give
the user access to synthesis without having to deal with the constraint solvers
directly.
For example, \citeauthor{Gulwani2017} describe the SKETCH system as a ``compiler
[that] relies on a SAT solver to materialize some language constructs''.
ROSETTE~\cite{Torlak:2013:GSL} is a framework for developing solver-aided
programming languages embedded in Racket that provides constructs not only for
synthesis, but also for verification, debugging and angelic execution.
