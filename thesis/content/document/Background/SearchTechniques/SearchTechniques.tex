\section{Search Techniques}
\label{sec:search-techniques}

The second part of solving a program synthesis problem is deciding which search
technique to apply in order to find the intended program.
First, we want to ensure that the program satisfies the semantic and syntactic
specifications.
Second, we want to leverage the specifications and the knowledge we have from
the problem domain in order to guide the search.
Common search techniques are
% deductive search (Section~\ref{sec:deductive-synthesis}),
enumerative search~\cite{Alur:2017:SEP}
(Section~\ref{sec:enumerative-search}),
stochastic search~\cite{Schkufza:2013:SS,Singh:ranking:2015}
(Section~\ref{sec:stochastic-search}), and
constraint solving~\cite{Feng:2018:PSU,Feng:2017:CST,Feng:2017:CSC}
(Section~\ref{sec:constraint-solving}).
Modern synthesizers usually apply a combination of those, and so we can also
talk about frameworks for structuring their construction, such as
\gls{cegis}~\cite{Solar-Lezama:2008},
\gls{cegis}($\mathcal{T}$)~\cite{Abate:2018:CMT}, and
\gls{ogis}~\cite{Jha:2017:TFS}
(Section~\ref{sec:ogis}).


% \subsection{Deductive Synthesis}
\label{sec:deductive-synthesis}

The fundamental idea in deductive synthesis is to extract a program
automatically from the constructive proof of a theorem induced by the
specification. Typically, the kinds of specifications associated with this
method of synthesis are logical relations between inputs and the outputs of the
program. 

These systems tend to require a complete formal specification, and the level
of expertise needed to use them turn them easily unapproachable by the user
which is not mathematically inclined.

\fixme{Expand on this. References: Polikarpova:2019:SSH, Manna:1971:TAP,
  Green:1969:ATP}{These refs. are old!}

\subsection{Enumerative Search}
\label{sec:enumerative-search}

In the context of program synthesis, enumerative search consists of enumerating
programs by working the intrinsic structure of the program space to guide the
search.
The programs can be ordered using many different program metrics, the simplest
one being program size, and pruned by means of semantic equivalence checks with
respect to the specification.
Perhaps surprisingly, synthesizers based on enumerative search have been some of
the most effective to synthesize short programs in complex program spaces.
A reason why is that the search can be precisely tailored for the domain at
hand, encoding domain-specific heuristics and case-by-case scenarios that
result in highly effective pruning strategies.

In their overview of the field of program synthesis~\cite{Gulwani2017},
\citeauthor{Gulwani2017} describe some enumerative search algorithms for finding
programs in program spaces defined by a \glsfmtfull{cfg}, which we describe
next.
The algorithms can be generalized to other types of program spaces such as,
e.g., sketches.

\subsubsection{Top-Down Tree Search}
\label{sec:top-down-tree-search}

\begin{algorithm}
  \DontPrintSemicolon
  \LinesNotNumbered

  \SetKw{Not}{not}

  \SetKwInOut{Input}{input}
  \SetKwInOut{Output}{output}

  \SetKwFunction{Queue}{Queue}
  \SetKwFunction{PopFirst}{popFirst}
  \SetKwFunction{NonTerminals}{nonTerminals}
  \SetKwFunction{Subsumed}{subsumed}

  \Input{A specification $\phi{}$ and a \gls{cfg} $G$}
  \Output{A program $p$ in the grammar $G$ that satisfies $\phi{}$}
  \Begin{
    $P\leftarrow$\Queue{}\;
    $P'\leftarrow \{S\}$\;

    \While{$P\neq \emptyset$}{
      $p\leftarrow$\PopFirst{$P$}\;
      \If{$p$ satisfies $\phi{}$}{
        \Return{$p$}\;
      }
      \For{$a \in$ \NonTerminals{$p$}}{
        \For{$b \in \{b | (a,b) \in R\}$}{
          \If{\Not \Subsumed{$p[a\rightarrow b]$, $P'$}}{
            $P\leftarrow P \cup \{p[a\rightarrow b]\}$\;
            $P'\leftarrow P' \cup \{p[a\rightarrow b]\}$\;
          }
        }
      }
    }
  }
  \caption{Enumerative Top-Down Tree Search.
    Adapted from \citeauthor{Gulwani2017}'s overview~\cite{Gulwani2017}.}
  \label{alg:enum-top-down}
\end{algorithm}

The first enumerative strategy is the top-down tree search algorithm
(Algorithm~\ref{alg:enum-top-down}).
It takes as input a \gls{cfg} $G = (V, \Sigma{}, R, S)$ and a specification
$\phi{}$, and works by exploring the derivations of $G$ in a best-first top-down
fashion.
The algorithm stores the current programs in a priority queue, $P$, and stores
all the programs found so far in the set $P'$.
Both $P$ and $P'$ are initialized with the partial program that corresponds to
the start symbol $S$ of $G$.
The algorithm runs until it finds a program $p$ that matches the specification
$\phi{}$ or there are no more programs waiting in the queue (meaning that the
algorithm fails).
At every iteration, we take the program $p$ with the highest priority from the
queue and check whether it satisfies $\phi{}$.
If yes, we return $p$.
Otherwise, the algorithm finds new (possibly partial) programs by applying the
production rules of the grammar to $p$.
The program space is pruned in the next step by ignoring programs that are
semantically equivalent (with respect to $\phi{}$) to programs already
considered in the past (i.e., subsumed within $P'$).

\subsubsection{Bottom-Up Tree Search}
\label{sec:bottom-up-tree-search}

\begin{algorithm}
  \DontPrintSemicolon
  \LinesNotNumbered

  \SetKw{Not}{not}

  \SetKwInOut{Input}{input}
  \SetKwInOut{Output}{output}

  \SetKwFunction{EnumerateExprs}{enumerateExprs}
  \SetKwFunction{Subsumed}{subsumed}

  \Input{A specification $\phi{}$ and a \gls{cfg} $G$}
  \Output{A program $p$ in the grammar $G$ that satisfies $\phi{}$}
  \Begin{
    $P\leftarrow \emptyset$\;
    \For{progSize $= 1,2,\ldots$}{
      $P'\leftarrow$\EnumerateExprs{$G$, $E$, progSize}\;
      \For{$p \in P'$}{
        \If{$p$ satisfies $\phi{}$}{
          \Return{$p$}\;
        }
        \If{\Not \Subsumed{$p$, $P$}}{
          $P\leftarrow P \cup \{p\}$
        }
      }
    }
  }
  \caption{Enumerative Bottom-Up Tree Search.
    Adapted from \citeauthor{Gulwani2017}'s overview~\cite{Gulwani2017}.}
  \label{alg:enum-bottom-up}
\end{algorithm}

The bottom-up tree search algorithm (Algorithm~\ref{alg:enum-bottom-up}) is the
dual to top-down tree search algorithm.
It also takes a \gls{cfg} $G = (V, \Sigma{}, R, S)$ and a specification
$\phi{}$, and works by exploring the derivations of the grammar in a bottom-up
dynamic programming fashion.
This strategy has the advantage over the top-down search that (in general) only
complete programs may be evaluated for semantic equivalence.
The algorithm maintains a set of equivalent expressions, first considering the
programs corresponding to leafs of the syntax tree of the grammar $G$, and then
composing them in order to build expressions of increasing complexity,
essentially applying the rules of the grammar in the opposite direction.

\subsubsection{Bidirectional Tree Search}
\label{sec:bidirectional-search}

We can see that a top-down tree search starts from a set of input states, while
a bottom-up tree search starts from a set of output states.
In both approaches the size of the search space grows exponentially with size of
the programs.
The bidirectional tree search algorithm tries to attenuate this problem by
combining the previous two approaches, starting from both a set of input states
and a set of output states.
It maintains both sets, evolving in the same way as the previous two
algorithms, and stops when it finds a state that belongs to both sets in a
sort of meet-in-the-middle approach.

\subsection{Stochastic Search}
\label{sec:stochastic-search}

Stochastic search is an approach to program synthesis where the synthesizer uses
probabilistic reasoning to learn a program conditioned on the specification. Two
typical approaches to applying stochastic search in synthesizers are as follows.
One is to learn a \textit{guiding function} that works on top of an enumerative
search algorithm to help it predict which program from the search space is most
likely to meet the desired specification. Another one is to learn a probability
distribution over the search space in order to sample programs that are more
promising. Below, we can find high-level descriptions of specific instances of
these kinds of synthesis.

\subsubsection{Guiding the Search}
\label{sec:guiding}

\subsubsection{Sampling the Search Space}
\label{sec:sampling}

In this section we describe the stochastic synthesizer used by
\citeauthor{Alur:sygus:2013} in their syntax-guided synthesis
paper~\cite{Alur:sygus:2013}. Their synthesizer learns from examples and is
adapted from work on superoptimization of loop-free binary
programs~\cite{Schkufza:2013:SS}. They define a score function, $Score$, that
measures the extent which a given program is consistent with the specification.
Then they perform a probabilistic walk over the search space while maximizing
this score function. The algorithm works by first picking a program $p$ of fixed
size $n$ uniformly at random. They then pick a node from its parse tree
uniformly at random and consider the subprogram rooted at that node. They then
substitute it with another subprogram of same size and type, chosen uniformly at
random, obtaining a new program $p'$. The probability of discarding $p$ for $p'$
is given by the formula $min(1, Score(p')/Score(p))$. It remains to say how to
pick the value of $n$. Typically we do not know the size of the desired program
from the start. In order to tackle this problem, they start by fixing its value
at $n = 1$ and at each iteration change its value to $min(1, n\pm{}1)$ with with
some small probability (default is 0.01).

\subsection{Constraint Solving}
\label{sec:constraint-solving}

Another approach to program synthesis is to reduce the problem to that of
constraint solving by the use of off-the-shelf automated constraint
solvers~\cite{Shi:2019:FCS,Feng:2018:PSU,Feng:2017:CST,Feng:2017:CSC,Solar-Lezama:2008,Jha:oracle:2010}
(typically SAT or SMT solvers).

One idea is to encode the specification in a logical constraint whose solution
corresponds to the desired program. \citeauthor{Gulwani2017} illustrate this
idea nicely with an example~\cite{Gulwani2017} which we adapt here.
Suppose our programs are composed of operations over two input bitvectors, $x$
and $y$ of length eight:

\begin{figure}[h!]
  \centering
  \includegraphics[width=\textwidth]{assets/constraint-solving-example.png}
  \caption{FIXME: This example is a placeholder and should be adapted.}
\end{figure}

We consider an expression to be the input variable $x$ or an 8-bit constant. A
program consists of additions and multiplications between expressions, or of
shift left/right operations over an expression by a constant.

Imagine we are interested in a program that \todo{<insert interesting bit
  twiddling hack>}{check Hacker's Delight}. In the theory of
\todo{bitvectors}{missing ref}, the program we are interested in can be encoded
by the formula \todo{<insert formula here>}{add formula correspondent to the
hack}. To find the program we can write the constraints in the SMT-LIB format
and feed them to an SMT solver:

\begin{figure}[h!]
  \centering
  \includegraphics[width=\textwidth]{assets/constraint-solving-smtlib.png}
  \caption{FIXME: This example is a placeholder and should be adapted.}
\end{figure}

This example shows an end-to-end constraint solving approach to program
synthesis. However, enconding the problem this way can sometimes be non-trivial
or time-consuming. This idea led to the appearance of the concept of
\textit{solver-aided programming}, where programming languages are enlarged with
high-level constructs that give the user access to synthesis without having to
deal with the constraint solvers directly.

For example, \citeauthor{Gulwani2017} describe the SKETCH system as a ``compiler
[that] relies on a SAT solver to materialize some language constructs''.
ROSETTE~\cite{Torlak:2013:GSL} is a framework for developing solver-aided
programming languages embedded in Racket that provides constructs not only for
synthesis, but also for verification, debugging and angelic execution.

\subsection{Oracle-Guided Inductive Synthesis}
\label{sec:ogis}

\Acrfull{ogis} is an approach to program synthesis where the synthesizer is
split into two components: the \textit{learner} and the \textit{oracle}. The two
components communicate in an iterative \textit{query/response} cycle, as shown
in \fixme{Figure}{missing figure}. The learner implements the search strategy to
find the program and is parameterized by some form of program specification
and/or syntactic bias (see~\ref{sec:specifications}). The usefulness of the
oracle is defined by the type of queries it can handle and the properties of its
responses. The characteristics of these components are typically imposed by the
\todo{application}{give an example}.

% TODO: Talk about the Bounded Observation Hypothesis

\begin{figure}
  \centering
  \begin{tikzpicture}
    [semithick, >=stealth, auto,
    component/.style={rectangle, draw, rounded corners, text width=4cm,
      align=center, minimum size=1.5cm}]

    \node [component] (S)                   {Solver \\ (\textit   {search} component)};
    \node [component] (V)  [right=4cm of S] {Verifier \\ (\textit {validation} component)}
      ([yshift=0.2cm]S.east) edge [->] node        {Candidate program $P$}  ([yshift=0.2cm]V.west)
      ([yshift=-.2cm]S.east) edge [<-] node [swap] {Counterexample $x^{-}$} ([yshift=-.2cm]V.west);

    \node [above=of S] {Search space}
      edge [->] (S);
    \node [above=of V] {Specification $\phi$}
      edge [->] (V);

    \node [below=of S] {\xmark{} Fail}
      edge [<-] (S);
    \node [below=of V] {\checkmark{} Success}
      edge [<-] (V);
  \end{tikzpicture}
  \caption{Counterexample-guided inductive synthesis.}
  \label{fig:cegis}
\end{figure}

Common queries and response types include the following:

\begin{itemize}
\item Membership queries, where, given an I/O example $x$, the oracle responds
  with the answer to whether $x$ is positive or not.
\item Positive (resp. negative) witness queries, where the oracle responds with a
  positive (resp. negative) I/O example, if it can find any, or $\bot$, otherwise.
\item Counterexample queries, where, given a candidate program $p$, the oracle
  responds with an I/O counterexample that $p$ does not satisfy, if it can find
  any, or $\bot$, otherwise.
\item Correctness queries, where, given a candidate program $p$, the oracle
  responds with the answer to whether $p$ is correct or not. If it is not, the
  oracle responds with an I/O counterexample.
\item Verification queries, where, given a program $p$ and specification $\phi$,
  the oracle responds with the answer to whether $p$ satisfies $\phi$.
\item Distinguishing input queries, where, given a program $p$ and set of I/O
  examples $X$ that $p$ \todo{satisfies}{did I define what satisfying a set of
    examples means?}, the oracle responds with a new program $p'$ and a
  counterexample $x \not\in X$ such that $p'$ satisfies $X \cup x$ and $p$ does not satisfy
  $x$.
\end{itemize}

% TODO: Discussion on the various types of queries.
  % The counterexample is a \textit{constructive} proof that the
  % program is \todo{incorrect}{should I define somewhere what this means?}.

  % The distinguishing input query has been found useful in scenarios where it is
  % computationally hard to check correctness using the specificatio, such as in
  % malware deobfuscation [30].

  % semantic equivalence, see overview

  % counterexample vs correctness
  % counterexample vs distinguishing (note that $x \not\in X$, but that doesn't
  % mean that $P'$ is the program we want)

% TODO: Where did it come from?

% ------------------------------------------------------------------------------

% - Introduce the concept of OGIS. (Why are we talking about it? Second-order problem reduction?)

% - Describe OGIS, the concept of oracles and distinguishing inputs, and comment
% on them (e.g., counterexample -> CEGIS, validation oracles may be too
% expensive, distinguishing inputs lend themselves to interactive program
% synthesis, etc)

% - Relate the oracles (in particular, the distinguishing oracle and relate it
% to interactive program synthesis)

% - Maybe add a note on the complexity of OGIS.

% In the paper of OGIS, they describe a component-based approach, where each
% component's specification is a set of I/O examples (TODO: check how they do this).
% This is in contrast to the approach used in Ruben's paper, where the
% specifications are more general.
% Moreover, the program specification in OGIS seems to be a set of I/O examples
% (TODO: check this).

% % Bounded Observation Hypothesis
% % \cite{Solar-Lezama:2008}
% ``The crucial observation that makes sketch [inductive] synthesis possible is
% that for many sketches, an implementation that works correctly for the common
% case and for all the different corner cases is likely to work correctly for all
% inputs.''
% % It then gives the example of doubly-linked lists.

% ------------------------------------------------------------------------------

% % \cite{Solar-Lezama:2008}
% ``Inductive synthesis is the process of generating a program from concrete
% observations of its behavior, where an observation describes the expected
% behavior of the program on a specific input. The inductive synthesizer uses each
% new observation to refine its hypothesis about what the correct program should
% be until it converges to a solution. Inductive synthesis had its origin in the
% work by Gold [33] on language learning, and the pioneering work by Shapiro [57]
% on inductive synthesis and its application to algorithmic debugging among
% others.''



