\subsection{Enumerative Search}
\label{sec:enumerative-search}

In the context of program synthesis, enumerative search consists of enumerating
programs by working the intrinsic structure of the program space in order to to
guide the search.
The programs can be ordered using many different program metrics, the simplest
one being program size, and pruned by means of semantic equivalence checks with
respect to the specification.
Perhaps surprisingly, synthesizers based on enumerative search have been some of
the most effective to synthesize short programs in complex program spaces.
A reason why is that the search can be precisely tailored for the domain at
hand, encoding domain-specific heuristics and case-by-case scenarios that
result in highly effective pruning strategies.

In their overview of the field of program synthesis~\cite{Gulwani2017},
\citeauthor{Gulwani2017} describe some enumerative search algorithms for finding
programs in program spaces defined by a \Gls{cfg}, which we describe next.
% The algorithms can be generalized to other types of program spaces such as,
% e.g., sketches.

\subsubsection{Top-Down Tree Search}
\label{sec:top-down-tree-search}

\begin{algorithm}
  \DontPrintSemicolon
  \LinesNotNumbered

  \SetKw{Not}{not}

  \SetKwInOut{Input}{input}
  \SetKwInOut{Output}{output}

  \SetKwFunction{Queue}{Queue}
  \SetKwFunction{PopFirst}{popFirst}
  \SetKwFunction{NonTerminals}{nonTerminals}
  \SetKwFunction{Subsumed}{subsumed}

  \Input{A specification $\phi{}$ and a \gls{cfg} $G$}
  \Output{A program $p$ in the grammar $G$ that satisfies $\phi{}$}
  \Begin{
    $P\leftarrow$\Queue{}\;
    $P'\leftarrow \{S\}$\;

    \While{$P\neq \emptyset$}{
      $p\leftarrow$\PopFirst{$P$}\;
      \If{$p$ satisfies $\phi{}$}{
        \Return{$p$}\;
      }
      \For{$a \in$ \NonTerminals{$p$}}{
        \For{$b \in \{b | (a,b) \in R\}$}{
          \If{\Not \Subsumed{$p[a\rightarrow b]$, $P'$}}{
            $P\leftarrow P \cup \{p[a\rightarrow b]\}$\;
            $P'\leftarrow P' \cup \{p[a\rightarrow b]\}$\;
          }
        }
      }
    }
  }
  \caption{Enumerative Top-Down Tree Search.
    Adapted from \citeauthor{Gulwani2017}'s overview~\cite{Gulwani2017}.}
  \label{alg:enum-top-down}
\end{algorithm}

The first enumerative strategy is the top-down tree search algorithm
(Algorithm~\ref{alg:enum-top-down}).
It takes as input a \gls{cfg} $G = (V, \Sigma{}, R, S)$ and a specification
$\phi{}$, and works by exploring the derivations of $G$ in a best-first top-down
fashion.
The algorithm stores the current programs in a priority queue, $P$, and stores
all the programs found so far in the set $P'$.
Both $P$ and $P'$ are initialized with the partial program that corresponds to
the start symbol $S$ of $G$.
The algorithm runs until it finds a program $p$ that matches the specification
$\phi{}$, or until there are no more programs waiting in the queue (meaning that
the algorithm fails).
At every iteration, we take the program $p$ with the highest priority from the
queue and check whether it satisfies $\phi{}$.
If yes, we return $p$.
Otherwise, the algorithm finds new (possibly partial) programs by applying the
production rules of the grammar to $p$.
The program space is pruned in the next step by ignoring programs that are
semantically equivalent (with respect to $\phi{}$) to programs already
considered in the past (i.e., subsumed within $P'$).

\subsubsection{Bottom-Up Tree Search}
\label{sec:bottom-up-tree-search}

\begin{algorithm}
  \DontPrintSemicolon
  \LinesNotNumbered

  \SetKw{Not}{not}

  \SetKwInOut{Input}{input}
  \SetKwInOut{Output}{output}

  \SetKwFunction{EnumerateExprs}{enumerateExprs}
  \SetKwFunction{Subsumed}{subsumed}

  \Input{A specification $\phi{}$ and a \gls{cfg} $G$}
  \Output{A program $p$ in the grammar $G$ that satisfies $\phi{}$}
  \Begin{
    $P\leftarrow \emptyset$\;
    \For{progSize $= 1,2,\ldots$}{
      $P'\leftarrow$\EnumerateExprs{$G$, $E$, progSize}\;
      \For{$p \in P'$}{
        \If{$p$ satisfies $\phi{}$}{
          \Return{$p$}\;
        }
        \If{\Not \Subsumed{$p$, $P$}}{
          $P\leftarrow P \cup \{p\}$
        }
      }
    }
  }
  \caption{Enumerative Bottom-Up Tree Search.
    Adapted from \citeauthor{Gulwani2017}'s overview~\cite{Gulwani2017}.}
  \label{alg:enum-bottom-up}
\end{algorithm}

The bottom-up tree search algorithm (Algorithm~\ref{alg:enum-bottom-up}) is dual
to the top-down tree search algorithm.
It also takes a \gls{cfg} $G = (V, \Sigma{}, R, S)$ and a specification
$\phi{}$, and works by exploring the derivations of the grammar in a bottom-up
dynamic programming fashion.
This strategy has the advantage over the top-down search that (in general) only
complete programs may be evaluated for semantic equivalence.
The algorithm maintains a set of equivalent expressions, first considering the
programs corresponding to leafs of the syntax tree of the grammar $G$, and then
composing them in order to build expressions of increasing complexity,
essentially applying the rules of the grammar in the opposite direction.

\subsubsection{Bidirectional Tree Search}
\label{sec:bidirectional-search}

We can see that a top-down tree search starts from a set of input states, while
a bottom-up tree search starts from a set of output states.
In both approaches the size of the search space grows exponentially with the
size of the programs.
The bidirectional tree search algorithm tries to attenuate this problem by
combining the previous two approaches, starting from both a set of input states
and a set of output states.
It maintains both sets, evolving in the same way as the previous two
algorithms, and stops when it finds a state that belongs to both sets in a
sort of meet-in-the-middle approach.
