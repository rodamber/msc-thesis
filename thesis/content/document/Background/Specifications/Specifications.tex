\section{Specifications}
\label{sec:specifications}

The first part of solving a program synthesis problem is figuring out how the
user will communicate their intention to the synthesizer. An intention is
communicated by a \textit{specification}, and may be given in many different
ways, including: logical specifications~\cite{Itzhaky:SIS:2010} \todo{such
as}{maybe add first-order logic?} type signatures~\cite{Osera:2015:TPS,
Frankle:2016:EST, Polikarpova:2016:PSP}; \todo{syntax-guided methods}{add more
refs}~\cite{Alur:sygus:2013} such as sketches~\cite{Solar-Lezama:2008}, or
components~\cite{Feng:2017:CST, Feng:2017:CSC, Feng:2018:PSU, Shi:2019:FCS};
inductive specifications such as input-output examples~\cite{Frankle:2016:EST,
Gulwani:2012:SDM, Leung:2015:IPS}, demonstrations~\cite{Lau2003}, or program
traces~\cite{Lau:traces:2003}; or even \todo{other programs}{add ref}. The kind
of specification should be chosen according to the particular use case and to
the background of the user, and it might dictate the type of techniques used to
solve the problem (section~\ref{sec:search-techniques}).

\subsection{Logical Specifications}
\label{sec:logical}

Logical specifications are the canonical way of introducing specifications. In
the sorting example from the introduction (\ref{sec:sorting-example}) we already
saw an example of this where the specifications were given as logical
pre/post-conditions over the inputs/outputs of the program. In that case, the
specifications were written as predicates in the host programming language.
Logical specifications may also be given as loop invariants or general
assertions in the code, in order to give more hints to the
synthesizer.
Complete logical specifications are often difficult to write, or expensive to
synthesize from.
They usually require a great deal of knowledge about the domain of operation,
and are typically not suitable for non-technical users.

\input{SyntacticBias}
\subsection{Inductive Synthesis}
\label{sec:inductive}

Inductive synthesis is an instance of the program synthesis problem where the
constraints are under-specified.

\subsubsection{Programming By Demonstration}

\pdfmarkupcomment[color=red]{Incomplete}{}

Lau et al. define it as "inferring generalized actions based on examples of the
state changes resulting from demonstrated actions".

Number of traces needed must be small in order to be practical.

Lau et al. applied VSA to PBD by implementing SMARTedit, a system that induces
repetitive text-editing programs from as few as one or two examples.

\subsubsection{Execution Traces}

\pdfmarkupcomment[color=red]{Incomplete}{}

In \cite{Lau:traces:2003} Lau et. al apply version space algebra to infer
procedural programs from execution traces.
They present a language-neutral framework and implementation of a system that
``correct programs from a remarkably short 5.1 traces on average''.

\subsubsection{Programming By Examples}

\pdfmarkupcomment[color=green]{...}{Should I put the PBE section here instead?}


%%% Local Variables:
%%% mode: latex
%%% TeX-master: "../../Thesis"
%%% End:

\subsection{Programs}
\label{sec:programs}

A program can also be used as a specification and the job of the synthesizer is
then to find another program with the same semantics. We might also be
interested in programs that behave the same way as the original one, but, for
example, are more efficient (according to some metric).

Typically, the original programs are not made to be efficient, but to be easy to
read or to prove correct. They may also appear naturally as specifications in
certain use cases, such as
superoptimization~\cite{Phothilimthana:2016:SUS},
deobfuscation~\cite{Jha:oracle:2010} and
synthesis of program inverses~\cite{Srivastava:2011:PIS}.

