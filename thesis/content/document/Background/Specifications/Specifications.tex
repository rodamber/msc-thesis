\section{Specifications}
\label{sec:specifications}

The first part of solving a program synthesis problem is figuring out how the
user will communicate their intention to the synthesizer. An intention is
communicated by a \textit{specification}, and may be given in many different
ways, including: logical specifications~\cite{Itzhaky:SIS:2010} \todo{such
as}{maybe add first-order logic?} type signatures~\cite{Osera:2015:TPS,
Frankle:2016:EST, Polikarpova:2016:PSP}; \todo{syntax-guided methods}{add more
refs}~\cite{Alur:sygus:2013} such as sketches~\cite{Solar-Lezama:2008}, or
components~\cite{Feng:2017:CST, Feng:2017:CSC, Feng:2018:PSU, Shi:2019:FCS};
inductive specifications such as input-output examples~\cite{Frankle:2016:EST,
Gulwani:2012:SDM, Leung:2015:IPS}, demonstrations~\cite{Lau2003}, or program
traces~\cite{Lau:traces:2003}; or even \todo{other programs}{add ref}. The kind
of specification should be chosen according to the particular use case and to
the background of the user, and it might dictate the type of techniques used to
solve the problem (section~\ref{sec:search-techniques}).

\subsection{Logical Specifications}
\label{sec:logical}

Logical specifications are the canonical way of introducing specifications. In
the sorting example from the introduction (\ref{sec:sorting-example}) we already
saw an example of this where the specifications were given as logical
pre/post-conditions over the inputs/outputs of the program. In that case, the
specifications were written as predicates in the host programming language.
Logical specifications \todo{may also be given as loop invariants or general
assertions in the code}{refs. leon?}, in order to give more hints to the
synthesizer.

\subsection{Syntax-Guided Synthesis}
\label{sec:sygus}

% \fixme{Not sure if this should be the first section in the chapter.}{}

Syntax-guided synthesis is an instance of program synthesis where the semantic
specification is complemented with some form of syntactic restriction over the
program candidates.

This approach was introduced in the \todo{SKETCH}{Explain its
shortcomings} system by Solar-Lezama \cite{Solar-Lezama:2008}, which allowed the
synthesis of imperative programs in a C-like language.
It was then generalized and formalized with the objective of ``formulating the
core computational problem common to many recent tools for program synthesis in
a canonical and logical manner'' \cite{Alur:sygus:2013}.

The syntactic restrictions are typically provided in the form of a context-free
grammar \cite{Alur:sygus:2013} or with sketches \cite{Solar-Lezama:2008}.
These restrictions provide structure to the set of candidate programs, possibly
resulting in more efficient search procedures, while making this problem
suitable for \todo{machine learning and inductive inference}{missing
references} \cite{Alur:sygus:2013}. They can also be used for the purpose of
performance optimizations, e.g., by limiting the search space to implementations
that only use a limited amount of lines of code.
The learned programs also tend to be more readable and explainable.

% ------------------------------------------------------------------------------

\subsubsection{Sketching and Metasketching}
\label{sec:sketching}

% \todo{This subsubsection was left kind of dull after the modifications... Maybe
% it should be removed?}{}

The idea of sketching is to provide skeletons of the programs we want to
synthesize, called \textit{sketches}, leaving missing details, called
\textit{holes}, for the synthesizer to fill.
The synthesizer is then directed by the high-level structure of the skeleton
while taking care of finding the low-level details according to user-specified
assertions.

Sketching is an accessible form of program synthesis, as it does not require
learning new specification languages, allowing the users to use the programming
model with which they are already familiarized.

% \textit{Metasketches} \cite{Bornholt:2016:OSM} generalize sketches.

% Instead of just one sketch, a metasketch represents an 

% enabling a description of an infinite space of candidate programs with a
% countable, ordered set of finite sketches.

% This representation permits fine-grained control over the shape of the candidate
% space, which is critical for effective search.

% A metasketch additionally provides a means of assigning cost to programs and of
% directing the search toward lower-cost regions of the candidate space.

% \todo{Check out PSKETCH}{\cite{Gulwani2017}: page 32}

% TODO: From now on, refer to ''specifications and syntactic bias`' simply as
% ''specifications.`'
% FIXME: Maybe change this to ''incomplete specifications``?
\subsection{Inductive Synthesis}
\label{sec:inductive}

Inductive synthesis is an instance of the program synthesis problem where the
constraints are under-specified.

\subsubsection{Programming By Demonstration}

\pdfmarkupcomment[color=red]{Incomplete}{}

Lau et al. define it as "inferring generalized actions based on examples of the
state changes resulting from demonstrated actions".

Number of traces needed must be small in order to be practical.

Lau et al. applied VSA to PBD by implementing SMARTedit, a system that induces
repetitive text-editing programs from as few as one or two examples.

\subsubsection{Execution Traces}

\pdfmarkupcomment[color=red]{Incomplete}{}

In \cite{Lau:traces:2003} Lau et. al apply version space algebra to infer
procedural programs from execution traces.
They present a language-neutral framework and implementation of a system that
``correct programs from a remarkably short 5.1 traces on average''.

\subsubsection{Programming By Examples}

\pdfmarkupcomment[color=green]{...}{Should I put the PBE section here instead?}

%%% Local Variables:
%%% mode: latex
%%% TeX-master: "../../Thesis"
%%% End:

\subsection{Programs}
\label{sec:programs}

A program can also be used as a specification and the job of the synthesizer is
then to find another program with the same semantics. We might also be
interested in programs that behave the same way as the original one, but, for
example, are more efficient (according to some metric).

Typically, the original programs are not made to be efficient, but to be easy to
read or to prove correct. They may also appear naturally as specifications in
certain use cases, such as
superoptimization~\cite{Phothilimthana:2016:SUS},
deobfuscation~\cite{Jha:oracle:2010} and
synthesis of program inverses~\cite{Srivastava:2011:PIS}.

