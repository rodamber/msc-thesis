\subsection{Logical Specifications}
\label{sec:logical}

Logical specifications are the canonical way of introducing specifications. In
the sorting example from the introduction (\ref{sec:sorting-example}) we already
saw an example of this where the specifications were given as logical
pre/post-conditions over the inputs/outputs of the program. In that case, the
specifications were written as predicates in the host programming language.
Logical specifications may also be given as loop invariants or general
assertions in the code, in order to give more hints to the
synthesizer.
Complete logical specifications are often difficult to write, or expensive to
synthesize from.
They usually require a great deal of knowledge about the domain of operation,
and are typically not suitable for non-technical users.
