\subsection{Syntactic Specifications}
\label{sec:syntactic-specifications}

A specification can be seen as a constraint over the space of all possible
programs. We have already seen one type of specification, logical
specifications, which are a kind of \textit{semantic} specification, meaning
that they constrain the program space over \textit{behaviour}. Syntactic
specifications are a method of specifying intent in program synthesis where a
semantic specification (like logical specifications) is complemented with some
form of \textit{syntactic} constraints on the shape that the desired program can
take.

Syntactic specifications are typically provided in the form of a context-free
grammar \cite{Alur:sygus:2013}, or with sketches \cite{Solar-Lezama:2008}. These
restrictions provide structure to the set of candidate programs, possibly
resulting in more efficient search procedures. They can also be used for the
purpose of performance optimizations (of the desired program), e.g., by limiting
the search space to implementations that only use a limited amount of lines of
code. The learned programs also tend to be more readable and explainable.

\subsubsection{Sketching}
\label{sec:sketching}

The idea of sketching is to provide skeletons of the programs we want to
synthesize, called \textit{sketches}, leaving missing details, called
\textit{holes}, for the synthesizer to fill. The synthesizer is then directed by
the high-level structure of the skeleton while taking care of finding the
low-level details according to user-specified assertions.

Sketching is an accessible form of program synthesis, as it does not require
learning new specification languages and formalizations, allowing the users to
use the programming model with which they are already familiarized.

This approach was introduced in the SKETCH system by
Solar-Lezama~\cite{Solar-Lezama:2008}, which allowed the synthesis of imperative
programs in a C-like language.

\subsubsection{Component-Based Synthesis}
\label{sec:components}

In component-based
synthesis~\cite{Shi:2019:FCS,Feng:2018:PSU,Feng:2017:CST,Feng:2017:CSC,Jha:oracle:2010}.
we are interested in finding a loop-free program made out of a combination of
fundamental building blocks called \textit{components}. These components could
be, for example, methods in a library
\gls{api}~\cite{Shi:2019:FCS,Feng:2017:CSC}, and the way in which they can be
combined forms the syntactic specification for the programs we want to find.
They may also be supplemented by additional constraints in the form of logical
formulas~\cite{Feng:2018:PSU}.

\subsubsection{Syntax-Guided Synthesis}
\label{sec:sygus}

The problem of program synthesis with syntactic specifications was generalized
and formalized in the work on
\gls{sygus}~\cite{Alur:sygus:2013}\footnote{\url{https://sygus.org}}.
\gls{sygus} is a community effort with the objective of ``formulating the core
computational
problem common to many recent tools for program synthesis in a canonical and
logical manner''. The input to this problem consists of a background theory,
that defines the language, a semantic correctness specification defined by a
logical formula in that theory, and a syntactic specification in the form of a
\gls{cfg}.

This effort has helped to create a common format for the definition of program
synthesis problems and a growing repository of benchmarks. It has also led to
the creation of the SyGuS-Comp annual competition.