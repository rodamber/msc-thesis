\subsection{Inductive Synthesis}
\label{sec:inductive}

Inductive synthesis is an instance of the program synthesis problem where the
constraints are underspecified.
Sometimes the domain we want to model is complex enough that a complete
specification could be as hard to produce as the program itself, or might not
even exist.
In other cases, we might want the synthesizer to be as easy and intuitive to use
as possible, e.g., for users coming from non-technical backgrounds.

\subsubsection{\Glsfmtfull{pbe}}

\gls{pbe} is an instance of inductive synthesis where the specification is given
by input-output examples that the desired program must satisfy.
Explicitly giving examples can be preferred due to their ease of use, especially
by non-programmers, when compared to more technical kinds of specification, such
as logical formulas.
The examples may be either \textit{positive}, i.e., an example that the desired
program must satisfy, or \textit{negative}, i.e., an example that the desired
should \textit{not} satisfy.
More generally, given some (implicit) input-output example, we may include asserting
properties of the output instead of specifying it completely
\cite{Polozov:2015:FFI}.
This can be helpful if it is impractical or impossible
to write the output concretely, e.g., if it is infinite.

\subsubsection{\Glsfmtfull{pbd}}

In \gls{pbd} the user does not write a specification \textit{per se};
instead the synthesizer is given a sequence of transformation steps (a
demonstration) on concrete inputs, and uses them to infer the intended program.
The program must be general enough to be used with different inputs.
\gls{pbd} can be seen as a refinement of \gls{pbe} that
considers an entire execution trace (i.e., step-by-step instructions of the
program behavior on a given input) instead of a single input-output example.
It depicts \textit{how} to achieve the corresponding output instead of just
specifying \textit{what} it should be.

Though the concept of \gls{pbd} is easy to understand, the task of the user can
be tedious and time-consuming. Therefore, the synthesizer must be able to infer
the intended program from a small set of user demonstrations.
Ideally, it would also be able to interact effectively and receive feedback from
the user.
However, the concept might also be interesting when applied to non-interactive
contexts, such as \textit{reverse engineering}.

Lau et al. applied \gls{pbd} to the text-editing domain by implementing
SMARTedit, a system that induces repetitive text-editing programs from as few as
one or two examples. The system resembles familiar keystroke-based macro
interfaces, but it generalizes to a more robust program that is likely to work
in more situations \cite{Lau2003}.
They have also presented a language-neutral framework and an implementation of a
system that learns procedural programs from just 5.1 traces on average
\cite{Lau:traces:2003}.

\subsubsection{Ambiguity}
\label{sec:ambiguity}

In inductive synthesis the specifications are inherently ambiguous.
Therefore, the intended program should not only satisfy the specifications,
but also generalize, effectively trying to figure out the user's intention.
Typically, two approaches have been used to solve this problem.

The first approach works by \textit{ranking} the set of programs consistent with
the examples according to their likelihood of being the desired program.
This ranking function should allow for the efficient identification of the
top-ranked program without having to perform costly enumeration.
There have been manual approaches to create such functions, but it is a
time-consuming process that requires a lot of domain expertise.
It is also a fragile approach because it depends too much on the underlying
\gls{dsl}.
Recently, more automated approaches have been proposed~\cite{Singh:ranking:2015,
  Ramsey:2017:LTL}, usually relying on machine learning techniques.
However, as is usual with machine learning techniques, they require large
labeled training datasets.

The second approach is called \textit{active learning}, and (usually) relies
on interaction between the user and the synthesizer.
Typically, this happens by asking the user for (a small set of) additional
input-output examples.
Another idea, introduced by \citeauthor{Jha:oracle:2010}~\cite{Jha:oracle:2010},
works by finding a \textit{distinguishing input}, an input on which two
candidate programs differ, and query the user what the expected output should be
for the intended program.
Other types of active learning exist such as rephrasing the program in natural
language, or accepting negative examples~\cite{Frankle:2016:EST}.


