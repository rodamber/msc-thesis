\subsection{Programs}
\label{sec:programs}

A program can also be used as a specification, and the job of the synthesizer is
then to find another program with the same semantics.
We might also be interested in programs that behave the same way as the original
one, but, for example, are more efficient, or shorter.

Typically, the specification programs are not made to be efficient, but to be
easy to read, or to prove correct.
They may also appear naturally as specifications in certain use cases, such as
superoptimization~\cite{Phothilimthana:2016:SUS},
deobfuscation~\cite{Jha:oracle:2010}, and
synthesis of program inverses~\cite{Srivastava:2011:PIS}.
