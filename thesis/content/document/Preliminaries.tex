\chapter{Preliminaries}
\label{chapter:preliminaries}

% sat, smt, encoding programs

\section{General Principles}
\label{sec:general-principles}

% Overview: ch.1, pages 7-13
% user intent
% search space
% search technique

% % DSL design
% expressiveness
% choice of operators
% naturalness
% efficiency

% % Program ranking
% Program speed - superopt
% Robustness - pbe
% Naturalness and readability

\section{Version Space Algebra}
\label{sec:vsa}

\gls{vsa} is a data structure that allows for an efficient representation of a
very large space of program sets by exploiting sharing among program
subexpressions. \pdfmarkupcomment[color=yellow]{It also allows for efficient
operations on these sets such as:} {list and explain the operations}.

\pdfmargincomment{expand on this}
They were introduced by Lau et al. in \cite{Lau:2000} and applied in ... for ...
In particular, they were used in FlashFill...

% \gls{vsa} is a data structure which allows for a \pdfcomment{on the size of
% what?} polynomial space representation of sets of programs of exponential size.

