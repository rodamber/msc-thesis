%%%%%%%%%%%%%%%%%%%%%%%%%%%%%%%%%%%%%%%%%%%%%%%%%%%%%%%%%%%%%%%%%%%%%%
% Document preamble
%%%%%%%%%%%%%%%%%%%%%%%%%%%%%%%%%%%%%%%%%%%%%%%%%%%%%%%%%%%%%%%%%%%%%%

%% Builds upon the graphics  package, providing a key-value interface
%% for optional arguments to the \includegraphics command that go far
%% beyone what the graphics package offers.
%% http://www.ctan.org/tex-archive/help/Catalogue/entries/graphicx.html
%% if you use PostScript figures in your article
%% use the graphics package for simple commands
%% \usepackage{graphics}
%% or use the graphicx package for more complicated commands
%% \usepackage{graphicx}
%% or use the epsfig package if you prefer to use the old commands
%% \usepackage{epsfig}
\usepackage{graphicx} % Enhanced LaTeX Graphics

% Multiple figures
\usepackage{subfigure} % subcaptions for subfigures
\usepackage{subfigmat} % matrices of similar subfigures


% multirow
\usepackage{multirow}

% Declaring new column types
% 'dcolumn' package defines D to be a column specifier with
% three arguments: D{<sep.tex>}{<sep.dvi>}{<decimal places>}
%                  D{<sep.tex>}{<sep.dvi>}{<left digit places>.<right digit places>}
\usepackage{dcolumn}           % decimal-aligned tabular math columns
% d takes a single argument specifying the number of decimal places, e.g., d{2}
% or the number of digits to the left and right of the seperator, e.g., d{3.2}
\newcolumntype{.}   {D{.}{.}{-1}} % column alignedd on the point separator '.'
\newcolumntype{d}[1]{D{.}{.}{#1}} % column centered on the point separator '.'
\newcolumntype{e}   {D{E}{E}{-1}} % column centered on the exponent 'E'
\newcolumntype{E}[1]{D{E}{E}{#1}} % column centered on the exponent 'E'

%% American Mathematical Society (AMS) plain Tex macros
%%
%% The amsmath package is the principal package in the AMS-LaTeX distribution
%% http://www.ctan.org/tex-archive/help/Catalogue/entries/amsmath.html
\usepackage{amsmath}
%%
%% The amsfonts package provides extended TeX fonts
%% http://www.ctan.org/tex-archive/help/Catalogue/entries/amsfonts.html
\usepackage{amsfonts}
%% The amssymb package provides various useful mathematical symbols
\usepackage{amssymb}
%%
%% The amsthm package provides extended theorem environments
%% http://www.ctan.org/tex-archive/help/Catalogue/entries/amsthm.html
\usepackage{amsthm}

%% Improves the interface for defining floating objects such as figures and tables.
%% The package also provides the H float modifier option of the obsolete here package.
%% http://www.ctan.org/tex-archive/help/Catalogue/entries/float.html
\usepackage{float}

\usepackage{setspace}
\usepackage{natbib}
\usepackage[utf8]{inputenc}
\usepackage[T1]{fontenc}
% -----------------------------  algorithm2e -------------------------------------
% This package lets us display pseudocode in various fashions
\usepackage[linesnumbered,ruled,vlined]{algorithm2e}
\DontPrintSemicolon
\newcommand{\originalMathStyle}{\itshape\fontfamily{lmr}\selectfont}
\SetDataSty{originalMathStyle}
\SetArgSty{originalMathStyle}
% hack to have normal space in algorithms
\SetAlFnt{\setstretch{1.2}}
%
% Local Keywords
%\SetKwFunction{TransformCycle}{transformCycle}
%

%% Control sectional headers. 
%% http://www.ctan.org/tex-archive/help/Catalogue/entries/sectsty.html
\usepackage{sectsty}
%%
%% Redefine the font size of the 'section' and 'subsection' headings
\newcommand{\myFontSize}{\fontsize{10}{0}\selectfont}
\sectionfont{\myFontSize}       % 10pt, Bold face (default)
\subsectionfont{\rm\myFontSize} % 10pt, Plain face
\subsubsectionfont{\rm\myFontSize} % 10pt, Plain face

%% Select alternative section titles.
%% http://www.ctan.org/tex-archive/help/Catalogue/entries/titlesec.html
\usepackage{titlesec}
%%
%% Left indent, before and after spacing
%% (The starred version kills the indentation of the paragraph following the title)
\titlespacing*{\section}{0pt}{10pt}{0pt}
\titlespacing*{\subsection}{0pt}{10pt}{0pt}
\titlespacing*{\subsubsection}{0pt}{10pt}{0pt}

%% Section numbers with trailing dots. 
%% http://www.ctan.org/tex-archive/help/Catalogue/entries/secdot.html
\usepackage{secdot}
%%
%% Also put a dot after the subsection number
\sectiondot{subsection}
%% Set a space between dot and heading text
\sectionpunct{section}{. }    % By default, \sectiondot places a \quad
\sectionpunct{subsection}{. } % after the number
\sectionpunct{subsubsection}{. } % after the number

% These are exact settings for a A4 page with top margin of
% 25 mm, bottom margin of 30 mm, left and right margins of 25 mm,
% printable area 242 X 160 mm.

\setlength{\topmargin}{-10.4mm}
\setlength{\headheight}{0.0mm}
\setlength{\headsep}{10.0mm}
\setlength{\textwidth}{160mm}
\setlength{\textheight}{242mm}
\setlength{\oddsidemargin}{0mm}
\setlength{\evensidemargin}{0mm}
\setlength{\marginparwidth}{0mm}
\setlength{\marginparsep}{0mm}

% New command to refer to equations as Eq.(1),Eq.(2),...
\newcommand{\eqnref}[1]{Eq.(\ref{#1})}

% ==============================================================================
%                                  New Stuff

\usepackage{breakurl}
\usepackage[pdftex]{hyperref}       % enhance documents that are to be
                                    % output as HTML and PDF
\hypersetup{colorlinks,             % color text of links and anchors,
                                    % eliminates borders around links
            linkcolor=black,        % color for normal internal links
            anchorcolor=black,      % color for anchor text
            citecolor=black,        % color for bibliographical citations
            filecolor=black,        % color for URLs which open local files
            menucolor=black,        % color for Acrobat menu items
            pagecolor=black,        % color for links to other pages
            urlcolor=black,         % color for linked URLs
	          bookmarksopen=false,    % don't expand bookmarks
	          bookmarksnumbered=true, % number bookmarks
	          pdftitle={Thesis},
            pdfauthor={Rodrigo André Moreira Bernardo},
            pdfstartview=FitV,
            pdfdisplaydoctitle=true,
            pdfpagemode=UseOutlines
}

% Newtheorem
\newtheorem{definition}{Definition}
\newtheorem{example}{Example}
\numberwithin{example}{section}

% Listings
\usepackage{listings}
% \definecolor{dkgreen}{rgb}{0,0.6,0}
% \definecolor{gray}{rgb}{0.5,0.5,0.5}
% \definecolor{mauve}{rgb}{0.58,0,0.82}

\lstset{
  % frame=tb,
  % language=Haskell,
  % aboveskip=3mm,
  % belowskip=3mm,
  showstringspaces=false,
  columns=flexible,
  basicstyle={\ttfamily},
  numbers=none,
  numberstyle=\tiny\color{gray},
  keywordstyle=\color{blue},
  commentstyle=\color{dkgreen},
  stringstyle=\color{mauve},
  breaklines=true,
  breakatwhitespace=true,
  tabsize=3,
  literate={->}{$\rightarrow$}{2}
}

% Context-Free Grammars
\newcommand*\OR{\ |\ }

% Inline lists
\usepackage[inline, shortlabels]{enumitem}


% Document management
\usepackage{import}

\usepackage[acronym,toc]{glossaries-extra} % must be after hyperref
% \makeglossaries
\loadglsentries[main]{Glossary}

% Tikz
\usepackage{tikz}
\usetikzlibrary{arrows,backgrounds,positioning, fit,shapes}

% Nice, clean tables.
\usepackage{booktabs}

% Math binary operators
\newcommand{\strconcat}{\mathbin{++}}
