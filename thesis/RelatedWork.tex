\chapter{Related Work}
\label{chapter:relatedWork}

% Note that the reader might not read the document in the order it is presented.

\todo{Should ``program synthesis'' be capitalized?}
This chapter discusses the state of the art of the field of program synthesis.
Section~\ref{sec:challenges} presents the inherent challenges of the problem,
such as resolving ambiguity or the size of the search space;
\todo{Very ugly...}section~\ref{sec:specifications} presents the various forms
that specifications can take;
section~\ref{sec:applications} presents applications of the problem;
section~\ref{sec:search-techniques} presents techniques to prune the search
space;
section~\ref{sec:inductive-programming} presents program synthesis with
approximate specifications;
section~\ref{sec:pbe} refers to work done specifically in the area of
\ac{PBE}.

\section{Challenges}
\label{sec:challenges}

\section{Specifications}
\label{sec:specifications}

% syntactic bias, sketches, metasketches (sketch generation), templates, skeleton,
%   grammar, dsls, holes, SyGuS
% programming by example, input-output examples/constraints, inductive programming
% from execution traces
% another program
% logical relation

\section{Applications}
\label{sec:applications}

% flash fill, super-optimization, reverse engineering, etc
% data wrangling, filling incomplete programs

\section{Search Techniques}
\label{sec:search-techniques}

% deductive search
% transformation-based search
% enumerative search (top-down, bottom-up, bidirectional, offline exhaustive enumeration and composition)
% oracles (validation, correction, programs), CEGIS, OGIS, (universal) distinguishing inputs
% constraint-based search (component based, sketch generation/completion, solver-aided programming, ILP, )
% statistical search (machine learning, genetic programming, MCMC, sampling, probabilistic inference)
% Neural-guided search
% Graph neural networks
% PBE VSAs, deduction-based, inverse semantics, type-based, ambiguity, intent
% Active learning, interaction, distinguishing inputs

% meta-synthesis, rosette, prose, sketch, solver-aided programming, domain separation

% ranking approaches: MCMC, VSA, ML, Metasketches

\section{Inductive Programming}
\label{sec:inductive-programming}

This refers to any kind of program synthesis problem where the constraints are
under-specified. \todo{TODO}

\subsection{Programming by Examples}
\label{sec:pbe}
% PBE (inductive programming), examples, inverse semantics, VSA

\section{Synthesis of Data Transformation Programs}
\label{sec:data-trans-synth}
