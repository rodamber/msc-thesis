%%%%%%%%%%%%%%%%%%%%%%%%%%%%%%%%%%%%%%%%%%%%%%%%%%%%%%%%%%%%%%%%%%%%%%%%
%                                                                      %
%     File: Thesis_RelatedWork.tex                                     %
%     Tex Master: Thesis.tex                                           %
%                                                                      %
%%%%%%%%%%%%%%%%%%%%%%%%%%%%%%%%%%%%%%%%%%%%%%%%%%%%%%%%%%%%%%%%%%%%%%%%

\chapter{Related Work}
\label{chapter:relatedWork}


\section{Program Synthesis in General}


\subsection{Challenges and General Principles}

% Overview: ch.1, pages 7-13
% user intent
% search space
% search technique

% % DSL design
% expressivity
% choice of operators
% naturalness
% efficiency

% % Program ranking
% Program speed - superopt
% Robustness - pbe
% Naturalness and readability


\subsection{Specifications}

% syntactic bias, sketches, metasketches (sketch generation), templates, skeleton,
%   grammar, dsls, holes, SyGuS
% programming by example, input-output examples/constraints, inductive programming
% from execution traces
% another program
% logical relation


\subsection{Applications}

% flash fill, superoptimization, etc


\subsection{Search Techniques}

% deductive search
% transformation-based search
% enumerative search (top-down, bottom-up, bidirectional, offline exhaustive enumeration and composition)
% oracles (validation, correction, programs), CEGIS, OGIS, (universal) distinguishing inputs
% constraint-based search (component based, sketch generation/completion, solver-aided programming, ILP, )
% statistical search (machine learning, genetic programming, MCMC, sampling, probabilistic inference)
% Neural-guided search
% PBE VSAs, deduction-based, inverse semantics, type-based, ambiguity, intent
% Active learning, interaction, distinguishing inputs

% meta-synthesis, rosette, prose, sketch, solver-aided programming, domain separation

% ranking approaches: MCMC, VSA, ML, Metasketches


\section{Programming by Examples}

% PBE (inductive programming), examples, inverse semantics, VSA


\section{Synthesis of Data Transformation Programs}

% \subsection{How does it differ from Machine Learning?}


%%% Local Variables:
%%% mode: latex
%%% TeX-master: "Thesis"
%%% End:

