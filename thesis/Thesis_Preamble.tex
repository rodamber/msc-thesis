%%%%%%%%%%%%%%%%%%%%%%%%%%%%%%%%%%%%%%%%%%%%%%%%%%%%%%%%%%%%%%%%%%%%%%%%
%                                                                      %
%     File: Thesis_Preamble.tex                                        %
%     Tex Master: Thesis.tex                                           %
%                                                                      %
%     Author: Andre C. Marta                                           %
%     Last modified : 28 Feb 2014                                      %
%                                                                      %
%%%%%%%%%%%%%%%%%%%%%%%%%%%%%%%%%%%%%%%%%%%%%%%%%%%%%%%%%%%%%%%%%%%%%%%%

% ----------------------------------------------------------------------
% Define document language.
% ----------------------------------------------------------------------

% 'inputenc' package
%
% Accept different input encodings.
% http://www.ctan.org/tex-archive/macros/latex/base/
%
% > allows typing non-english text in LaTeX sources.
%
% ******************************* SELECT *******************************
%\usepackage[latin1]{inputenc} % <<<<< Windows
\usepackage[utf8]{inputenc}   % <<<<< Linux
% ******************************* SELECT *******************************


% 'babel' package
%
% Multilingual support for Plain TeX or LaTeX.
% http://www.ctan.org/tex-archive/macros/latex/required/babel/
%
% > sets the variable names according to the language selected
%
% ******************************* SELECT *******************************
%\usepackage[portuguese]{babel} % <<<<< Portuguese
\usepackage[english]{babel} % <<<<< English
% ******************************* SELECT *******************************


% List of LaTeX variable names: \abstractname, \appendixname, \bibname,
%   \chaptername, \contentsname, \listfigurename, \listtablename, ...)
% http://www.tex.ac.uk/cgi-bin/texfaq2html?label=fixnam
%
% Changing the words babel uses (uncomment and redefine as necessary...)
%
\newcommand{\acknowledgments}{@undefined} % new LaTeX variable name
%
% > English
%
\addto\captionsenglish{\renewcommand{\acknowledgments}{Acknowledgments}}
%\addto\captionsenglish{\renewcommand{\listtablename}{List of Tables}}
%\addto\captionsenglish{\renewcommand{\listfigurename}{List of Figures}}
%\addto\captionsenglish{\renewcommand{\nomname}{Nomenclature}}
%\addto\captionsenglish{\renewcommand{\glossaryname}{Glossary}}
%\addto\captionsenglish{\renewcommand{\acronymname}{List of Acronyms}}
%\addto\captionsenglish{\renewcommand{\bibname}{References}} % Bibliography
%\addto\captionsenglish{\renewcommand{\appendixname}{Appendix}}
% > Portuguese
%
\addto\captionsportuguese{\renewcommand{\acknowledgments}{Agradecimentos}}
%\addto\captionsportuguese{\renewcommand{\listtablename}{Lista de Figuras}}
%\addto\captionsportuguese{\renewcommand{\listfigurename}{Lista de Tabelas}}
\addto\captionsportuguese{\renewcommand{\nomname}{Lista de S\'{i}mbolos}} % Nomenclatura
%\addto\captionsportuguese{\renewcommand{\glossary}{Gloss\'{a}rio}}
%\addto\captionsportuguese{\renewcommand{\acronymname}{Lista de Abrevia\c{c}\~{o}es}}
%\addto\captionsportuguese{\renewcommand{\bibname}{Refer\^{e}ncias}} % Bibliografia
%\addto\captionsportuguese{\renewcommand{\appendixname}{Anexo}} % Apendice

\usepackage[printonlyused]{acronym}

% ----------------------------------------------------------------------
% Define default and cover page fonts.
% ----------------------------------------------------------------------

% Use Arial font as default
%
\renewcommand{\rmdefault}{phv}
\renewcommand{\sfdefault}{phv}

% Define cover page fonts
%
%         encoding     family       series      shape
%  \usefont{T1}     {phv}=helvetica  {b}=bold    {n}=normal
%                   {ptm}=times      {m}=normal  {sl}=slanted
%                                                {it}=italic
% see more examples at
% http://julien.coron.free.fr/languages/latex/fonts/
%
\def\FontLn{% 16 pt normal
  \usefont{T1}{phv}{m}{n}\fontsize{16pt}{16pt}\selectfont}
\def\FontLb{% 16 pt bold
  \usefont{T1}{phv}{b}{n}\fontsize{16pt}{16pt}\selectfont}
\def\FontMn{% 14 pt normal
  \usefont{T1}{phv}{m}{n}\fontsize{14pt}{14pt}\selectfont}
\def\FontMb{% 14 pt bold
  \usefont{T1}{phv}{b}{n}\fontsize{14pt}{14pt}\selectfont}
\def\FontSn{% 12 pt normal
  \usefont{T1}{phv}{m}{n}\fontsize{12pt}{12pt}\selectfont}


% ----------------------------------------------------------------------
% Define page margins and line spacing.
% ----------------------------------------------------------------------

% 'geometry' package
%
% Flexible and complete interface to document dimensions.
% http://www.ctan.org/tex-archive/macros/latex/contrib/geometry/
%
% > set the page margins (2.5cm minimum in every side, as per IST rules)
%
\usepackage{geometry}	
\geometry{verbose,tmargin=2.5cm,bmargin=2.5cm,lmargin=2.5cm,rmargin=2.5cm}

% 'setspace' package
%
% Set space between lines.
% http://www.ctan.org/tex-archive/macros/latex/contrib/setspace/
%
% > allow setting line spacing (line spacing of 1.5, as per IST rules)
%
\usepackage{setspace}
%\renewcommand{\baselinestretch}{1.5}
%\usepackage{setspace}
\setstretch{1.5}

% ----------------------------------------------------------------------
% Include external packages.
% Note that not all of these packages may be available on all system
% installations. If necessary, include the .sty files locally in
% the <jobname>.tex file directory.
% ----------------------------------------------------------------------

% 'graphicx' package
%
% Enhanced support for graphics.
% http://www.ctan.org/tex-archive/macros/latex/required/graphics/
%
% > extends arguments of the \includegraphics command
%
\usepackage{graphicx}
\usepackage{pdflscape}
\usepackage{lscape}
\usepackage{float}		%to add
%\floatstyle{boxed}		%box around
\restylefloat{figure}		%figure

% 'color' package
%
% Colour control for LaTeX documents.
% http://www.ctan.org/tex-archive/macros/latex/required/graphics/
%
% > defines color macros: \color{<color name>}
%
%\usepackage{color}


% 'amsmath' package
%
% Mathematical enhancements for LaTeX.
% http://www.ctan.org/tex-archive/macros/latex/required/amslatex/
%
% > American Mathematical Society plain Tex macros
%
\usepackage{amsmath}  % AMS mathematical facilities for LaTeX.
\usepackage{amsthm}   % Typesetting theorems (AMS style).
\usepackage{amsfonts} %
\usepackage{amssymb}

% ----------------------------- some useful packages -------------------------------------

%definitions on subsections
\newtheorem{theorem1}{Theorem}[section]
\numberwithin{theorem1}{section}
\newtheorem{subdef}[theorem1]{Definition}

%examples on subsections
\newtheorem{theorem2}{Theorem}[section]
\numberwithin{theorem2}{section}
\newtheorem{subexmp}[theorem2]{Example}

%propositions on subsections
\newtheorem{theorem3}{Theorem}[section]
\numberwithin{theorem3}{section}
\newtheorem{subprop}[theorem3]{Proposition} 

%
% ---------------------------------- subfigure ---------------------------------------
\usepackage{subfigure}


% 'wrapfig' package
%
% Produces figures which text can flow around.
% http://www.ctan.org/tex-archive/macros/latex/contrib/wrapfig/
%
% > wrap figures/tables in text (i.e., Di Vinci style)
%
% \usepackage{wrapfig}


% 'subfigure' package -> deprecated
%
% Deprecated: Figures divided into subfigures.
% http://www.ctan.org/tex-archive/obsolete/macros/latex/contrib/subfigure/
%
% > subcaptions for subfigures
%
%\usepackage{subfigure}


% 'subfigmat' package
%
% Automates layout when using the subfigure package.
% http://www.ctan.org/tex-archive/macros/latex/contrib/subfigmat/
%
% > matrices of similar subfigures
%
%\usepackage{subfigmat}

\usepackage{caption}

% 'url' package
%
% Verbatim with URL-sensitive line breaks.
% http://www.ctan.org/tex-archive/macros/latex/contrib/url/
%
% > URLs in BibTex
%
\usepackage{url}


% 'varioref' package
%
% Intelligent page references.
% http://www.ctan.org/tex-archive/macros/latex/required/tools/
%
% > smart page, figure, table and equation referencing
%
%\usepackage{varioref}


% 'dcolumn' package
%
% Align on the decimal point of numbers in tabular columns.
% http://www.ctan.org/tex-archive/macros/latex/required/tools/
%
% > decimal-aligned tabular math columns
%
\usepackage{dcolumn}
\newcolumntype{d}{D{.}{.}{-1}} % column aligned by the point separator '.'
\newcolumntype{e}{D{E}{E}{-1}} % column aligned by the exponent 'E'


% '' package
%
% Reimplementation of and extensions to LaTeX verbatim.
% http://www.ctan.org/tex-archive/macros/latex/required/tools/
%
% > provides the verbatim environment (\begin{verbatim},\end{verbatim})
%   and a comment environment (\begin{comment},  \end{comment})
%
% \usepackage{verbatim}


% 'moreverb' package
%
% Extended verbatim.
% http://www.ctan.org/tex-archive/macros/latex/contrib/moreverb/
%
% > supports tab expansion and line numbering
%
% \usepackage{moreverb}


%Appendix package
\usepackage{appendix}
\usepackage{longtable}

% 'nomencl' package
%
% Produce lists of symbols as in nomenclature.
% http://www.ctan.org/tex-archive/macros/latex/contrib/nomencl/
%
% The nomencl package makes use of the MakeIndex program
% in order to produce the nomenclature list.
%
% Nomenclature
% 1) On running the file through LATEX, the command \makenomenclature
%    in the preamble instructs it to create/open the nomenclature file
%    <jobname>.nlo corresponding to the LATEX file <jobname>.tex and
%    writes the information from the \nomenclature commands to this file.
% 2) The next step is to invoke MakeIndex in order to produce the
%    <jobname>.nls file. This can be achieved by making use of the
%    command: makeindex <jobname>.nlo -s nomencl.ist -o <jobname>.nls
% 3) The last step is to invoke LATEX on the <jobname>.tex file once
%    more. There, the \printnomenclature in the document will input the
%    <jobname>.nls file and process it according to the given options.
%
% http://www-h.eng.cam.ac.uk/help/tpl/textprocessing/nomencl.pdf
%
% Nomenclature (produces *.nlo *.nls files)
\usepackage{nomencl}
\makenomenclature
%
% Group variables according to their symbol type
%
\RequirePackage{ifthen} 
\ifthenelse{\equal{\languagename}{english}}%
    { % English
    \renewcommand{\nomgroup}[1]{%
      \ifthenelse{\equal{#1}{R}}{%
        \item[\textbf{Roman symbols}]}{%
        \ifthenelse{\equal{#1}{G}}{%
          \item[\textbf{Greek symbols}]}{%
          \ifthenelse{\equal{#1}{S}}{%
            \item[\textbf{Subscripts}]}{%
            \ifthenelse{\equal{#1}{T}}{%
              \item[\textbf{Superscripts}]}{}}}}}%
    }{% Portuguese
    \renewcommand{\nomgroup}[1]{%
      \ifthenelse{\equal{#1}{R}}{%
        \item[\textbf{Simbolos romanos}]}{%
        \ifthenelse{\equal{#1}{G}}{%
          \item[\textbf{Simbolos gregos}]}{%
          \ifthenelse{\equal{#1}{S}}{%
            \item[\textbf{Subscritos}]}{%
            \ifthenelse{\equal{#1}{T}}{%
              \item[\textbf{Sobrescritos}]}{}}}}}%
    }%


% 'glossary' package
%
% Create a glossary.
% http://www.ctan.org/tex-archive/macros/latex/contrib/glossary/
%
% Glossary (produces *.glo *.ist files)
\usepackage[number=none]{glossary}
% (remove blank line between groups)
\setglossary{gloskip={}}
% (redefine glossary style file)
%\renewcommand{\istfilename}{myGlossaryStyle.ist}
\makeglossary


% 'rotating' package
%
% Rotation tools, including rotated full-page floats.
% http://www.ctan.org/tex-archive/macros/latex/contrib/rotating/
%
% > show wide figures and tables in landscape format:
%   use \begin{sidewaystable} and \begin{sidewaysfigure}
%   instead of 'table' and 'figure', respectively.
%
\usepackage{rotating}


% 'hyperref' package
%
% Extensive support for hypertext in LaTeX.
% http://www.ctan.org/tex-archive/macros/latex/contrib/hyperref/
%
% > Extends the functionality of all the LATEX cross-referencing
%   commands (including the table of contents, bibliographies etc) to
%   produce \special commands which a driver can turn into hypertext
%   links; Also provides new commands to allow the user to write adhoc
%   hypertext links, including those to external documents and URLs.
%
\usepackage[pdftex]{hyperref} % enhance documents that are to be
                              % output as HTML and PDF
\hypersetup{colorlinks,       % color text of links and anchors,
                              % eliminates borders around links
%            linkcolor=red,    % color for normal internal links
            linkcolor=black,  % color for normal internal links
            anchorcolor=black,% color for anchor text
%            citecolor=green,  % color for bibliographical citations
            citecolor=black,  % color for bibliographical citations
%            filecolor=magenta,% color for URLs which open local files
            filecolor=black,  % color for URLs which open local files
%            menucolor=red,    % color for Acrobat menu items
            menucolor=black,  % color for Acrobat menu items
%            pagecolor=red,    % color for links to other pages
            pagecolor=black,  % color for links to other pages
%            urlcolor=cyan,    % color for linked URLs
            urlcolor=black,   % color for linked URLs
	          %bookmarks=true,         % create PDF bookmarks
	          bookmarksopen=false,    % don't expand bookmarks
	          bookmarksnumbered=true, % number bookmarks
	          pdftitle={Thesis},
            pdfauthor={Fernando César Silva Teixeira dos Santos},
            pdfsubject={Automatic Generation of Exercises for Massive Open Online Courses (MOOCs)},
            pdfkeywords={algorithm, graph, graph generator, education, MOOC},
            pdfstartview=FitV,
            pdfdisplaydoctitle=true}


% 'hypcap' package
%
% Adjusting the anchors of captions.
% http://www.ctan.org/tex-archive/macros/latex/contrib/oberdiek/
%
% > fixes the problem with hyperref, that links to floats points
%   below the caption and not at the beginning of the float.
%
\usepackage[figure,table]{hypcap}


% 'natbib' package
%
% Flexible bibliography support.
% http://www.ctan.org/tex-archive/macros/latex/contrib/natbib/
%
% > produce author-year style citations
%
% \citet  and \citep  for textual and parenthetical citations, respectively
% \citet* and \citep* that print the full author list, and not just the abbreviated one
% \citealt is the same as \citet but without parentheses. Similarly, \citealp is \citep without parentheses
% \citeauthor
% \citeyear
% \citeyearpar
%
% ******************************* SELECT *******************************
%\usepackage{natbib}          % <<<<< References in alphabetical list Correia, Silva, ...
\usepackage[numbers]{natbib} % <<<<< References in numbered list [1],[2],...
% ******************************* SELECT *******************************


% ----------------------------------------------------------------------
% Define new commands to assure consistent treatment throughout document
% ----------------------------------------------------------------------

\newcommand{\ud}{\mathrm{d}}                % total derivative
\newcommand{\degree}{\ensuremath{^\circ\,}} % degrees

% Abbreviations

\newcommand{\mcol}{\multicolumn}            % table format
\newcommand{\eqnref}[1]{(\ref{#1})}
\newcommand{\class}[1]{\texttt{#1}}
\newcommand{\package}[1]{\texttt{#1}}
\newcommand{\file}[1]{\texttt{#1}}
\newcommand{\BibTeX}{\textsc{Bib}\TeX}
\usepackage{multirow}

% Typefaces ( example: {\bf Bold text here} )
%
% > pre-defined
%   \bf % bold face
%   \it % italic
%   \tt % typewriter
%
% > newly defined
\newcommand{\tr}[1]{{\ensuremath{\textrm{#1}}}}   % text roman
\newcommand{\tb}[1]{{\ensuremath{\textbf{#1}}}}   % text bold face
\newcommand{\ti}[1]{{\ensuremath{\textit{#1}}}}   % text italic
\newcommand{\mc}[1]{{\ensuremath{\mathcal{#1}}}}  % math calygraphy
\newcommand{\mco}[1]{{\ensuremath{\mathcalold{#1}}}}% math old calygraphy
\newcommand{\mr}[1]{{\ensuremath{\mathrm{#1}}}}   % math roman
\newcommand{\mb}[1]{{\ensuremath{\mathbf{#1}}}}   % math bold face
\newcommand{\bs}[1]{\ensuremath{\boldsymbol{#1}}} % math symbol
\def\bm#1{\mathchoice                             % math bold
  {\mbox{\boldmath$\displaystyle#1$}}%
  {\mbox{\boldmath$#1$}}%
  {\mbox{\boldmath$\scriptstyle#1$}}%
  {\mbox{\boldmath$\scriptscriptstyle#1$}}}
\newcommand{\boldcal}[1]{{\ensuremath{\boldsymbol{\mathcal{#1}}}}}% math bold calygraphy



% ----------------------------- some useful packages -------------------------------------
% to mark math mode use \( and \) because is latex and provides better error message than $..$, which is tex

\usepackage{listings}           % for displaying code
\usepackage{enumerate}
%\usepackage{fixltx2e}           % allow \( inside headings and captions

\usepackage{pifont}             % some symbols from fonts
\usepackage{nameref}
\usepackage{soul}               % to striketrhourhg text
%\usepackage{subcaption}
%\usepackage{pdfpages}           % to include pfd files  
% -----------------------------  hyperref -------------------------------------
\usepackage[pdftex]{color}
\definecolor{citeblue}{rgb}{0.1,0,.4}
\definecolor{refcolor}{rgb}{0,0,0.4}
\definecolor{midgreen}{RGB}{0,150,0}
\definecolor{darkgreen}{RGB}{0,128,0}
\definecolor{darkblue}{RGB}{0,0,128}
\definecolor{darkred}{RGB}{192,0,0}
% \usepackage[pdftex%
% ,colorlinks=true%
% ,bookmarks=true%
% ,linkcolor=citeblue%
% ,citecolor=citeblue%
% ,urlcolor=blue%
% ,plainpages=false]{hyperref}

%
% -----------------------------  algorithm2e -------------------------------------
% This package lets us display pseudocode in various fashions
\usepackage[linesnumbered,ruled,vlined,algochapter]{algorithm2e}
\DontPrintSemicolon
\newcommand{\originalMathStyle}{\itshape\fontfamily{lmr}\selectfont}
\SetDataSty{originalMathStyle}
\SetArgSty{originalMathStyle}
% hack to have normal space in algorithms
\SetAlFnt{\setstretch{1.2}}
%
% Local Keywords
\SetKw{And}{and}
\SetKw{Break}{break}
\SetKw{Continue}{continue}
\SetKw{False}{false}
\SetKw{In}{in}
\SetKw{Make}{Make}
\SetKw{Nil}{NIL}
\SetKw{Or}{or}
\SetKw{Range}{range}
\SetKw{True}{true}

\SetKwData{AddableVertices}{addableVertices}
\SetKwData{Both}{both}
\SetKwData{CorrectOutput}{correct output}
\SetKwData{Cycle}{cycle}
\SetKwData{Density}{density}
\SetKwData{Graph}{graph}
\SetKwData{HasCycles}{has-cycles}
\SetKwData{Maxsp}{maxSp}
\SetKwData{Maxw}{maxW}
\SetKwData{Minsp}{minSp}
\SetKwData{Minw}{minW}
\SetKwData{Mvar}{m}
\SetKwData{Ncycles}{nc}
\SetKwData{Nc}{nc}
\SetKwData{NewPath}{newPath}
\SetKwData{Next}{Next}
\SetKwData{NoCycles}{no-cycles}
\SetKwData{Nvar}{n}
\SetKwData{Previous}{Previous}
\SetKwData{Random}{random-cycles}
\SetKwData{Reachability}{reachability}
\SetKwData{Reach}{reach}
\SetKwData{Rw}{rw}
\SetKwData{SavedV}{savedV}
\SetKwData{Seed}{seed}
\SetKwData{Sp}{sp}
\SetKwData{Status}{status}
\SetKwData{WeightCycles}{weight\&cycles}

\SetKwFunction{AssociateRandomWeights}{associateRandomWeights}
\SetKwFunction{AttributeWeights}{assignWeights}
\SetKwFunction{BFS}{BFS}
\SetKwFunction{BellmanFord}{Bellman-Ford}
\SetKwFunction{CreateNegativeCycle}{createNegativeCycle}
\SetKwFunction{DFS}{DFS}
\SetKwFunction{GraphGenerator}{GraphGenerator}
\SetKwFunction{IncrementalBFS}{incrementalBFS}
\SetKwFunction{Length}{length}
\SetKwFunction{NegativeValue}{negativeValue}
\SetKwFunction{NotEmpty}{!empty}
\SetKwFunction{Pushfront}{push-front}
\SetKwFunction{RandomBool}{randomBool}
\SetKwFunction{RandomDirection}{randomDirection}
\SetKwFunction{RandomEdge}{randomEdge}
\SetKwFunction{RandomReachableVertex}{randomReachableVertex}
\SetKwFunction{RandomSelect}{randomSelect}
\SetKwFunction{RandomVertex}{randomVertex}
\SetKwFunction{RandomWeight}{randomWeight}
\SetKwFunction{RandowReachableVertex}{randomReachableVertex}
\SetKwFunction{ReferenceSolution}{ReferenceSolution}
\SetKwFunction{Register}{register}
\SetKwFunction{StudentsSolution}{StudentsSolution}
\SetKwFunction{TransformCycle}{transformCycle}
%
% ---------------------------------- tikz ---------------------------------------
\usepackage{tikz}
\usetikzlibrary{arrows,automata,backgrounds,calc,chains,external,fit,graphs,patterns,positioning,shapes}

% predefining some styles for nodes and edges
%nodes
%\tikzstyle{bn}=[circle,fill=blue!10,draw=blue,thick,inner sep=1pt,%
%                minimum height=.5cm,minimum width=.5cm]
%\tikzstyle{gn}=[circle,fill=green!10,draw=green,thick,inner sep=1pt,%
%                minimum height=.5cm,minimum width=.5cm]
%\tikzstyle{on}=[circle,fill=orange!10,draw=orange,thick,inner sep=1pt,%
%                minimum height=.5cm,minimum width=.5cm]
%\tikzstyle{yn}=[circle,fill=yellow!10,draw=yellow,thick,inner sep=1pt,%
%                minimum height=.5cm,minimum width=.5cm]
    	
%edges
%\tikzstyle{be}=[very thick,blue!90]
%\tikzstyle{ge}=[very thick,green!90]
%\tikzstyle{oe}=[very thick,orange!90]
%\tikzstyle{ye}=[very thick,yellow!90]

%
%\tikzstyle{gns}=[rectangle,fill=green!10,draw=gray,thick,inner sep=2pt]
\tikzset{fancy/.style={rectangle,
		rounded corners=1mm,
		ultra thin,
		draw=white,
		top color=white,
		bottom color=black!20,
%		minimum height=.5cm,
%		minimum width=.5cm,
%		inner sep=2pt,
%		anchor=base,
		draw}}

% % Styles for graphs
\definecolor{highClr}{rgb}{1.0, 1.0, 0.0}


\colorlet{edgeClr}{orange!80!black}
% edges styles
\tikzset{sandEdge/.style={
		>=stealth,
		shorten >=1pt,
		thick,
		bend left,
		text=black,
		edgeClr,
	}}

\tikzset{fadedEdge/.style={
		->,
		>=stealth,
		shorten >=1pt,
		thick,
		edgeClr!20,
	}}

% labels styles
\tikzset{weigthLabel/.style={
		text=black,
		sloped,
		midway,
		}}

\tikzset{fadedWeigth/.style={
		text=lightgray!40,
		sloped,
		midway,
		anchor=south,
		}}

\tikzset{blueVertex/.style={
		% The shape:
		rectangle,minimum size=6mm,rounded corners=3mm,
		% The rest
		top color=white,bottom color=blue!35!cyan!25!,
		font=\ttfamily,
		text=black,
	}}

\tikzset{blueVertexG/.style={
		% The shape:
		rectangle,minimum size=6mm,rounded corners=3mm,
		% The rest
		top color=white,bottom color=blue!35!cyan!25!,
		font=\ttfamily,
		text=black,
		draw=green,
		thick,
	},
	blueVertexY/.style={
		% The shape:
		rectangle,minimum size=6mm,rounded corners=3mm,
		% The rest
		top color=white,bottom color=blue!35!cyan!25!,
		font=\ttfamily,
		text=black,
		draw=yellow,
		thick,
	},
	blueVertexO/.style={
		% The shape:
		rectangle,minimum size=6mm,rounded corners=3mm,
		% The rest
		top color=white,bottom color=blue!35!cyan!25!,
		font=\ttfamily,
		text=black,
		draw=orange,
		thick,
	}}
	

\colorlet{noteClr}{lightgray!30!white!50}

\tikzset{noteBckg/.style={
		rounded corners=8pt,fill=noteClr,
	},
	noteStl/.style={
		font=\scriptsize,
		align=center,
		text=black
	}}




% save tikz images to files (faster compilation)
%\tikzexternalize
% Default all images in the subfolder `temp/`
%\tikzsetexternalprefix{temp/}
%
% -------------------
\usepackage{pgfplots}          % to print charts
\pgfplotsset{compat=1.9}
\usepackage{pgfplotstable}     % to print tables
\usepackage{booktabs,colortbl} %needed by pgfplotstable
\usepackage{hhline}
%
% ------------------------------------------------------------------------------

%%% Local Variables:
%%% mode: latex
%%% TeX-master: "Thesis"
%%% End:

