% ----------------------------------------------------------------------
% Define document language.
% ----------------------------------------------------------------------

\usepackage[utf8]{inputenc}

\usepackage[english]{babel}
\newcommand{\acknowledgments}{@undefined}

% English
\addto\captionsenglish{\renewcommand{\acknowledgments}{Acknowledgments}}

% Portuguese
\addto\captionsportuguese{\renewcommand{\acknowledgments}{Agradecimentos}}
\addto\captionsportuguese{\renewcommand{\nomname}{Lista de S\'{i}mbolos}} % Nomenclatura

\usepackage[printonlyused]{acronym}

% ----------------------------------------------------------------------
% Define default and cover page fonts.
% ----------------------------------------------------------------------

% Use Arial font as default
\renewcommand{\rmdefault}{phv}
\renewcommand{\sfdefault}{phv}

% Define cover page fonts
\def\FontLn{% 16 pt normal
  \usefont{T1}{phv}{m}{n}\fontsize{16pt}{16pt}\selectfont}
\def\FontLb{% 16 pt bold
  \usefont{T1}{phv}{b}{n}\fontsize{16pt}{16pt}\selectfont}
\def\FontMn{% 14 pt normal
  \usefont{T1}{phv}{m}{n}\fontsize{14pt}{14pt}\selectfont}
\def\FontMb{% 14 pt bold
  \usefont{T1}{phv}{b}{n}\fontsize{14pt}{14pt}\selectfont}
\def\FontSn{% 12 pt normal
  \usefont{T1}{phv}{m}{n}\fontsize{12pt}{12pt}\selectfont}


% ----------------------------------------------------------------------
% Define page margins and line spacing.
% ----------------------------------------------------------------------

% > set the page margins (2.5cm minimum in every side, as per IST rules)
% > allow setting line spacing (line spacing of 1.5, as per IST rules)
\usepackage{geometry}
\geometry{verbose,tmargin=2.5cm,bmargin=2.5cm,lmargin=2.5cm,rmargin=2.5cm}
\usepackage{setspace}
\setstretch{1.5}

% % FIXME: Workaround to that todonotes fit in the margins.
% \usepackage[paperwidth=275.9mm, paperheight=279.4mm]{geometry}
% \setlength{\marginparwidth}{4cm}
% \usepackage[textwidth=4cm]{todonotes}


\usepackage{graphicx}
% \usepackage{pdflscape}
% \usepackage{lscape}
% \usepackage{float}
% \restylefloat{figure}

% \usepackage{amsmath}
% \usepackage{amsthm}
% \usepackage{amsfonts}
\usepackage{amssymb}

% \usepackage{subfigure}
% \usepackage{caption}
% \usepackage{url}

% \usepackage{dcolumn}
% \newcolumntype{d}{D{.}{.}{-1}} % column aligned by the point separator '.'
% \newcolumntype{e}{D{E}{E}{-1}} % column aligned by the exponent 'E'

%Appendix package
% \usepackage{appendix}
% \usepackage{longtable}

% 'nomencl' package
%
% Produce lists of symbols as in nomenclature.
% http://www.ctan.org/tex-archive/macros/latex/contrib/nomencl/
%
% The nomencl package makes use of the MakeIndex program
% in order to produce the nomenclature list.
%
% Nomenclature
% 1) On running the file through LATEX, the command \makenomenclature
%    in the preamble instructs it to create/open the nomenclature file
%    <jobname>.nlo corresponding to the LATEX file <jobname>.tex and
%    writes the information from the \nomenclature commands to this file.
% 2) The next step is to invoke MakeIndex in order to produce the
%    <jobname>.nls file. This can be achieved by making use of the
%    command: makeindex <jobname>.nlo -s nomencl.ist -o <jobname>.nls
% 3) The last step is to invoke LATEX on the <jobname>.tex file once
%    more. There, the \printnomenclature in the document will input the
%    <jobname>.nls file and process it according to the given options.
%
% http://www-h.eng.cam.ac.uk/help/tpl/textprocessing/nomencl.pdf
%
% Nomenclature (produces *.nlo *.nls files)
\usepackage{nomencl}
\makenomenclature
%
% Group variables according to their symbol type
%
\RequirePackage{ifthen} 
\ifthenelse{\equal{\languagename}{english}}%
    { % English
    \renewcommand{\nomgroup}[1]{%
      \ifthenelse{\equal{#1}{R}}{%
        \item[\textbf{Roman symbols}]}{%
        \ifthenelse{\equal{#1}{G}}{%
          \item[\textbf{Greek symbols}]}{%
          \ifthenelse{\equal{#1}{S}}{%
            \item[\textbf{Subscripts}]}{%
            \ifthenelse{\equal{#1}{T}}{%
              \item[\textbf{Superscripts}]}{}}}}}%
    }{% Portuguese
    \renewcommand{\nomgroup}[1]{%
      \ifthenelse{\equal{#1}{R}}{%
        \item[\textbf{Simbolos romanos}]}{%
        \ifthenelse{\equal{#1}{G}}{%
          \item[\textbf{Simbolos gregos}]}{%
          \ifthenelse{\equal{#1}{S}}{%
            \item[\textbf{Subscritos}]}{%
            \ifthenelse{\equal{#1}{T}}{%
              \item[\textbf{Sobrescritos}]}{}}}}}%
    }%


% Create a glossary (produces *.glo *.ist files).
\usepackage[number=none]{glossary}
\setglossary{gloskip={}} % Remove blank line between groups
\makeglossary

% % Rotation tools, including rotated full-page floats.
% % > show wide figures and tables in landscape format:
% %   use \begin{sidewaystable} and \begin{sidewaysfigure}
% %   instead of 'table' and 'figure', respectively.
% \usepackage{rotating}

\usepackage[pdftex]{hyperref}       % enhance documents that are to be
                                    % output as HTML and PDF
\hypersetup{colorlinks,             % color text of links and anchors,
                                    % eliminates borders around links
            linkcolor=black,        % color for normal internal links
            anchorcolor=black,      % color for anchor text
            citecolor=black,        % color for bibliographical citations
            filecolor=black,        % color for URLs which open local files
            menucolor=black,        % color for Acrobat menu items
            pagecolor=black,        % color for links to other pages
            urlcolor=black,         % color for linked URLs
	          bookmarksopen=false,    % don't expand bookmarks
	          bookmarksnumbered=true, % number bookmarks
	          pdftitle={Thesis},
            pdfauthor={Rodrigo André Moreira Bernardo},
            pdfstartview=FitV,
            pdfdisplaydoctitle=true,
            pdfpagemode=UseOutlines
}

% Flexible bibliography support.
\usepackage[numbers]{natbib}

% to mark math mode use \( and \) because is latex and provides better error message than $..$, which is tex

% \usepackage{listings}           % for displaying code
% \usepackage{enumerate}

% \usepackage{pifont}             % some symbols from fonts
% \usepackage{nameref}
% \usepackage{soul}               % to striketrhourhg text

% \usepackage[pdftex]{color}
% \definecolor{citeblue}{rgb}{0.1,0,.4}
% \definecolor{refcolor}{rgb}{0,0,0.4}
% \definecolor{midgreen}{RGB}{0,150,0}
% \definecolor{darkgreen}{RGB}{0,128,0}
% \definecolor{darkblue}{RGB}{0,0,128}
% \definecolor{darkred}{RGB}{192,0,0}

% This package lets us display pseudocode in various fashions
\usepackage[linesnumbered,ruled,vlined,algochapter]{algorithm2e}

% ----------------------------- Other -------------------------------------

\usepackage{lipsum}
\usepackage{outlines}

\usepackage{pdfcomment}
\usepackage{xcolor}

%%% Local Variables:
%%% mode: latex
%%% TeX-master: "Thesis"
%%% End:
